\documentclass[12pt]{article}

%%%% GRAPHICS %%%%
\usepackage{tikz}
\usepackage{circuitikz}
\usetikzlibrary{arrows.meta}
\usepackage{tikz-3dplot}
\usepackage{graphicx}
\usepackage{pgfplots}
  \pgfplotsset{compat=1.18}
\usetikzlibrary{arrows}
\newcommand{\midarrow}{\tikz \draw[-triangle 90] (0,0) -- +(.1,0);}

%%%% FIGURES %%%%
\usepackage{subcaption}
\usepackage{wrapfig}
\usepackage{float}
\usepackage[skip=5pt, font=footnotesize]{caption}

%%%% FORMATTING %%%%
\usepackage{parskip}
\usepackage{tcolorbox}
\usepackage{ulem}
% \usepackage{fancyhdr}

%%%% TABLE FORMATTING %%%%
\usepackage{tabularray}
\UseTblrLibrary{booktabs}

%%%% MATH AND LOGIC %%%%
\usepackage{xifthen}
\usepackage{amsmath}
\usepackage{amssymb}
\usepackage{amsfonts}

%%%% TEXT AND SYMBOLS %%%%
\usepackage[T1]{fontenc}
\usepackage{textcomp}
\usepackage{gensymb}

%%%% OTHER %%%%
\usepackage{standalone}

%%%% LOGIC SYMBOLS %%%%
\newcommand*\xor{\oplus}

%%%% STYLES %%%%

% Packages
\usepackage{fullpage}
\usepackage{titlesec}
\usepackage[rgb]{xcolor}
\selectcolormodel{natural}
\usepackage{ninecolors}
\selectcolormodel{rgb}

% Colors
\definecolor{pg}{HTML}{24273A}
\definecolor{fg}{HTML}{FFFFFF}
\definecolor{bg}{HTML}{24273A}
\definecolor{re}{HTML}{d20f39}
\definecolor{gr}{HTML}{40a02b}
\definecolor{ye}{HTML}{df8e1d}
\definecolor{or}{HTML}{fe640b}
\definecolor{bl}{HTML}{1e66f5}
\definecolor{ma}{HTML}{8839ef}
\definecolor{cy}{HTML}{179299}
\definecolor{pi}{HTML}{ea76cb}

\usepackage{nameref}
\makeatletter
\newcommand*{\currentname}{\@currentlabelname}
\makeatother

\titleformat{\section}
  {\normalfont\scshape\Large\bfseries}
  {\thesection}
  {0.75em}
  {}

\titleformat{\subsection}
  {\normalfont\scshape\large\bfseries}
  {\thesubsection}
  {0.75em}
  {}

\titleformat{\subsubsection}
  {\normalfont\scshape\normalsize\bfseries}
  {\thesubsubsection}
  {0.75em}
  {}

% Formula
\newcounter{formula}[section]
\newenvironment{formula}[1]{
  \stepcounter{formula}
  \begin{tcolorbox}[
    standard jigsaw, % Allows opacity
    colframe={fg},
    boxrule=1px,
    colback=bg,
    opacityback=0,
    sharp corners,
    sidebyside,
    righthand width=18px,
    coltext={fg}
  ]
  \centering
  \textbf{\uline{#1}}
}{
  \tcblower
  \textbf{\thesection.\theformula}
  \end{tcolorbox}
}

% Definition
\newcounter{definition}[section]

\newenvironment{definition*}[1]{
  \begin{tcolorbox}[
    standard jigsaw, % Allows opacity
    colframe={fg},
    boxrule=1px,
    colback=bg,
    opacityback=0,
    sharp corners,
    coltext={fg}
  ]
  \textbf{#1 \hfill}
  \vspace{5px}
  \hrule
  \vspace{5px}
  \noindent
}{
  \end{tcolorbox}
}

\newenvironment{definition}[1]{
  \stepcounter{definition}
  \begin{tcolorbox}[
    standard jigsaw, % Allows opacity
    colframe={fg},
    boxrule=1px,
    colback=bg,
    opacityback=0,
    sharp corners,
    coltext={fg}
  ]
  \textbf{#1 \hfill \thesection.\thedefinition}
  \vspace{5px}
  \hrule
  \vspace{5px}
  \noindent
}{
  \end{tcolorbox}
}

% Example Problem
\newcounter{example}[section]
\newenvironment{example}{
  \stepcounter{example}
  \begin{tcolorbox}[
    standard jigsaw, % Allows opacity
    colframe={fg},
    boxrule=1px,
    colback=bg,
    opacityback=0,
    sharp corners,
    coltext={fg}
  ]
  \textbf{Example \hfill \thesection.\theexample}
  \vspace{5px}
  \hrule
  \vspace{5px}
  \noindent
}{
  \end{tcolorbox}
}

\tikzset{
  cubeBorder/.style=fg,
  cubeFilling/.style={fg!20!bg, opacity=0.25},
  gridLine/.style={very thin, gray},
  graphLine/.style={-latex, thick, fg},
}

\pgfplotsset{
  basicAxis/.style={
    grid,
    major grid style={line width=.2pt,draw=fg!50!bg},
    axis lines = box,
    axis line style = {line width = 1px},
  }
}

%%%% REFERENCES %%%%
\usepackage{hyperref}
\hypersetup{
  colorlinks  = true,
  linkcolor   = pi,
  anchorcolor = pi,
  citecolor   = pi,
  filecolor   = pi,
  menucolor   = pi,
  runcolor    = pi,
  urlcolor    = pi,
}

\author{Ethan Anthony}


\title{Lecture 001}
\date{September 09, 2024}

\begin{document}

\section{Foundational Concepts}
\label{sec:foundationalConcepts}

\subsection{Analog and Digital}
\label{ssec:analogAndDigital}

The world we live in is \textbf{analog}, meaning signals and information are variable and
continuous. Digital systems, on the other hand, are represented by discrete values.

\begin{figure}[H]
  \centering
  \begin{subfigure}[H]{0.45\textwidth}
    \centering
    \includestandalone{figures/fig_001}
    \caption{Analog}
    \label{fig:001}
  \end{subfigure}
  \begin{subfigure}[H]{0.45\textwidth}
    \centering
    \includestandalone{figures/fig_002}
    \caption{Digital}
    \label{fig:002}
  \end{subfigure}
  \caption{Analog vs. Digital}
  \label{fig:analogVsDigital}
\end{figure}

Digital is used for many different reasons:
\begin{itemize}
  \itemsep0em
  \item Represent and manipulate discrete elements of information
  \item Perform numeric computations on information with digital computers
  \item Use digital systems to manipulate elements of information
  \item Signals as elements of information
  \item Two-valued (binary) number system
  \item Integrated Circuit (IC) technology
\end{itemize}

\subsection{Basic Definition}
\label{ssec:basicDefinition}

On the very basic level, there are a few definitions that will be used throughout the course.

\begin{definition}{Digital Systems}
  Manipulate \textbf{discrete elements} of information
\end{definition}

\begin{definition}{Hardware System}
  A system whose physical components are constructed from electronic building blocks or modules.
\end{definition}

\begin{definition}{Binary Digital System}
  Constrains the values of \textbf{digital} inputs and outputs to two distinct values (1s and 0s).
\end{definition}

\subsection{Positional Notation}
\label{ssec:positionalNotation}

\begin{definition}{Positional Notation}
  Used to represent a number as a position. In base ten, every value to the left of the 
  decimal point

  \begin{align*}
    N_r &= (a_n \cdot a_{n-1} ... a_1 \cdot a_0 . a_{-1} ... a_{-m})_r \\
    \textup{decimal value($N_r$)} &= \sum_{j=-m}^{n} (a_j \times r^j)
  \end{align*}

  Where:
  \begin{align*}
    r   &= \textup{is the \textbf{radix} or \textbf{base} of the number system} \\
    N_r &= \textup{is the number in base-\textbf{r}} \\
    a_j &= \textup{is a coefficient of the number with a decimal value in the range of $0$ to $(r-1)$} \\
    j   &= \textup{is the place (or position) of the digit}
  \end{align*}
\end{definition}

\begin{figure}[H]
  \centering
  \begin{tblr}{c|cccccccccc}
    \toprule
    \textbf{Base} ($r$)     & \SetCell[c=10]{c} $5$ \\
    \textbf{Number} ($N_r$) & \SetCell[c=10]{c} $4021.23_5$ \\
    \midrule
    \textbf{$N_r$}                                       &                                                                              \\
    \textbf{Position ($j$)}                              & 5     & 4     & 3     &     2 &     1 &     0 & . & -1     &     -2 & -3     \\
    \textbf{Digit ($a_j$)}                               & 0     & 0     & 4     &     0 &     2 &     1 & . & 2      &     3  & 0      \\
    \textbf{Position Value ($r^j$)}                      & $5^5$ & $5^4$ & $5^3$ & $5^2$ & $5^1$ & $5^0$ & . & $5^-1$ & $5^-2$ & $5^-3$ \\
    \textbf{Digit Position Value ($a_j \times r^j$)}     & 0     & 0     & 500   & 0     & 10    & 1     & . &    0.4 &   0.12 & 0      \\
    \textbf{Decimal Value ($\sum_{}^{} a_j \times r^j$)} & \SetCell[c=10]{c} $511.52$                                                   \\
    \bottomrule
  \end{tblr}
  \caption{Base Five Representation}
  \label{tbl:baseFiveRepresentation}
\end{figure}

\subsection{Binary Number System}
\label{ssec:binaryNumberSystem}

\begin{wrapfigure}[6]{r}{0.3\textwidth}
  \begin{tblr}{lcc}
    \toprule
    $2^10$ & \textbf{K} & Binary Kilo \\
    $2^20$ & \textbf{M} & Binary Mega \\
    $2^30$ & \textbf{G} & Binary Giga \\
    $2^50$ & \textbf{T} & Binary Tera \\
    \bottomrule
  \end{tblr}
  \caption{Special Powers of 2}
  \label{tbl:pwrtwo}
\end{wrapfigure}

Digital systems manipulate binary calues. The digits in a binary number are called \textbf{bits}
(\textbf{b}inary dig\textbf{its}). A group of 8 bits is called a \textbf{byte}.

When in the binary number system, some unit prefixes take on a different meaning. This can
be seen in Figure \ref{tbl:pwrtwo}.

\subsection{Number Base Conversion}
\label{ssec:numberBaseConversion}

Conversion of decimal numbers $\rightarrow$ base-r numbers.

\begin{enumerate}
  \itemsep0em
  \item Integer Conversion
    \begin{enumerate}
      \itemsep0em
      \item Divide the number by the base r; save the remainder and the quotient.
      \item Divide the quotient from the previous step by r; save the remainder and \textit{new} quotient.
      \item Repeat step (a) until the \textit{new} quotient is 0.
      \item The integer in base-r is the \textbf{remainders} written down in \textbf{reverse} order from which they were generated.
    \end{enumerate}
  \item Fractional Conversion
    \begin{enumerate}
      \itemsep0em
      \item Multiply the fractional portion of the number by $r$; save the result and the number to the left of hte decimal point
      \item Multiply the fractional portion the result from the previous step by $r$; save the result and the \textit{new} number to the left of the decimal point
      \item Repeat step (b) until the fractional portion is zero or the desired precision is reached.
      \item The fraction in base-$r$ is the digits save that were left of the decimal point in order.
    \end{enumerate}
\end{enumerate}

\begin{figure}[H]
  \centering
  \begin{subfigure}[b]{0.45\textwidth}
    \centering
    \begin{align*}
      47_{10} &= ?_2                    \\
      47 \div 2 &= 23 \textup{\ remainder \ } 1 \\
      23 \div 2 &= 11 \textup{\ remainder \ } 1 \\
      11 \div 2 &= 5  \textup{\ remainder \ } 1 \\
      5  \div 2 &= 2  \textup{\ remainder \ } 1 \\
      2  \div 2 &= 1  \textup{\ remainder \ } 0 \\
      1  \div 2 &= 0  \textup{\ remainder \ } 1 \\
      47_{10} &= 101111_2
    \end{align*}
    \caption{Integer Conversion}
    \label{fig:integerConversion}
  \end{subfigure}
  \begin{subfigure}[b]{0.45\textwidth}
    \centering
    \begin{align*}
      0.3_{10}     &= ?_2       \\
      0.3 \times 2 &= 0.6       \\
      0.6 \times 2 &= 1.2       \\
      0.2 \times 2 &= 0.4       \\
      0.4 \times 2 &= 0.8       \\
      0.8 \times 2 &= 1.6       \\
      0.3_{10}     &= 0.0\overline{1001}_2
    \end{align*}
    \caption{Fractional Conversion}
    \label{fig:fractionalConversion}
  \end{subfigure}
  \caption{Fractional and Integer Conversion}
  \label{fig:fractionalAndIntegerConversion}
\end{figure}

This method of converting between base systems works universally. Additionally, they can be
used in tandem to convert between a number with an integer and fraction part.

\subsection{Octal and Hexadecimal}
\label{ssec:octalAndHexadecimal}

These are very important within digital design. Notably, both octal (base 8) and hexadecimal (base 16)
are both powers of binary (base 2). When any number base is a power of another number base
a quicker conversion method can be used.

\begin{enumerate}
  \itemsep0em
  \item Starting from the binary point and proceeding to the left and to the right, partition
    the binary number into groups of $x$ bits, where $x$ is the power to which the lower base
    is raised to reach the higher base.
    \begin{itemize}
      \itemsep0em
      \item For example, converting from octal to binary, $x = 3$ since $2^3 = 8$.
    \end{itemize}
  \item Convert each group of $x$ bits into the equivalent binary representation.
\end{enumerate}

\end{document}
