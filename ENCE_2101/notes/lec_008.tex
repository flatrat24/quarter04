\documentclass[12pt]{article}

%%%% GRAPHICS %%%%
\usepackage{tikz}
\usepackage{circuitikz}
\usetikzlibrary{arrows.meta}
\usepackage{tikz-3dplot}
\usepackage{graphicx}
\usepackage{pgfplots}
  \pgfplotsset{compat=1.18}
\usetikzlibrary{arrows}
\newcommand{\midarrow}{\tikz \draw[-triangle 90] (0,0) -- +(.1,0);}

%%%% FIGURES %%%%
\usepackage{subcaption}
\usepackage{wrapfig}
\usepackage{float}
\usepackage[skip=5pt, font=footnotesize]{caption}

%%%% FORMATTING %%%%
\usepackage{parskip}
\usepackage{tcolorbox}
\usepackage{ulem}
% \usepackage{fancyhdr}

%%%% TABLE FORMATTING %%%%
\usepackage{tabularray}
\UseTblrLibrary{booktabs}

%%%% MATH AND LOGIC %%%%
\usepackage{xifthen}
\usepackage{amsmath}
\usepackage{amssymb}
\usepackage{amsfonts}

%%%% TEXT AND SYMBOLS %%%%
\usepackage[T1]{fontenc}
\usepackage{textcomp}
\usepackage{gensymb}

%%%% OTHER %%%%
\usepackage{standalone}

%%%% LOGIC SYMBOLS %%%%
\newcommand*\xor{\oplus}

%%%% STYLES %%%%

% Packages
\usepackage{fullpage}
\usepackage{titlesec}
\usepackage[rgb]{xcolor}
\selectcolormodel{natural}
\usepackage{ninecolors}
\selectcolormodel{rgb}

% Colors
\definecolor{pg}{HTML}{24273A}
\definecolor{fg}{HTML}{FFFFFF}
\definecolor{bg}{HTML}{24273A}
\definecolor{re}{HTML}{d20f39}
\definecolor{gr}{HTML}{40a02b}
\definecolor{ye}{HTML}{df8e1d}
\definecolor{or}{HTML}{fe640b}
\definecolor{bl}{HTML}{1e66f5}
\definecolor{ma}{HTML}{8839ef}
\definecolor{cy}{HTML}{179299}
\definecolor{pi}{HTML}{ea76cb}

\usepackage{nameref}
\makeatletter
\newcommand*{\currentname}{\@currentlabelname}
\makeatother

\titleformat{\section}
  {\normalfont\scshape\Large\bfseries}
  {\thesection}
  {0.75em}
  {}

\titleformat{\subsection}
  {\normalfont\scshape\large\bfseries}
  {\thesubsection}
  {0.75em}
  {}

\titleformat{\subsubsection}
  {\normalfont\scshape\normalsize\bfseries}
  {\thesubsubsection}
  {0.75em}
  {}

% Formula
\newcounter{formula}[section]
\newenvironment{formula}[1]{
  \stepcounter{formula}
  \begin{tcolorbox}[
    standard jigsaw, % Allows opacity
    colframe={fg},
    boxrule=1px,
    colback=bg,
    opacityback=0,
    sharp corners,
    sidebyside,
    righthand width=18px,
    coltext={fg}
  ]
  \centering
  \textbf{\uline{#1}}
}{
  \tcblower
  \textbf{\thesection.\theformula}
  \end{tcolorbox}
}

% Definition
\newcounter{definition}[section]

\newenvironment{definition*}[1]{
  \begin{tcolorbox}[
    standard jigsaw, % Allows opacity
    colframe={fg},
    boxrule=1px,
    colback=bg,
    opacityback=0,
    sharp corners,
    coltext={fg}
  ]
  \textbf{#1 \hfill}
  \vspace{5px}
  \hrule
  \vspace{5px}
  \noindent
}{
  \end{tcolorbox}
}

\newenvironment{definition}[1]{
  \stepcounter{definition}
  \begin{tcolorbox}[
    standard jigsaw, % Allows opacity
    colframe={fg},
    boxrule=1px,
    colback=bg,
    opacityback=0,
    sharp corners,
    coltext={fg}
  ]
  \textbf{#1 \hfill \thesection.\thedefinition}
  \vspace{5px}
  \hrule
  \vspace{5px}
  \noindent
}{
  \end{tcolorbox}
}

% Example Problem
\newcounter{example}[section]
\newenvironment{example}{
  \stepcounter{example}
  \begin{tcolorbox}[
    standard jigsaw, % Allows opacity
    colframe={fg},
    boxrule=1px,
    colback=bg,
    opacityback=0,
    sharp corners,
    coltext={fg}
  ]
  \textbf{Example \hfill \thesection.\theexample}
  \vspace{5px}
  \hrule
  \vspace{5px}
  \noindent
}{
  \end{tcolorbox}
}

\tikzset{
  cubeBorder/.style=fg,
  cubeFilling/.style={fg!20!bg, opacity=0.25},
  gridLine/.style={very thin, gray},
  graphLine/.style={-latex, thick, fg},
}

\pgfplotsset{
  basicAxis/.style={
    grid,
    major grid style={line width=.2pt,draw=fg!50!bg},
    axis lines = box,
    axis line style = {line width = 1px},
  }
}

%%%% REFERENCES %%%%
\usepackage{hyperref}
\hypersetup{
  colorlinks  = true,
  linkcolor   = pi,
  anchorcolor = pi,
  citecolor   = pi,
  filecolor   = pi,
  menucolor   = pi,
  runcolor    = pi,
  urlcolor    = pi,
}

\author{Ethan Anthony}


\title{Lecture 008}
\date{November 06, 2024}

\begin{document}

\newpage
\section{Sequential Circuit Synthesis}
\label{sec:sequentialCircuitSynthesis}

The steps to synthesizing a circuit from a description of its behavior are as follows:
\begin{enumerate}
  \itemsep0em
  \item Draw a \textbf{state diagram}
  \item Construct a \textbf{state/output table}, then assign unique binary values to each
    state and output in the table
  \item Create a \textbf{transition/output table}
  \item Choose a flip-flop
  \item Construct an \textbf{excitation table}
  \item Use K-Maps (or another method) to \textbf{derive the logic equations}
  \item Construct the circuit
\end{enumerate}

To illustrate this process more clearly, the following design specifications will be used
to derive a fully functioning circuit. The circuit will operate as a stop light rotating
through green, red, and yellow:
\begin{enumerate}
  \itemsep0em
  \item All changes in the circuit occur on the positive edge of a clock signal
  \item The circuit has three outputs ($r$, $y$, and $g$)
    \begin{enumerate}
      \itemsep-0.5em
      \item The output $r$ is equal to 1 if the previous output was $y$
      \item The output $y$ is equal to 1 if the previous output was $g$
      \item The output $g$ is equal to 1 if the previous output was $r$ or there is no
        previous output
    \end{enumerate}
\end{enumerate}
\begin{center}
  \begin{tblr}{r|cccccccccccccccccccc}
    \toprule
    Clockcycle & $t_{0}$ & $t_{1}$ & $t_{2}$ & $t_{3}$ & $t_{4}$ & $t_{5}$ & $t_{6}$ & $t_{7}$ & $t_{8}$ & $t_{9}$ & $t_{10}$ \\
    \midrule
    $r$        & 0 & 0 & 1 & 0 & 0 & 1 & 0 & 0 & 1 & 0 & 0 \\
    $y$        & 0 & 1 & 0 & 0 & 1 & 0 & 0 & 1 & 0 & 0 & 1 \\
    $g$        & 1 & 0 & 0 & 1 & 0 & 0 & 1 & 0 & 0 & 1 & 0 \\
    \bottomrule
  \end{tblr}
\end{center}

\subsection{State Diagram}
\label{ssec:stateDiagram}

\begin{wrapfigure}[7]{r}{0.45\textwidth}
  \vspace{-45pt}
  \centering
  \includestandalone{figures/fig_028}
  \caption{State Diagram}
  \label{fig:stateDiagramOne}
\end{wrapfigure}

In a simple circuit with no inputs (other than a clock cycle), the conditions for each
state are straightforward: if the current state is {\color{re!75!fg} red} ($S_2$), then
the next state is {\color{gr!75!fg} green} ($S_0$).

Note that a starting state is required, otherwise the initial state of the circuit will
be undefined behavior.

\subsection{State/Output Table}
\label{ssec:stateOutputTable}

\begin{figure}[H]
  \centering
  \begin{subfigure}[H]{0.45\textwidth}
    \centering
    \begin{tblr}{ccc}
      \toprule
      $S$ & $S^+$ & $O$ \\
      \midrule
      $S_0$ & $S_1$ & {\color{gr!75!fg}$g$} \\
      $S_1$ & $S_2$ & {\color{ye!75!fg}$y$} \\
      $S_2$ & $S_0$ & {\color{re!75!fg}$r$} \\
      \bottomrule
    \end{tblr}
    \caption{Abstract Representation}
    \label{fig:stateOutputTableAbstractRepresentationOne}
  \end{subfigure}
  \begin{subfigure}[H]{0.45\textwidth}
    \centering
    \begin{tblr}{cc|cc|ccc}
      \toprule
      \SetCell[c=2]{c} $S$ && \SetCell[c=2]{c} $S^+$ && \SetCell[c=3]{c} $O$ && \\
      \midrule
      0 & 0 & 0 & 1 & {\color{gr!75!fg}$1$} & {\color{gr!75!fg}$0$} & {\color{gr!75!fg}$0$} \\
      0 & 1 & 1 & 0 & {\color{ye!75!fg}$0$} & {\color{ye!75!fg}$1$} & {\color{ye!75!fg}$0$} \\
      1 & 0 & 0 & 0 & {\color{re!75!fg}$0$} & {\color{re!75!fg}$0$} & {\color{re!75!fg}$1$} \\
      1 & 1 & * & * & *                     & *                     & *                     \\
      \bottomrule
    \end{tblr}
    \caption{Binary Representation}
    \label{fig:stateOutputTableBinaryRepresentationOne}
  \end{subfigure}
  \caption{State/Output Table}
  \label{fig:stateOutputTable}
  \vspace{-15pt}
\end{figure}

Based on the state diagram (Figure \ref{fig:stateDiagramOne}), a state/output table can be
derived. This is essentially a tabular representation of the state diagram, showing the
current state ($S$), the next state ($S^+$), and the output corresponding to each state
($O$).

In Figure \ref{fig:stateOutputTableAbstractRepresentationOne}, a simple and abstracted table
is shown, directly mirroring the information in the state diagram. In Figure
\ref{fig:stateOutputTableBinaryRepresentationOne}, each state and output is replaced with
actual binary values to represent them.

\subsection{Transition/Output Table}
\label{ssec:transitionOutputTable}

\begin{figure}[H]
  \centering
  \begin{tblr}{cc|cc|ccc}
    \toprule
    \SetCell[c=2]{c} $S$ && \SetCell[c=2]{c} $S^+$ && \SetCell[c=3]{c} $O$ && \\
    $S_0$ & $S_1$         & $S_0^+$ & $S_1^+$       & $O_0$ & $O_1$ & $O_2$   \\
    \midrule
    0 & 0 & 0 & 1 & {\color{gr!75!fg}$1$} & {\color{gr!75!fg}$0$} & {\color{gr!75!fg}$0$} \\
    0 & 1 & 1 & 0 & {\color{ye!75!fg}$0$} & {\color{ye!75!fg}$1$} & {\color{ye!75!fg}$0$} \\
    1 & 0 & 0 & 0 & {\color{re!75!fg}$0$} & {\color{re!75!fg}$0$} & {\color{re!75!fg}$1$} \\
    1 & 1 & * & * & *                     & *                     & *                     \\
    \bottomrule
  \end{tblr}
  \caption{Transition Table}
  \label{fig:transitionTableOne}
  \vspace{-15pt}
\end{figure}

Notably, this transition table is essentially identical to the table in Figure
\ref{fig:stateOutputTableBinaryRepresentationOne}. This is because, when assigning binary
values to each state/output in the circuit, the transitions become apparent anyway.

\subsection{Choose a Flip-Flop}
\label{ssec:chooseAFlipFlop}

There are many Flip-Flops to choose from. However, three important ones (each with its own
derivations) exist: the \textbf{T Flip-Flop}, the \textbf{D Flip-Flop}, and the
\textbf{JK Flip-Flop}.

\newpage
\subsubsection{D Flip-Flop}
\label{sssec:DFlipFlop}
\begin{wrapfigure}[]{r}{0.5\textwidth}
  \vspace{-30pt}
  \centering
  \begin{subfigure}[H]{0.15\textwidth}
    \centering
    \begin{circuitikz}
      \node[fill=fg!10!bg, thick, latch] (D) at (0,0){};
    \end{circuitikz}
  \end{subfigure}
  \begin{subfigure}[H]{0.30\textwidth}
    \centering
    \begin{tblr}{cc|cc}
      \toprule
      $D$ & $C$ & $Q^+$ & $\overline{Q}^+$  \\
      \midrule
      0 & $\uparrow$ & $0$ & $1$            \\
      1 & $\uparrow$ & $1$ & $0$            \\
      X & 1          & $Q$ & $\overline{Q}$ \\
      \bottomrule
    \end{tblr}
  \end{subfigure}
\end{wrapfigure}
The D Flip-Flop is described by the characteristic equation:
\begin{equation*}
  Q^+ = D
\end{equation*}
The D Flip-Flop captures the value of the $D$-input at a definite portion of the clock
cycle (generally at the rising edge of the clock). That captured value becomes the $Q$
output. At other times, the output $Q$ does not change.

\subsubsection{T Flip-Flop}
\label{sssec:TFlipFlop}
\begin{wrapfigure}[]{l}{0.5\textwidth}
  \centering
  \begin{subfigure}[H]{0.15\textwidth}
    \centering
    \begin{circuitikz}
      \node[fill=fg!10!bg, thick, flipflop T] (T) at (0,0){};
    \end{circuitikz}
  \end{subfigure}
  \begin{subfigure}[H]{0.30\textwidth}
    \centering
    \begin{tblr}{cc|cc}
      \toprule
      $T$ & $C$ & $Q^+$ & $\overline{Q}^+$ \\
      \midrule
      0 & $\uparrow$ & $Q$ & $\overline{Q}$ \\
      1 & $\uparrow$ & $\overline{Q}$ & $Q$ \\
      X & 1          & $Q$ & $\overline{Q}$ \\
      X & 0          & $Q$ & $\overline{Q}$ \\
      \bottomrule
    \end{tblr}
  \end{subfigure}
\end{wrapfigure}

If the $T$ input is high, the T Flip-Flop changes state ("toggles") whenever the clock input
rises. If the $T$ input is low, the flip-flop holds the previous value. This behavior
is described by the characteristic equation:
\begin{equation*}
  Q^+ = T \oplus Q = (T \land \overline{Q}) \lor (\overline{T} \land Q)
\end{equation*}
The toggle only occurs on the rising edge of a clock signal, meaning any input is meaningless
when not paired with a rising edge. This is seen in rows $2$ and $3$ of the table.

\subsubsection{JK Flip-Flop}
\label{sssec:jkFlipFlop}

\begin{wrapfigure}[8]{r}{0.5\textwidth}
  \centering
  \begin{subfigure}[H]{0.15\textwidth}
    \centering
    \begin{circuitikz}
      \node[fill=fg!10!bg, thick, flipflop JK] (JK) at (0,0){};
    \end{circuitikz}
  \end{subfigure}
  \begin{subfigure}[H]{0.30\textwidth}
    \centering
    \begin{tblr}{ccc|cc}
      \toprule
      $J$ & $K$ & $C$ & $Q^+$ & $\overline{Q}^+$ \\
      \midrule
      0 & 0 & $\uparrow$ & $Q$            & $\overline{Q}$ \\
      0 & 1 & $\uparrow$ & 0              & 1              \\
      1 & 0 & $\uparrow$ & 1              & 0              \\
      1 & 1 & $\uparrow$ & $\overline{Q}$ & $Q$            \\
      X & X & 0          & $Q$            & $\overline{Q}$ \\
      X & X & 1          & $Q$            & $\overline{Q}$ \\
      \bottomrule
    \end{tblr}
  \end{subfigure}
\end{wrapfigure}

The JK Flip-Flop, augments the behavior of the SR flip-flop ($J$: Set, $K$: Reset) by 
interpreting the $J = K = 1$ condition as a "flip" or toggle command.
\begin{align*}
  J=1, K=1 &\rightarrow \textup{Toggle} \\
  J=0, K=1 &\rightarrow \textup{Reset} \\
  J=1, K=0 &\rightarrow \textup{Set} \\
  J=0, K=0 &\rightarrow \textup{No Change}
\end{align*}
Similarly to the T Flip-Flop, the changes can only occur on the rising edge of a clock
signal, making any input outside of that rising edge meaningless. The JK Flip-Flop is
described by the characteristic equation:
\begin{equation*}
  Q^+ = (J \land \overline{Q}) \lor (\overline{K} \land Q)
\end{equation*}
\hrule
\vspace{8pt}
For the purposes of this circuit, the \textbf{D Flip-Flop} is sufficient and will be used.

\subsection{Construct an Excitation Table}
\label{ssec:constructAnExcitationTable}

An excitation table is shows the information required to change (or "excite") the circuit
into moving from one state to the next. Since flip-flops all function differently, an
excitation table for a circuit cannot be generalized to any flip-flop.

For the current design, the following excitation table can be constructed from the 
transition/output table.

\begin{figure}[H]
  \centering
  \begin{subfigure}[H]{0.40\textwidth}
    \centering
    \begin{tblr}{cc|cc|ccc}
      \toprule
      \SetCell[c=2]{c} $S$ && \SetCell[c=2]{c} $S^+$ && \SetCell[c=3]{c} $O$ && \\
      $S_0$ & $S_1$         & $S_0^+$ & $S_1^+$       & $O_0$ & $O_1$ & $O_2$   \\
      \midrule
      0 & 0 & 0 & 1 & {\color{gr!75!fg}$1$} & {\color{gr!75!fg}$0$} & {\color{gr!75!fg}$0$} \\
      0 & 1 & 1 & 0 & {\color{ye!75!fg}$0$} & {\color{ye!75!fg}$1$} & {\color{ye!75!fg}$0$} \\
      1 & 0 & 0 & 0 & {\color{re!75!fg}$0$} & {\color{re!75!fg}$0$} & {\color{re!75!fg}$1$} \\
      1 & 1 & * & * & *                     & *                     & *                     \\
      \bottomrule
    \end{tblr}
    \caption{Transition/Output Table}
    \label{fig:transitionOutputTableOneTwo}
  \end{subfigure}
  \begin{subfigure}[H]{0.50\textwidth}
    \centering
    \begin{tblr}{cc|cc|cc|ccc}
      \toprule
      \SetCell[c=2]{c} $S$ && \SetCell[c=2]{c} $S^+$ && \SetCell[c=2]{c} $D$ && \SetCell[c=3]{c} $O$ && \\
      $S_0$ & $S_1$         & $S_0^+$ & $S_1^+$       & $D_0$ & $D_1$         & $O_0$ & $O_1$ & $O_2$   \\
      \midrule
      0 & 0 & 0 & 1 & 0 & 1 & {\color{gr!75!fg}$1$} & {\color{gr!75!fg}$0$} & {\color{gr!75!fg}$0$} \\
      0 & 1 & 1 & 0 & 1 & 0 & {\color{ye!75!fg}$0$} & {\color{ye!75!fg}$1$} & {\color{ye!75!fg}$0$} \\
      1 & 0 & 0 & 0 & 0 & 0 & {\color{re!75!fg}$0$} & {\color{re!75!fg}$0$} & {\color{re!75!fg}$1$} \\
      1 & 1 & * & * & * & * & *                     & *                     & *                     \\
      \bottomrule
    \end{tblr}
    \caption{Excitation Table}
    \label{fig:excitationTableOne}
  \end{subfigure}
\end{figure}
Clearly, both the transition table and excitation table are incredibly similar. The only
difference is the addition of the $D$ column (referring to each of the two D Flip-Flops
required in the design). This column holds the required inputs to each of the Flip-Flops
to properly transition the circuit between states.

\subsection{Derive the Logic Equations}
\label{ssec:deriveTheLogicEquations}

From the excitation table in Figure \ref{fig:excitationTableOne}, a K-Map can be created for
each of the outputs as well as each of the states. From the K-Maps, the state and output
equations can be derived.

\begin{figure}[H]
  \centering
  \begin{subfigure}[H]{0.45\textwidth}
    \centering
    \includestandalone{figures/fig_029}
    \begin{equation*}
      S_0^+ = S_1
    \end{equation*}
  \end{subfigure}
  \vrule
  \begin{subfigure}[H]{0.45\textwidth}
    \centering
    \includestandalone{figures/fig_030}
    \begin{equation*}
      S_1^+ = \overline{S_0} \land \overline{S_1} = S_0 \overline{\lor} S_1
    \end{equation*}
  \end{subfigure}
  \caption{State K-Maps}
  \label{fig:stateKMapsOne}
\end{figure}

\begin{figure}[H]
  \centering
  \begin{subfigure}[H]{0.45\textwidth}
    \centering
    \includestandalone{figures/fig_031}
    \begin{equation*}
      {\color{gr!50!fg}O_0} = \overline{S_0} \land \overline{S_1} = S_0 \overline{\lor} S_1
    \end{equation*}
  \end{subfigure}
  \vrule
  \begin{subfigure}[H]{0.45\textwidth}
    \centering
    \includestandalone{figures/fig_032}
    \begin{equation*}
      {\color{ye!50!fg}O_1} = S_1
    \end{equation*}
  \end{subfigure}
  \vspace{12pt}
  \hrule
  \vspace{12pt}
  \begin{subfigure}[H]{0.45\textwidth}
    \centering
    \includestandalone{figures/fig_033}
    \begin{equation*}
      {\color{re!75!fg}O_2} = S_0
    \end{equation*}
  \end{subfigure}
  \caption{State K-Maps}
  \label{fig:outputKMapsOne}
\end{figure}

\subsection{Construct the Circuit}
\label{ssec:constructTheCircuit}

Using the equations derived in the previous step, all that is left is to construct the
circuit accordingly:

\begin{figure}[H]
  \centering
  \includestandalone{figures/fig_034}
  \caption{Fully Constructed Circuit}
  \label{fig:fullyConstructedCircuitOne}
\end{figure}

\end{document}
