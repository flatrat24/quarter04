\documentclass[12pt]{article}

\input{../../xlatex/imports/preamble}

\title{Lecture 006}
\date{October 23, 2024}

\begin{document}
\newpage
\section{Latches and Flip-Flops}
\label{sec:latchesAndFlipFlops}

\subsection{Latch Gate}
\label{ssec:latchGate}

\begin{definition}{Latch}
  A latch is a \uline{bi-stable} memory element, meaning that it can store data in two
  different states ($1$ or $0$). Additionally, it is also \uline{level-sensitive}, meaning
  that the state that it stores is dependent on the inputs to the latch.
\end{definition}
A latch can be implemented in many ways, one of which being with two NOR gates as seen in
Figure \ref{fig:latchCircuit}.

\begin{figure}[H]
  \centering
  \begin{subfigure}[H]{0.5\textwidth}
    \centering
    \includestandalone{figures/fig_018}
    \caption{}
    \label{fig:018}
  \end{subfigure}
  \begin{subfigure}[H]{0.4\textwidth}
    \centering
    \includestandalone{figures/fig_019}
    \caption{}
    \label{fig:019}
  \end{subfigure}
  \caption{Latch Implementation With Nor Gates}
  \label{fig:latchCircuit}
\end{figure}

\hrule

\begin{center}
  {\large \textbf{Operating a Latch Gate}}
\end{center}

The operation of a latch follows two simple rules:
\begin{enumerate}
  \itemsep0em
  \item For all states, $P \equiv \overline{Q}$. $P$ is always the complement of $Q$
  \item The inputs $R$ and $S$ must never change simultaneously
\end{enumerate}

Following these simple rules, manipulating the values of $R$ and $S$ changes the outputs of $P$
and $Q$ according to the following table.

\begin{figure}[H]
  \centering
  \begin{subfigure}[H]{0.45\textwidth}
    \centering
    \begin{tblr}{cc|cc}
      \toprule
      S & R & P & Q \\
      \midrule
      0 & 0 & P & Q \\
      0 & 1 & 0 & 1 \\
      1 & 0 & 1 & 0 \\
      1 & 1 & 0 & 0 \\
      \bottomrule
    \end{tblr}
    \caption{Characteristic Table}
    \label{fig:characteristicTable}
  \end{subfigure}
  \begin{subfigure}[H]{0.45\textwidth}
    \centering
    \begin{equation*}
      Q^+ = S \lor (\overline{R} \land Q)
    \end{equation*}
    \caption{Characteristic Equation}
    \label{fig:characteristicEquation}
  \end{subfigure}
  \caption{}
  \label{fig:tempTwo}
\end{figure}

Consider Figure \ref{fig:latchOne}. Starting in the initial state of $R,S=0$,
changing the value of $R$ to $1$ results in the output of $P$ changing from $0$ to $1$,
and the output of $Q$ changing from $1$ to $0$.

\begin{figure}[H]
  \centering
  \includestandalone{figures/fig_020}
  \caption{}
  \label{fig:latchOne}
\end{figure}

However, going from the resulting state in Figure \ref{fig:latchOne} and changing $R$ back
to $0$, the state of $P$ and $Q$ are unaffected. Thus, we have a situation in which the
inputs of $R=0$ and $S=0$ correspond to outputs of $P=0$ and $Q=1$. Previously, the same
inputs corresponded to the complement of the outputs.

\begin{figure}[H]
  \centering
  \includestandalone{figures/fig_021}
  \caption{}
  \label{fig:latchTwo}
\end{figure}

Starting from the resulting state of Figure \ref{fig:latchOne}, changing $R$ back to $0$
has no effect on the outputs of the latch. This is a demonstration of the latch's "latching"
behavior; in certain states, it retains a memory of its outputs independentally of its inputs.

\begin{figure}[H]
  \centering
  \includestandalone{figures/fig_022}
  \caption{}
  \label{fig:latchThree}
\end{figure}

From the result of Figure \ref{fig:latchTwo}, raising $S$ to $1$ has the effect of
complementing both of the outputs; $P$ goes to $1$ from $0$, and $Q$ goes to $0$ from $1$.

\begin{figure}[H]
  \centering
  \includestandalone{figures/fig_023}
  \caption{}
  \label{fig:latchFour}
\end{figure}

After lowering $S$ back to $0$, again, the latch retains its outputs.

\begin{figure}[H]
  \centering
  \includestandalone{figures/fig_024}
  \caption{}
  \label{fig:latchFive}
\end{figure}

Lastly, by raising $R$ back to $1$, the outputs of the latch are the same as they were in
the initial state. If $R$ were to be lowered again to $0$, the latch would retain its
memory of the state and would find itself back in the same initial state as it was in 
Figure \ref{fig:latchOne}.

Notice that the outputs only ever change when either $S$ or $R$ is raised to $1$. When
either is lowered, the outputs remain constant.
\vspace{12pt}
\hrule

Latches are useful for signal "debouncing". When a physical button is pressed, often there
is excess kinetic energy that causes to circuit to flicker, leading to several input
changes despite the button being pressed only a single time.
\begin{figure}[H]
  \centering
  \includestandalone{figures/fig_025}
  \caption{Bouncing Signal}
  \label{fig:025}
  \vspace{-15pt}
\end{figure}
Since a latch only changes its output when an input is \textbf{raised}, but not when it
is lowered, only the initial rise will change the output. Each lowering of the input due
to the bouncing is negated.
\begin{figure}[H]
  \centering
  \includestandalone{figures/fig_026}
  \caption{Debounced Signal}
  \label{fig:026}
\end{figure}

\subsection{Gated Latch}
\label{ssec:gatedLatch}

\begin{figure}[H]
  \centering
  \begin{subfigure}[H]{0.45\textwidth}
    \centering
    \begin{tblr}{ccc|cc}
      \toprule
      Clk & S & R & P & Q \\
      \midrule
      0 & x & x & P & Q \\
      1 & 0 & 0 & P & Q \\
      1 & 0 & 1 & 0 & 1 \\
      1 & 1 & 0 & 1 & 0 \\
      1 & 1 & 1 & x & x \\
      \bottomrule
    \end{tblr}
    \caption{Characteristic Table}
    \label{fig:characteristicTableTwo}
  \end{subfigure}
  \begin{subfigure}[H]{0.45\textwidth}
    \centering
    \begin{equation*}
      Q = Clk \big(S + \overline{R} \cdot Q \big) + \overline{Clk} \cdot Q
    \end{equation*}
    \caption{Characteristic Equation}
    \label{fig:characteristicEquationTwo}
  \end{subfigure}
  \caption{}
  \label{fig:tempOne}
\end{figure}

A gated latch operated similarly to a regular latch in that it uses two inputs $R$ and $S$
to manage a circuit capable of storing memory. However, the difference is that it connects
each input through an AND gate with a clock.

Since the state of a latch can only change when $R$ or $S$ is raised, and $R$ or $S$ can
only be raised when the clock ticks, the changing of $P$ or $Q$ is essentially restricted
to when the clock ticks. In other words, the gated latch functions identically to the 
regular latch, but it only can change states during the tick of a clock.

\begin{figure}[H]
  \centering
  \includestandalone{figures/fig_027}
  \caption{Gated Latch}
  \label{fig:027}
\end{figure}

\subsection{Timing Diagrams}
\label{ssec:timingDiagrams}

\begin{definition}{Timing Diagram}
  A timing diagram is a graphical view of a circuits behavior that shows how the outputs
  of a circuit change in response to the inputs which are varying as a function of time.
\end{definition}

For a circuit such as a three-input AND gate, a possible timing diagram would look like
Figure \ref{fig:035}.

\begin{figure}[H]
  \centering
  \includestandalone{figures/fig_036}
\end{figure}

\begin{figure}[H]
  \centering
  \includestandalone{figures/fig_035}
  \caption{AND Timing Diagram}
  \label{fig:035}
\end{figure}

When given arbitrary high and low values for the inputs $x$, $y$, and $z$, the output $O$
is only high when all three inputs are high. This is just a visual representation of
the operation of an AND gate.

With more complex circuits, timing diagrams get a little harder to parse. However, the
same basic concept applies: each line for each input, output, or state should logically
correspond to each other line of the diagram. Consider the timing diagram for some D
Flip-Flops in Figure \ref{fig:timingDiagramForSomeDFlipFlops}.

\begin{figure}[H]
  \begin{subfigure}[H]{0.35\textwidth}
    \centering
    \includestandalone{figures/fig_037}
  \end{subfigure}
  \begin{subfigure}[H]{0.55\textwidth}
    \centering
    \includestandalone{figures/fig_038}
  \end{subfigure}
  \caption{Timing Diagram for D Flip-Flops}
  \label{fig:timingDiagramForSomeDFlipFlops}
\end{figure}

Flip-flop $a$ has a direct $Clk$ input, meaning it just takes the value of the clock and
combined it with input $D$. Flip-flop $b$ triggers only on the rising edge, and flip-flop
$c$ triggers only on a falling edge.

\end{document}
