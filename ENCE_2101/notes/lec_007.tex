\documentclass[12pt]{article}

\input{../../xlatex/imports/preamble}

\title{Lecture 006}
\date{October 23, 2024}

\begin{document}
\newpage
\section{Sequential Circuits}
\label{sec:sequentialCircuits}

A \textbf{combinational circuit} is the result of interconnecting logic gates. The
functioning of a combinational circuit is instentaneous. The current state of the inputs
entirely determines the state of the outputs.

\begin{definition}{Sequential Circuit}
  Contains \textbf{combinational circuits} and \textbf{memory elements} with a feedback
  path from the memory element to the combinational circuit.
\end{definition}

The output of a sequential circuit will be dependent on \textbf{both} the current inputs
as well as the memory, or past history, of the circuit stored in the memory elements. The
data stored in the memory is referred to as \textbf{the state} of the circuit.

\subsection{Synchronous Sequential Circuits}
\label{ssec:synchronousSequentialCircuits}
A \textbf{synchronous sequential circuit} is a circuit whose behavior can be defined from
the knowledge of its signals at \uline{discrete instants of time}. A clock signal is used
to control and synchronize the circuit behavior. These circuits are called \textbf{clocked
sequential circuits}.
\begin{figure}[H]
  \centering
  \includestandalone{figures/fig_017}
  \caption{Clock Signals}
  \label{fig:017}
\end{figure}
These clock signals "ticks" each time there is a change from low to high or from high to
low. A circuit can use either the leading
(\tikz{\draw[draw=bl!75!fg, ultra thick, stealth-] (0,0) -- ++(0,-0.3);}) or trailing
(\tikz{\draw[draw=re!75!fg, ultra thick, stealth-] (0,0) -- ++(0,0.3);}) edge of a pulse
to mark the "tick" of the clock pulse.

\subsection{Sequential Circuit Analysis}
\label{ssec:sequentialCircuitAnalysis}

Analyzing a sequential circuit follows three steps:
\begin{enumerate}
  \itemsep0em
  \item Derive the \textbf{state equations} and the \textbf{output equations} for the
    circuit
    \begin{itemize}
      \itemsep0em
      \item State equations model the next state as a function of the current states and
        any inputs
      \item Output equations model the outputs as a function of the current states and any
        inputs
    \end{itemize}
  \item Derive the \textbf{transition table} and the \textbf{state table} for the circuit
  \item Derive the \textbf{state diagram} for the circuit
\end{enumerate}

\subsubsection{State and Output Equations}
\label{sssec:stateAndOutputEquations}



\end{document}
