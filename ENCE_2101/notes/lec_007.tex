\documentclass[12pt]{article}

%%%% GRAPHICS %%%%
\usepackage{tikz}
\usepackage{circuitikz}
\usetikzlibrary{arrows.meta}
\usepackage{tikz-3dplot}
\usepackage{graphicx}
\usepackage{pgfplots}
  \pgfplotsset{compat=1.18}
\usetikzlibrary{arrows}
\newcommand{\midarrow}{\tikz \draw[-triangle 90] (0,0) -- +(.1,0);}

%%%% FIGURES %%%%
\usepackage{subcaption}
\usepackage{wrapfig}
\usepackage{float}
\usepackage[skip=5pt, font=footnotesize]{caption}

%%%% FORMATTING %%%%
\usepackage{parskip}
\usepackage{tcolorbox}
\usepackage{ulem}
% \usepackage{fancyhdr}

%%%% TABLE FORMATTING %%%%
\usepackage{tabularray}
\UseTblrLibrary{booktabs}

%%%% MATH AND LOGIC %%%%
\usepackage{xifthen}
\usepackage{amsmath}
\usepackage{amssymb}
\usepackage{amsfonts}

%%%% TEXT AND SYMBOLS %%%%
\usepackage[T1]{fontenc}
\usepackage{textcomp}
\usepackage{gensymb}

%%%% OTHER %%%%
\usepackage{standalone}

%%%% LOGIC SYMBOLS %%%%
\newcommand*\xor{\oplus}

%%%% STYLES %%%%

% Packages
\usepackage{fullpage}
\usepackage{titlesec}
\usepackage[rgb]{xcolor}
\selectcolormodel{natural}
\usepackage{ninecolors}
\selectcolormodel{rgb}

% Colors
\definecolor{pg}{HTML}{24273A}
\definecolor{fg}{HTML}{FFFFFF}
\definecolor{bg}{HTML}{24273A}
\definecolor{re}{HTML}{d20f39}
\definecolor{gr}{HTML}{40a02b}
\definecolor{ye}{HTML}{df8e1d}
\definecolor{or}{HTML}{fe640b}
\definecolor{bl}{HTML}{1e66f5}
\definecolor{ma}{HTML}{8839ef}
\definecolor{cy}{HTML}{179299}
\definecolor{pi}{HTML}{ea76cb}

\usepackage{nameref}
\makeatletter
\newcommand*{\currentname}{\@currentlabelname}
\makeatother

\titleformat{\section}
  {\normalfont\scshape\Large\bfseries}
  {\thesection}
  {0.75em}
  {}

\titleformat{\subsection}
  {\normalfont\scshape\large\bfseries}
  {\thesubsection}
  {0.75em}
  {}

\titleformat{\subsubsection}
  {\normalfont\scshape\normalsize\bfseries}
  {\thesubsubsection}
  {0.75em}
  {}

% Formula
\newcounter{formula}[section]
\newenvironment{formula}[1]{
  \stepcounter{formula}
  \begin{tcolorbox}[
    standard jigsaw, % Allows opacity
    colframe={fg},
    boxrule=1px,
    colback=bg,
    opacityback=0,
    sharp corners,
    sidebyside,
    righthand width=18px,
    coltext={fg}
  ]
  \centering
  \textbf{\uline{#1}}
}{
  \tcblower
  \textbf{\thesection.\theformula}
  \end{tcolorbox}
}

% Definition
\newcounter{definition}[section]

\newenvironment{definition*}[1]{
  \begin{tcolorbox}[
    standard jigsaw, % Allows opacity
    colframe={fg},
    boxrule=1px,
    colback=bg,
    opacityback=0,
    sharp corners,
    coltext={fg}
  ]
  \textbf{#1 \hfill}
  \vspace{5px}
  \hrule
  \vspace{5px}
  \noindent
}{
  \end{tcolorbox}
}

\newenvironment{definition}[1]{
  \stepcounter{definition}
  \begin{tcolorbox}[
    standard jigsaw, % Allows opacity
    colframe={fg},
    boxrule=1px,
    colback=bg,
    opacityback=0,
    sharp corners,
    coltext={fg}
  ]
  \textbf{#1 \hfill \thesection.\thedefinition}
  \vspace{5px}
  \hrule
  \vspace{5px}
  \noindent
}{
  \end{tcolorbox}
}

% Example Problem
\newcounter{example}[section]
\newenvironment{example}{
  \stepcounter{example}
  \begin{tcolorbox}[
    standard jigsaw, % Allows opacity
    colframe={fg},
    boxrule=1px,
    colback=bg,
    opacityback=0,
    sharp corners,
    coltext={fg}
  ]
  \textbf{Example \hfill \thesection.\theexample}
  \vspace{5px}
  \hrule
  \vspace{5px}
  \noindent
}{
  \end{tcolorbox}
}

\tikzset{
  cubeBorder/.style=fg,
  cubeFilling/.style={fg!20!bg, opacity=0.25},
  gridLine/.style={very thin, gray},
  graphLine/.style={-latex, thick, fg},
}

\pgfplotsset{
  basicAxis/.style={
    grid,
    major grid style={line width=.2pt,draw=fg!50!bg},
    axis lines = box,
    axis line style = {line width = 1px},
  }
}

%%%% REFERENCES %%%%
\usepackage{hyperref}
\hypersetup{
  colorlinks  = true,
  linkcolor   = pi,
  anchorcolor = pi,
  citecolor   = pi,
  filecolor   = pi,
  menucolor   = pi,
  runcolor    = pi,
  urlcolor    = pi,
}

\author{Ethan Anthony}


\title{Lecture 006}
\date{October 23, 2024}

\begin{document}
\newpage
\section{Sequential Circuits}
\label{sec:sequentialCircuits}

A \textbf{combinational circuit} is the result of interconnecting logic gates. The
functioning of a combinational circuit is instentaneous. The current state of the inputs
entirely determines the state of the outputs.

\begin{definition}{Sequential Circuit}
  Contains \textbf{combinational circuits} and \textbf{memory elements} with a feedback
  path from the memory element to the combinational circuit.
\end{definition}

The output of a sequential circuit will be dependent on \textbf{both} the current inputs
as well as the memory, or past history, of the circuit stored in the memory elements. The
data stored in the memory is referred to as \textbf{the state} of the circuit.

\subsection{Synchronous Sequential Circuits}
\label{ssec:synchronousSequentialCircuits}
A \textbf{synchronous sequential circuit} is a circuit whose behavior can be defined from
the knowledge of its signals at \uline{discrete instants of time}. A clock signal is used
to control and synchronize the circuit behavior. These circuits are called \textbf{clocked
sequential circuits}.
\begin{figure}[H]
  \centering
  \includestandalone{figures/fig_017}
  \caption{Clock Signals}
  \label{fig:017}
\end{figure}
These clock signals "ticks" each time there is a change from low to high or from high to
low. A circuit can use either the leading
(\tikz{\draw[draw=bl!75!fg, ultra thick, stealth-] (0,0) -- ++(0,-0.3);}) or trailing
(\tikz{\draw[draw=re!75!fg, ultra thick, stealth-] (0,0) -- ++(0,0.3);}) edge of a pulse
to mark the "tick" of the clock pulse.

\subsection{Sequential Circuit Analysis}
\label{ssec:sequentialCircuitAnalysis}

Analyzing a sequential circuit follows three steps:
\begin{enumerate}
  \itemsep0em
  \item Derive the \textbf{state equations} and the \textbf{output equations} for the
    circuit
    \begin{itemize}
      \itemsep0em
      \item State equations model the next state as a function of the current states and
        any inputs
      \item Output equations model the outputs as a function of the current states and any
        inputs
    \end{itemize}
  \item Derive the \textbf{transition table} and the \textbf{state table} for the circuit
  \item Derive the \textbf{state diagram} for the circuit
\end{enumerate}

\subsubsection{State and Output Equations}
\label{sssec:stateAndOutputEquations}



\end{document}
