\documentclass[12pt]{article}

%%%% GRAPHICS %%%%
\usepackage{tikz}
\usepackage{circuitikz}
\usetikzlibrary{arrows.meta}
\usepackage{tikz-3dplot}
\usepackage{graphicx}
\usepackage{pgfplots}
  \pgfplotsset{compat=1.18}
\usetikzlibrary{arrows}
\newcommand{\midarrow}{\tikz \draw[-triangle 90] (0,0) -- +(.1,0);}

%%%% FIGURES %%%%
\usepackage{subcaption}
\usepackage{wrapfig}
\usepackage{float}
\usepackage[skip=5pt, font=footnotesize]{caption}

%%%% FORMATTING %%%%
\usepackage{parskip}
\usepackage{tcolorbox}
\usepackage{ulem}
% \usepackage{fancyhdr}

%%%% TABLE FORMATTING %%%%
\usepackage{tabularray}
\UseTblrLibrary{booktabs}

%%%% MATH AND LOGIC %%%%
\usepackage{xifthen}
\usepackage{amsmath}
\usepackage{amssymb}
\usepackage{amsfonts}

%%%% TEXT AND SYMBOLS %%%%
\usepackage[T1]{fontenc}
\usepackage{textcomp}
\usepackage{gensymb}

%%%% OTHER %%%%
\usepackage{standalone}

%%%% LOGIC SYMBOLS %%%%
\newcommand*\xor{\oplus}

%%%% STYLES %%%%

% Packages
\usepackage{fullpage}
\usepackage{titlesec}
\usepackage[rgb]{xcolor}
\selectcolormodel{natural}
\usepackage{ninecolors}
\selectcolormodel{rgb}

% Colors
\definecolor{pg}{HTML}{24273A}
\definecolor{fg}{HTML}{FFFFFF}
\definecolor{bg}{HTML}{24273A}
\definecolor{re}{HTML}{d20f39}
\definecolor{gr}{HTML}{40a02b}
\definecolor{ye}{HTML}{df8e1d}
\definecolor{or}{HTML}{fe640b}
\definecolor{bl}{HTML}{1e66f5}
\definecolor{ma}{HTML}{8839ef}
\definecolor{cy}{HTML}{179299}
\definecolor{pi}{HTML}{ea76cb}

\usepackage{nameref}
\makeatletter
\newcommand*{\currentname}{\@currentlabelname}
\makeatother

\titleformat{\section}
  {\normalfont\scshape\Large\bfseries}
  {\thesection}
  {0.75em}
  {}

\titleformat{\subsection}
  {\normalfont\scshape\large\bfseries}
  {\thesubsection}
  {0.75em}
  {}

\titleformat{\subsubsection}
  {\normalfont\scshape\normalsize\bfseries}
  {\thesubsubsection}
  {0.75em}
  {}

% Formula
\newcounter{formula}[section]
\newenvironment{formula}[1]{
  \stepcounter{formula}
  \begin{tcolorbox}[
    standard jigsaw, % Allows opacity
    colframe={fg},
    boxrule=1px,
    colback=bg,
    opacityback=0,
    sharp corners,
    sidebyside,
    righthand width=18px,
    coltext={fg}
  ]
  \centering
  \textbf{\uline{#1}}
}{
  \tcblower
  \textbf{\thesection.\theformula}
  \end{tcolorbox}
}

% Definition
\newcounter{definition}[section]

\newenvironment{definition*}[1]{
  \begin{tcolorbox}[
    standard jigsaw, % Allows opacity
    colframe={fg},
    boxrule=1px,
    colback=bg,
    opacityback=0,
    sharp corners,
    coltext={fg}
  ]
  \textbf{#1 \hfill}
  \vspace{5px}
  \hrule
  \vspace{5px}
  \noindent
}{
  \end{tcolorbox}
}

\newenvironment{definition}[1]{
  \stepcounter{definition}
  \begin{tcolorbox}[
    standard jigsaw, % Allows opacity
    colframe={fg},
    boxrule=1px,
    colback=bg,
    opacityback=0,
    sharp corners,
    coltext={fg}
  ]
  \textbf{#1 \hfill \thesection.\thedefinition}
  \vspace{5px}
  \hrule
  \vspace{5px}
  \noindent
}{
  \end{tcolorbox}
}

% Example Problem
\newcounter{example}[section]
\newenvironment{example}{
  \stepcounter{example}
  \begin{tcolorbox}[
    standard jigsaw, % Allows opacity
    colframe={fg},
    boxrule=1px,
    colback=bg,
    opacityback=0,
    sharp corners,
    coltext={fg}
  ]
  \textbf{Example \hfill \thesection.\theexample}
  \vspace{5px}
  \hrule
  \vspace{5px}
  \noindent
}{
  \end{tcolorbox}
}

\tikzset{
  cubeBorder/.style=fg,
  cubeFilling/.style={fg!20!bg, opacity=0.25},
  gridLine/.style={very thin, gray},
  graphLine/.style={-latex, thick, fg},
}

\pgfplotsset{
  basicAxis/.style={
    grid,
    major grid style={line width=.2pt,draw=fg!50!bg},
    axis lines = box,
    axis line style = {line width = 1px},
  }
}

%%%% REFERENCES %%%%
\usepackage{hyperref}
\hypersetup{
  colorlinks  = true,
  linkcolor   = pi,
  anchorcolor = pi,
  citecolor   = pi,
  filecolor   = pi,
  menucolor   = pi,
  runcolor    = pi,
  urlcolor    = pi,
}

\author{Ethan Anthony}


\title{Lecture 002}
\date{September 16, 2024}

\begin{document}

\section{Truth Tables}
\label{sec:truthTables}

Truth tables are useful for depicting logical functions. A truth table shows the output value
of a logic function or expression for every combination of values of the input variables.

In Figure \ref{fig:truthTableTemplate}, 

\begin{figure}[H]
  \centering
  \begin{tblr}{c|ccc|c}
    \toprule
    \textbf{Row \#} & $i_0$ & $i_1$ & $i_n$ & \textbf{Output Values} \\
    \midrule
    0              & 0     & 0     & 0     & 0                      \\
    1              & 0     & 0     & 1     & 0                      \\
    2              & 0     & 1     & 0     & 0                      \\
    3              & 0     & 1     & 1     & 0                      \\
    $2^n - 1$      & 1     & 1     & 1     & 0                      \\
    \bottomrule
  \end{tblr}
  \caption{Truth Table Template}
  \label{fig:truthTableTemplate}
\end{figure}

\subsection{Boolean Algebra}
\label{ssec:booleanAlgebra}

Boolean algebra consists of:
\begin{itemize}
  \itemsep0em
  \item A set of elements/variables (such as $x$, $y$, and $z$) that represent a condition or a physical quantity.
  \item Values for elements/variables (i.e. 0 = FALSE, 1 = TRUE) that represent the state of the variable.
  \item Primitive logical operations (\textbf{AND}, \textbf{OR}, \textbf{NOT}) for combining the values of the variables.
  \item A set of axioms, postulates, and theorems that define the algebra.
\end{itemize}

\textbf{Boolean Operators} are foundational to boolean algebra. They take a set of inputs, whether it be a single
input or a pair, and return a value based on the states of those inputs. These are the logical
operations that combine values of variables in boolean algebra.

\begin{definition}{AND Operator}
  \begin{wrapfigure}[]{r}{0.2\textwidth}
    \centering
    \begin{circuitikz}
      \ctikzset{
        logic ports=ieee,
        logic ports/scale=1,
        logic ports/fill=fg!25!bg
      }
      \node[thick, and port] (AND1) at (0,0) {};
    \end{circuitikz}
  \end{wrapfigure}

  When all inputs are $1$, the output value of \textbf{and} is $1$. Otherwise, the output is
  $0$.

  \begin{center}
    $AB$, $A \cdot B$, $A \land B$, $A$ \& $B$, or $A$ and $B$.
  \end{center}
\end{definition}

\begin{definition}{OR Operator}
  \begin{wrapfigure}[]{r}{0.2\textwidth}
    \centering
    \begin{circuitikz}
      \ctikzset{
        logic ports=ieee,
        logic ports/scale=1,
        logic ports/fill=fg!25!bg
      }
      \node[thick, or port] (OR1) at (0,0) {};
    \end{circuitikz}
  \end{wrapfigure}

  When any input is $1$, the output value of \textbf{or} is $1$. Otherwise, the output is
  $0$.

  \begin{center}
    $X + Y$, $X \# Y$, $X | Y$, $X \lor Y$, or $X or Y$.
  \end{center}
\end{definition}

\begin{definition}{NOT Operator}
  \begin{wrapfigure}[]{r}{0.2\textwidth}
    \centering
    \begin{circuitikz}
      \ctikzset{
        logic ports=ieee,
        logic ports/scale=1,
        logic ports/fill=fg!25!bg
      }
      \node[thick, not port] (NOT1) at (0,0) {};
    \end{circuitikz}
  \end{wrapfigure}

  The output of \textbf{not} is the \textit{inverse} of the output; $1$ becomes $0$, and
  $0$ becomes $1$.

  \begin{center}
    $\overline{Y}$, $Y'$, $!Y$, $\neg Y$, or not $Y$.
  \end{center}
\end{definition}

\begin{definition}{XOR Operator}
  \begin{wrapfigure}[]{r}{0.2\textwidth}
    \centering
    \begin{circuitikz}
      \ctikzset{
        logic ports=ieee,
        logic ports/scale=1,
        logic ports/fill=fg!25!bg
      }
      \node[thick, xor port] (XOR1) at (0,0) {};
    \end{circuitikz}
  \end{wrapfigure}

  The output of \textbf{XOR} is 1 is \textbf{only} one of the inputs is 1, but not both and
  not neither.

  \begin{center}
    $A \oplus B$ or $A$ XOR $B$.
  \end{center}
\end{definition}

\begin{definition}{NAND and NOR Operators}
  \begin{wrapfigure}[4]{r}{0.35\textwidth}
    \centering
    \begin{circuitikz}
      \ctikzset{
        logic ports=ieee,
        logic ports/scale=1,
        logic ports/fill=fg!25!bg
      }
      \node[thick, nand port] (NAND1) at (0,0) {};
      \node[thick, nor port] (NOR1) at (2.5,0) {};
    \end{circuitikz}
  \end{wrapfigure}

  By adding \textbf{N} to the beginning of a gate, the gate is converted from what it was
  to the inverse of it. \textbf{AND} becomes \textbf{NOT AND}, \textbf{OR} becomes \textbf{NOT OR}.
  Functionally, the outputs of either of the gates is just the inverse of the original gate.
  To convert a drawn AND or OR gate to NAND or NOR respectively, simply add a circle to the
  output.
\end{definition}

Just like "regular" algebra, boolean algebra can be operated on through methods such as factoring,
as well as logical properties such as those listen in Figure \ref{fig:listOfLogicalEquivalences}.
Using these, we can simplify complex equations to simpler ones:
\begin{align*}
  F &= (\overline{x} \land \overline{y} \land z) \lor (x \land y \land z) \lor (\overline{x} \land y \land z) \lor (x \land \overline{y} \land z) \\
  F &= (\overline{x} \land \overline{y} \land z) \lor (\overline{x} \land y \land z) \lor (x \land y \land z) \lor (x \land \overline{y} \land z) \\
  F &= [\overline{x} \land z(\overline{y} \lor y)] \lor [x \land z(y \lor \overline{y})] \\
  F &= (\overline{x} \land z \land 1) \lor (x \land z \land 1) \\
  F &= (\overline{x} \land z) \lor (x \land z) \\
  F &= z \land (\overline{x} \lor x) \\
  F &= z \land 1 \\
  F &= z \\
\end{align*}
Thus, through algebraic manipulation, it can be seen that the original equation:
\begin{center}
  $F = (\overline{x} \land \overline{y} \land z) \lor (x \land y \land z) \lor (\overline{x} \land y \land z) \lor (x \land \overline{y} \land z)$
\end{center}
is equivalent to the direct state of $z$:
\begin{center}
  $F = z$
\end{center}

\begin{figure}[H]
  \centering
  \begin{tblr}{cc}
    \toprule
    \textbf{Mathematical Representation} & \textbf{Name of the Law} \\
    \midrule
    $p\lor\neg p \equiv T$ & Law of Excluded Middle \\
    $p\land \neg p \equiv F$ & Law of Non-Contradiction \\
    $p \land T \equiv p$ & \SetCell[r=2]{c} Identity Laws \\
    $p \lor F \equiv p$ \\
    $p \land F \equiv F$ & \SetCell[r=2]{c} Domination Laws \\
    $p \lor T \equiv T$ \\
    $p \lor p \equiv p$ & \SetCell[r=2]{c} Idempotent Laws \\
    $p \land p \equiv p$ \\
    $\neg\neg p \equiv p$ & Laws of Double Negation Elimination \\
    $p \lor q \equiv q \lor q$ & \SetCell[r=2]{c} Commutative Laws \\
    $p \land q \equiv q \land q$ \\
    $(p \lor q) \lor r \equiv p \lor (q \lor r)$ & \SetCell[r=2]{c} Associative Laws \\
    $(p \land q) \land r \equiv p \land (q \land r)$ \\
    $p \lor (q \land r) \equiv (p \lor q) \land (p \lor r)$ & \SetCell[r=2]{c} Distributive Laws \\
    $p \land (q \lor r) \equiv (p \land q) \lor (p \land r)$ \\
    $\neg(p \land q) \equiv \neg p \lor \neg q$ & \SetCell[r=2]{c} DeMorgan's Laws \\
    $\neg(p \lor q) \equiv \neg p \land \neg q$ \\
    $p \lor (p \land q) \equiv p$ & \SetCell[r=2]{c} Absorption Laws \\
    $p \land (p \lor q) \equiv p$ \\
    $p \Rightarrow q \equiv q \lor \neg p$ & $\lor$ Restatement of Implication \\
    $p \Rightarrow q \equiv \neg q \Rightarrow \neg p$ & Contraposition \\
    $(p \Rightarrow q) \land (p \Rightarrow r) \equiv p \Rightarrow (q \land r)$ & Conjunction of Implications \\
    $(p \Rightarrow r) \land (q \Rightarrow r) \equiv (p \lor q) \Rightarrow r$ & Will Show This Below \\
    $(p \Rightarrow q) \lor (p \Rightarrow r) \equiv p \Rightarrow q \lor r$ & \SetCell[r=2]{c} Disjunction of Implications \\
    $(p \Rightarrow r) \lor (q \Rightarrow r) \equiv (p \land q) \Rightarrow r$ \\
    $p \Leftrightarrow q \equiv (p \Rightarrow q) \land (q \Rightarrow p)$ & Conjunction of Implications \\
    $p \Leftrightarrow q \equiv \neg p \Leftrightarrow \neg q$ & Negation Restatement \\
    $\neg(p \Leftrightarrow q) \equiv \neg p \Leftrightarrow q$ & Negation \\
    \bottomrule
  \end{tblr}
  \caption{List of Logical Equivalences}
  \label{fig:listOfLogicalEquivalences}
\end{figure}

\newpage

\vspace{12pt}
\hrule

\begin{center}
  Definitions
\end{center}
A \textbf{literal} is a variable or the complement of a variable.

A \textbf{Product Term} is a single literal or the logical product of two or more literals.

A \textbf{Sum of Products (SOP)} expression is a logical sum of product terms.

A \textbf{Sum Term} is a single literal or a logical sum of two or more literals.

A \textbf{Product of Sums (POS)} expression is a logical product of sum terms.

A \textbf{Normal Term} is a product or sum term in which no variable appears more than once.
\vspace{12pt}
\hrule

\subsection{Canonical Notation}
\label{ssec:canonicalNotation}

\begin{definition}{Product Term}
  Denoted with $m_n$, a product term is the product (and) of each input. Inputs default to $1$,
  with $0$ being the complement of the input.
\end{definition}

For example, if $A=1$, $B=0$, $C=1$, then $m_5 = A \overline{B} C$.
$n=5$ because $101$ occurs on the $5^{th}$ row. See Figure \ref{fig:productAndSumTerms}.

\begin{definition}{Sum Term}
  Denoted with $M_n$, a sum term is the sum (or) of each input. Input defaults to $0$,
  with $1$ being the complement of the input.
\end{definition}

For example, if $A=1$, $B=0$, $C=1$, then $M_5 = \overline{A} + B + \overline{C}$.
$n=5$ because $101$ occurs on the $5^{th}$ row. See Figure \ref{fig:productAndSumTerms}.

\begin{figure}[t]
  \centering
  \begin{tblr}{c|ccc|c|c}
    \toprule
    Row \# & A & B & C & Product Term & Sum Term \\
    \midrule
    0 & 0 & 0 & 0 & $\overline{A} \overline{B} \overline{C} = m_0$ & $A + B + C                                  = M_0$ \\
    1 & 0 & 0 & 1 & $\overline{A} \overline{B} C            = m_1$ & $A + B + \overline{C}                       = M_1$ \\
    2 & 0 & 1 & 0 & $\overline{A} B \overline{C}            = m_2$ & $A + \overline{B} + C                       = M_2$ \\
    3 & 0 & 1 & 1 & $\overline{A} B C                       = m_3$ & $A + \overline{B} + \overline{C}            = M_3$ \\
    0 & 1 & 0 & 0 & $A \overline{B} \overline{C}            = m_4$ & $\overline{A} + B + C                       = M_4$ \\
    1 & 1 & 0 & 1 & $A \overline{B} C                       = m_5$ & $\overline{A} + B + \overline{C}            = M_5$ \\
    2 & 1 & 1 & 0 & $A B \overline{C}                       = m_6$ & $\overline{A} + \overline{B} + C            = M_6$ \\
    3 & 1 & 1 & 1 & $A B C                                  = m_7$ & $\overline{A} + \overline{B} + \overline{C} = M_7$ \\
    \bottomrule
  \end{tblr}
  \caption{Product and Sum Terms}
  \label{fig:productAndSumTerms}
\end{figure}

\begin{definition}{Canonical Notation}
  Either Sum of Products (SOP) or Product of Sums (POS), canonical notation is used to express a boolean
  equation by showing where the outputs of a single type are.
\end{definition}

In Canonical Notation, the \textbf{Sum of Products} shows where the 1s are in the truth table,
with the \textbf{Product of Sums} showing where the 0s are.

\begin{figure}[H]
  \centering
  \begin{subfigure}[H]{0.45\textwidth}
    \centering
    \begin{gather*}
      F = \sum_{A,B,C}^{} (1, 3, 4, 6) \\
      F(A,B,C) = \sum_{}^{} m(1, 3, 4, 6)
    \end{gather*}
    \vspace{-20pt}
    \caption{List of Product Terms}
    \label{fig:listOfProductTerms}
  \end{subfigure}
  \begin{subfigure}[H]{0.45\textwidth}
    \centering
    \begin{gather*}
      F = \prod_{A,B,C}^{} (0, 2, 5, 7) \\
      F(A,B,C) = \prod_{}^{} m(0, 2, 5, 7)
    \end{gather*}
    \vspace{-20pt}
    \caption{List of Sum Terms}
    \label{fig:listOfSumTerms}
  \end{subfigure}
\end{figure}
\begin{gather*}
  \textup{\textbf{Canonical Sum (SOP)}:} \\
  F(A,B,C) = m_1 + m_3 + m_4 + m_6 = \overline{AB}C + \overline{A}BC + A \overline{BC} + AB \overline{C}
\end{gather*}
\begin{gather*}
  \textup{\textbf{Canonical Product (POS)}:} \\
  F(A,B,C) = m_0 + m_2 + m_5 + m_7 = (\overline{A} + \overline{B} + C)(A + \overline{B} + C) + A \overline{BC} + AB \overline{C}
\end{gather*}
\hrule
A \textbf{Don't Care} is any row that has input that is unimportant for the relevant use case.
To notate a "Don't Care", $dc$ is followed by each unimportant row in parentheses:
\begin{equation*}
  F(A,B,C) = \prod_{}^{} m(0, 2, 5, 7) + dc(1, 3, 4, 8)
\end{equation*}
\hrule


\end{document}
