\documentclass[12pt]{article}
\usepackage{darkmode}
%%%% GRAPHICS %%%%
\usepackage{tikz}
\usepackage{circuitikz}
\usetikzlibrary{arrows.meta}
\usepackage{tikz-3dplot}
\usepackage{graphicx}
\usepackage{pgfplots}
  \pgfplotsset{compat=1.18}
\usetikzlibrary{arrows}
\newcommand{\midarrow}{\tikz \draw[-triangle 90] (0,0) -- +(.1,0);}

%%%% FIGURES %%%%
\usepackage{subcaption}
\usepackage{wrapfig}
\usepackage{float}
\usepackage[skip=5pt, font=footnotesize]{caption}

%%%% FORMATTING %%%%
\usepackage{parskip}
\usepackage{tcolorbox}
\usepackage{ulem}
% \usepackage{fancyhdr}

%%%% TABLE FORMATTING %%%%
\usepackage{tabularray}
\UseTblrLibrary{booktabs}

%%%% MATH AND LOGIC %%%%
\usepackage{xifthen}
\usepackage{amsmath}
\usepackage{amssymb}
\usepackage{amsfonts}

%%%% TEXT AND SYMBOLS %%%%
\usepackage[T1]{fontenc}
\usepackage{textcomp}
\usepackage{gensymb}

%%%% OTHER %%%%
\usepackage{standalone}

%%%% LOGIC SYMBOLS %%%%
\newcommand*\xor{\oplus}

%%%% STYLES %%%%

% Packages
\usepackage{fullpage}
\usepackage{titlesec}
\usepackage[rgb]{xcolor}
\selectcolormodel{natural}
\usepackage{ninecolors}
\selectcolormodel{rgb}

% Colors
\definecolor{pg}{HTML}{24273A}
\definecolor{fg}{HTML}{FFFFFF}
\definecolor{bg}{HTML}{24273A}
\definecolor{re}{HTML}{d20f39}
\definecolor{gr}{HTML}{40a02b}
\definecolor{ye}{HTML}{df8e1d}
\definecolor{or}{HTML}{fe640b}
\definecolor{bl}{HTML}{1e66f5}
\definecolor{ma}{HTML}{8839ef}
\definecolor{cy}{HTML}{179299}
\definecolor{pi}{HTML}{ea76cb}

\usepackage{nameref}
\makeatletter
\newcommand*{\currentname}{\@currentlabelname}
\makeatother

\titleformat{\section}
  {\normalfont\scshape\Large\bfseries}
  {\thesection}
  {0.75em}
  {}

\titleformat{\subsection}
  {\normalfont\scshape\large\bfseries}
  {\thesubsection}
  {0.75em}
  {}

\titleformat{\subsubsection}
  {\normalfont\scshape\normalsize\bfseries}
  {\thesubsubsection}
  {0.75em}
  {}

% Formula
\newcounter{formula}[section]
\newenvironment{formula}[1]{
  \stepcounter{formula}
  \begin{tcolorbox}[
    standard jigsaw, % Allows opacity
    colframe={fg},
    boxrule=1px,
    colback=bg,
    opacityback=0,
    sharp corners,
    sidebyside,
    righthand width=18px,
    coltext={fg}
  ]
  \centering
  \textbf{\uline{#1}}
}{
  \tcblower
  \textbf{\thesection.\theformula}
  \end{tcolorbox}
}

% Definition
\newcounter{definition}[section]

\newenvironment{definition*}[1]{
  \begin{tcolorbox}[
    standard jigsaw, % Allows opacity
    colframe={fg},
    boxrule=1px,
    colback=bg,
    opacityback=0,
    sharp corners,
    coltext={fg}
  ]
  \textbf{#1 \hfill}
  \vspace{5px}
  \hrule
  \vspace{5px}
  \noindent
}{
  \end{tcolorbox}
}

\newenvironment{definition}[1]{
  \stepcounter{definition}
  \begin{tcolorbox}[
    standard jigsaw, % Allows opacity
    colframe={fg},
    boxrule=1px,
    colback=bg,
    opacityback=0,
    sharp corners,
    coltext={fg}
  ]
  \textbf{#1 \hfill \thesection.\thedefinition}
  \vspace{5px}
  \hrule
  \vspace{5px}
  \noindent
}{
  \end{tcolorbox}
}

% Example Problem
\newcounter{example}[section]
\newenvironment{example}{
  \stepcounter{example}
  \begin{tcolorbox}[
    standard jigsaw, % Allows opacity
    colframe={fg},
    boxrule=1px,
    colback=bg,
    opacityback=0,
    sharp corners,
    coltext={fg}
  ]
  \textbf{Example \hfill \thesection.\theexample}
  \vspace{5px}
  \hrule
  \vspace{5px}
  \noindent
}{
  \end{tcolorbox}
}

\tikzset{
  cubeBorder/.style=fg,
  cubeFilling/.style={fg!20!bg, opacity=0.25},
  gridLine/.style={very thin, gray},
  graphLine/.style={-latex, thick, fg},
}

\pgfplotsset{
  basicAxis/.style={
    grid,
    major grid style={line width=.2pt,draw=fg!50!bg},
    axis lines = box,
    axis line style = {line width = 1px},
  }
}

%%%% REFERENCES %%%%
\usepackage{hyperref}
\hypersetup{
  colorlinks  = true,
  linkcolor   = pi,
  anchorcolor = pi,
  citecolor   = pi,
  filecolor   = pi,
  menucolor   = pi,
  runcolor    = pi,
  urlcolor    = pi,
}

\author{Ethan Anthony}


\title{Lecture 003}
\date{September 30, 2024}

\begin{document}

\newpage
\section{Karnaugh Maps}
\label{sec:karnaughMaps}

A \textbf{Karnaugh Map} is a graphical representation of a boolean function's truth table.
A function of $n$ variables will have a map of $2^n$ cells, with each cell representing a
unique combination of the inputs of the function, similar to a truth table.

\begin{definition}{Grey Code}
  A method of sequencing binary data such that each successive number only changes by a single
  bit. For example, with a 3-digit binary number, the sequence would follow as: $000$, $001$, $011$, $010$, $110$, $100$, etc.
\end{definition}

When constructing a Karnaugh Map, the row and column labels must follow a \textbf{Grey Code}
sequence, this is so that the map is able to "wrap around" (the end of a row/column is adjacent to the beginning).

Each cell is doubly labeled, first with its corresponding row number in the truth table in
the upper right, and second with its logical value in the center.

\begin{figure}[H]
  \centering
  \begin{subfigure}[H]{0.3\textwidth}
    \centering
    \begin{tblr}{c|ccc|c}
      \toprule
      Row & $A$ & $B$ & $C$ & $F$ \\
      \midrule
      0 & 0 & 0 & 0 & 0 \\
      1 & 0 & 0 & 1 & 0 \\
      2 & 0 & 1 & 0 & 1 \\
      3 & 0 & 1 & 1 & 1 \\
      4 & 1 & 0 & 0 & 0 \\
      5 & 1 & 0 & 1 & 1 \\
      6 & 1 & 1 & 0 & 0 \\
      7 & 1 & 1 & 1 & 1 \\
      \bottomrule
    \end{tblr}
    \caption{Truth Table}
    \label{fig:truthTable1111}
  \end{subfigure}
  \begin{subfigure}[H]{0.6\textwidth}
    \centering
    \includestandalone{figures/fig_004}
    \caption{Karnaugh Map}
    \label{fig:004}
  \end{subfigure}
  \caption{Karnaugh Map Synthesis}
  \label{fig:karnaughMapsSynthesis}
\end{figure}

\begin{wrapfigure}[7]{l}{0.35\textwidth}
  \centering
  \includestandalone{figures/fig_005}
  \caption{Grouping Implicants}
  \label{fig:005}
\end{wrapfigure}

To construct a SOP expression from a Karnaugh Map, 1-cells are group together into \textbf{prime implicants}
in groups of size $2^k$. Once grouped, expressions can be derived for each prime implicant
based on the input values that don't change throughout the group.
\begin{align*}
  F(A,B,C) = \sum_{}^{} (2,3,5,7) = \color{or!50!fg}\overline{A}B \color{fg} + \color{ma!50!fg}AC
\end{align*}

\begin{definition}{Implicant}
  A product term of a complete function such that the product term implies the function.
  A \textbf{prime implicant} is the largest set that can be made to include any
  specific 1-cell in a SOP-expression. An \textbf{essential prime implicant} is a prime
  implicant that covers one or more \textit{distinguished 1-cells}.
\end{definition}

\begin{wrapfigure}[7]{r}{0.35\textwidth}
  \centering
  \includestandalone{figures/fig_006}
  \caption{Grouping Implicates}
  \label{fig:006}
\end{wrapfigure}

Similarly, to construct a POS expression from a Karnaugh Map, 0-cells are group together into \textbf{prime implicates}
in groups of size $2^k$. Once grouped, expressions can be derived for each prime implicant
based on the input values that don't change throughout the group.
\begin{align*}
  F(A,B,C) = \prod_{}^{} (0,1,4,6) = (\color{ye!50!fg}A+B\color{fg})(\color{bl!50!fg}\overline{A} + C \color{fg})
\end{align*}

\begin{definition}{Implicate}
  A sum term of a complete function such that the sum term implies the function.
\end{definition}

\begin{definition}{Distinguished 1/0-Cell}
  An input combination that is only covered by a single prime implicant/implicate.
\end{definition}
\vspace{-20pt}
\begin{align*}
  f(w,x,y,z) = \sum_{}^{} m(0,1,2,4,5,6,9,10,11,13,14,15)
\end{align*}
\begin{figure}[H]
  \vspace{-20pt}
  \begin{subfigure}[H]{0.325\textwidth}
    \includestandalone{figures/fig_007}
  \end{subfigure}
  \begin{subfigure}[H]{0.325\textwidth}
    \includestandalone{figures/fig_008}
  \end{subfigure}
  \begin{subfigure}[H]{0.325\textwidth}
    \includestandalone{figures/fig_009}
  \end{subfigure}
  \caption{Boolean Expression with Multiple Solutions}
  \label{fig:009}
\end{figure}
It is possible that a boolean expression can be grouped into a Karnaugh Map in different ways, providing multiple
solutions. Assuming the maps are grouped correctly, there is no single best solution.

When constructing a Karnaugh Map of an expression that includes "don't cares", two rules should
be followed to construct the minimum SOP or POS expression:
\begin{itemize}
  \itemsep0em
  \item Include don't cares in the prime implicates/implicants to make them as big as possible
  \item Don't create an implicate/implicant containing \textbf{only} don't cares
\end{itemize}
\begin{align*}
  F(w,x,y,z) = \sum_{}^{} (1,2,3,5,7) + d(10,11,12,13,14,15)
\end{align*}
\begin{figure}[H]
  \vspace{-20pt}
  \centering
  \begin{subfigure}[H]{0.45\textwidth}
    \centering
    \includestandalone{figures/fig_010}
  \end{subfigure}
  \begin{subfigure}[H]{0.45\textwidth}
    \centering
    \includestandalone{figures/fig_011}
  \end{subfigure}
  \caption{Don't Care Karnaugh Map}
  \label{fig:010}
\end{figure}

\end{document}
