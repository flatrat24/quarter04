% \begin{tblr}{ccccc|c}
%   \toprule
%   15 & $\div$ & 2 & $=$ & 7 & 1    \\
%   \bottomrule
%   \SetCell[c=4]{r} Result: &&&&& 0111 \\
% \end{tblr}

\documentclass[12pt]{article}

%%%% GRAPHICS %%%%
\usepackage{tikz}
\usepackage{circuitikz}
\usetikzlibrary{arrows.meta}
\usepackage{tikz-3dplot}
\usepackage{graphicx}
\usepackage{pgfplots}
  \pgfplotsset{compat=1.18}
\usetikzlibrary{arrows}
\newcommand{\midarrow}{\tikz \draw[-triangle 90] (0,0) -- +(.1,0);}

%%%% FIGURES %%%%
\usepackage{subcaption}
\usepackage{wrapfig}
\usepackage{float}
\usepackage[skip=5pt, font=footnotesize]{caption}

%%%% FORMATTING %%%%
\usepackage{parskip}
\usepackage{tcolorbox}
\usepackage{ulem}
% \usepackage{fancyhdr}

%%%% TABLE FORMATTING %%%%
\usepackage{tabularray}
\UseTblrLibrary{booktabs}

%%%% MATH AND LOGIC %%%%
\usepackage{xifthen}
\usepackage{amsmath}
\usepackage{amssymb}
\usepackage{amsfonts}

%%%% TEXT AND SYMBOLS %%%%
\usepackage[T1]{fontenc}
\usepackage{textcomp}
\usepackage{gensymb}

%%%% OTHER %%%%
\usepackage{standalone}

%%%% LOGIC SYMBOLS %%%%
\newcommand*\xor{\oplus}

%%%% STYLES %%%%

% Packages
\usepackage{fullpage}
\usepackage{titlesec}
\usepackage[rgb]{xcolor}
\selectcolormodel{natural}
\usepackage{ninecolors}
\selectcolormodel{rgb}

% Colors
\definecolor{pg}{HTML}{24273A}
\definecolor{fg}{HTML}{FFFFFF}
\definecolor{bg}{HTML}{24273A}
\definecolor{re}{HTML}{d20f39}
\definecolor{gr}{HTML}{40a02b}
\definecolor{ye}{HTML}{df8e1d}
\definecolor{or}{HTML}{fe640b}
\definecolor{bl}{HTML}{1e66f5}
\definecolor{ma}{HTML}{8839ef}
\definecolor{cy}{HTML}{179299}
\definecolor{pi}{HTML}{ea76cb}

\usepackage{nameref}
\makeatletter
\newcommand*{\currentname}{\@currentlabelname}
\makeatother

\titleformat{\section}
  {\normalfont\scshape\Large\bfseries}
  {\thesection}
  {0.75em}
  {}

\titleformat{\subsection}
  {\normalfont\scshape\large\bfseries}
  {\thesubsection}
  {0.75em}
  {}

\titleformat{\subsubsection}
  {\normalfont\scshape\normalsize\bfseries}
  {\thesubsubsection}
  {0.75em}
  {}

% Formula
\newcounter{formula}[section]
\newenvironment{formula}[1]{
  \stepcounter{formula}
  \begin{tcolorbox}[
    standard jigsaw, % Allows opacity
    colframe={fg},
    boxrule=1px,
    colback=bg,
    opacityback=0,
    sharp corners,
    sidebyside,
    righthand width=18px,
    coltext={fg}
  ]
  \centering
  \textbf{\uline{#1}}
}{
  \tcblower
  \textbf{\thesection.\theformula}
  \end{tcolorbox}
}

% Definition
\newcounter{definition}[section]

\newenvironment{definition*}[1]{
  \begin{tcolorbox}[
    standard jigsaw, % Allows opacity
    colframe={fg},
    boxrule=1px,
    colback=bg,
    opacityback=0,
    sharp corners,
    coltext={fg}
  ]
  \textbf{#1 \hfill}
  \vspace{5px}
  \hrule
  \vspace{5px}
  \noindent
}{
  \end{tcolorbox}
}

\newenvironment{definition}[1]{
  \stepcounter{definition}
  \begin{tcolorbox}[
    standard jigsaw, % Allows opacity
    colframe={fg},
    boxrule=1px,
    colback=bg,
    opacityback=0,
    sharp corners,
    coltext={fg}
  ]
  \textbf{#1 \hfill \thesection.\thedefinition}
  \vspace{5px}
  \hrule
  \vspace{5px}
  \noindent
}{
  \end{tcolorbox}
}

% Example Problem
\newcounter{example}[section]
\newenvironment{example}{
  \stepcounter{example}
  \begin{tcolorbox}[
    standard jigsaw, % Allows opacity
    colframe={fg},
    boxrule=1px,
    colback=bg,
    opacityback=0,
    sharp corners,
    coltext={fg}
  ]
  \textbf{Example \hfill \thesection.\theexample}
  \vspace{5px}
  \hrule
  \vspace{5px}
  \noindent
}{
  \end{tcolorbox}
}

\tikzset{
  cubeBorder/.style=fg,
  cubeFilling/.style={fg!20!bg, opacity=0.25},
  gridLine/.style={very thin, gray},
  graphLine/.style={-latex, thick, fg},
}

\pgfplotsset{
  basicAxis/.style={
    grid,
    major grid style={line width=.2pt,draw=fg!50!bg},
    axis lines = box,
    axis line style = {line width = 1px},
  }
}

%%%% REFERENCES %%%%
\usepackage{hyperref}
\hypersetup{
  colorlinks  = true,
  linkcolor   = pi,
  anchorcolor = pi,
  citecolor   = pi,
  filecolor   = pi,
  menucolor   = pi,
  runcolor    = pi,
  urlcolor    = pi,
}

\author{Ethan Anthony}


\title{Lecture 004}
\date{October 05, 2024}

\begin{document}
\newpage
\section{Binary Systems and Encoding}
\label{sec:binarySystemsAndEncoding}

\subsection{Number Format}
\label{ssec:numberFormat}

Inside of a computer, numbers are represented as a stream of digits in the computer's memory.
The format the number is stored must be known for a computer to be able to use the value
properly.

\begin{definition}{Number Format}
  The components of the way a number is expressed that determine how it is encoded and 
  decoded. These components include:
  \begin{itemize}
    \itemsep0em
    \item Number base
    \item Number of digits in the number
    \item Integer or floating point
    \item Where the radix point is
    \item The complement system, if any, being used
  \end{itemize}
\end{definition}

\begin{center}
  {\large \textbf{Signed Magnitude}}
\end{center}

\begin{center}
  {\large \textbf{Binary Signed-Magnitude}}
\end{center}

\begin{center}
  {\large \textbf{Complement Systems}}
\end{center}
A complement system \textit{negates} a number by taking the complement of it according
to the rules of a specific complement system.

Given a (fixed) number of ways we uniquely can represent values in a number system, we
will assign a portion of the representations to represent negative (or inverse) values and
develop a system of arithmetic using this new number system.

Complement systems are useful because addition of One's or Two's Complemented numbers is
done simply by adding each digit as usual. This type of binary addition is sign-agnostic.

\begin{definition}{One's Complement}
  The complement of a number according to One's Complement follows the pattern of subracting
  each digit of the number from the radix ($r$) minus one ($r-1$).
  \begin{center}
    \begin{tblr}{r|r|r|r}
      $0=0000_2$  & $2=0010_2$  & $7=0111_2$  & $5=0101_2$  \\
      $-0=1111_2$ & $-2=1101_2$ & $-7=0111_2$ & $-7=1010_2$ \\
    \end{tblr}
  \end{center}
\end{definition}

One's complement is a symmetric number system. Both $0$ and $-0$ have different expressions
under One's complement despite being equivalent. Thus, the system is symmetric in terms of
the number of negatives and positives that can be expressed given a bit size. Two's complement
is asymmetric as it expresses one more negative than it does positive.

\begin{definition}{Two's Complement}
  Given the number $a$, the radix $r$, and the number of digits $n$, the Two's complement
  of $a$ is defined as any of the following:
  \begin{align*}
    (-a)_2 &= [\textup{one's complement of }a] + 1 \\
           &= r^n - a                              \\
           &= [\textup{each digit of }a \rightarrow (r-1)-\textup{digit}] + 1 \\
           &= \textup{complement each digit to the left of the rightmost one}
  \end{align*}
  \hrule
  \vspace{12pt}
  \begin{center}
    \begin{tblr}{r|r|r|r}
      $0=0000_2$  & $2=0010_2$  & $7=0111_2$  & $5=0101_2$  \\
      $-0=0000_2$ & $-2=1110_2$ & $-7=1001_2$ & $-7=1011_2$ \\
    \end{tblr}
  \end{center}
\end{definition}

\begin{figure}[H]
  \centering
  \begin{tblr}{c|ccc}
    \toprule
    \textbf{Decimal} & \textbf{Two's Complement} & \textbf{One's Complement} & \textbf{Signed Magnitude} \\
    \midrule
     7 & 0111 & 0111 & 0111 \\
     6 & 0110 & 0110 & 0110 \\
     5 & 0101 & 0101 & 0101 \\
     4 & 0100 & 0100 & 0100 \\
     3 & 0011 & 0011 & 0011 \\
     2 & 0010 & 0010 & 0010 \\
     1 & 0001 & 0001 & 0001 \\
     0 & 0000 & 0000 & 0000 \\
    -0 & ---  & 1111 & 1000 \\
    -1 & 1111 & 1110 & 1001 \\
    -2 & 1110 & 1101 & 1010 \\
    -3 & 1101 & 1100 & 1011 \\
    -4 & 1100 & 1011 & 1100 \\
    -5 & 1011 & 1010 & 1101 \\
    -6 & 1010 & 1001 & 1110 \\
    -7 & 1001 & 1000 & 1111 \\
    -8 & 1000 & ---  & ---  \\
    \bottomrule
  \end{tblr}
  \caption{Binary Number Systems in 4-Bits}
  \label{fig:binaryNumberSystemsIn4Bits}
\end{figure}

\subsection{Binary Encoding}
\label{ssec:binaryEncoding}

\begin{center}
  {\large \textbf{ASCII Character Codes}}
\end{center}
ASCII (\textbf{A}merican \textbf{S}tandard \textbf{C}ode for \textbf{I}nformation
\textbf{I}nterchange) is a 7-bit coding standard that identifies letters, numbers, and
various symbols of a teletype machine. With $2^7$, or $128$, possible combinations, that
many possible values can be encoded as binary.

When encoding in ASCII, a \textbf{parity bit} can be added to the front to ensure that
there is a number of $1$s in the code that is either always even or always odd. This is
done as a check to make sure that the codes being sent and received make sense.

\end{document}
