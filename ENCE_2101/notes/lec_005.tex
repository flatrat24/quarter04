\documentclass[12pt]{article}

%%%% GRAPHICS %%%%
\usepackage{tikz}
\usepackage{circuitikz}
\usetikzlibrary{arrows.meta}
\usepackage{tikz-3dplot}
\usepackage{graphicx}
\usepackage{pgfplots}
  \pgfplotsset{compat=1.18}
\usetikzlibrary{arrows}
\newcommand{\midarrow}{\tikz \draw[-triangle 90] (0,0) -- +(.1,0);}

%%%% FIGURES %%%%
\usepackage{subcaption}
\usepackage{wrapfig}
\usepackage{float}
\usepackage[skip=5pt, font=footnotesize]{caption}

%%%% FORMATTING %%%%
\usepackage{parskip}
\usepackage{tcolorbox}
\usepackage{ulem}
% \usepackage{fancyhdr}

%%%% TABLE FORMATTING %%%%
\usepackage{tabularray}
\UseTblrLibrary{booktabs}

%%%% MATH AND LOGIC %%%%
\usepackage{xifthen}
\usepackage{amsmath}
\usepackage{amssymb}
\usepackage{amsfonts}

%%%% TEXT AND SYMBOLS %%%%
\usepackage[T1]{fontenc}
\usepackage{textcomp}
\usepackage{gensymb}

%%%% OTHER %%%%
\usepackage{standalone}

%%%% LOGIC SYMBOLS %%%%
\newcommand*\xor{\oplus}

%%%% STYLES %%%%

% Packages
\usepackage{fullpage}
\usepackage{titlesec}
\usepackage[rgb]{xcolor}
\selectcolormodel{natural}
\usepackage{ninecolors}
\selectcolormodel{rgb}

% Colors
\definecolor{pg}{HTML}{24273A}
\definecolor{fg}{HTML}{FFFFFF}
\definecolor{bg}{HTML}{24273A}
\definecolor{re}{HTML}{d20f39}
\definecolor{gr}{HTML}{40a02b}
\definecolor{ye}{HTML}{df8e1d}
\definecolor{or}{HTML}{fe640b}
\definecolor{bl}{HTML}{1e66f5}
\definecolor{ma}{HTML}{8839ef}
\definecolor{cy}{HTML}{179299}
\definecolor{pi}{HTML}{ea76cb}

\usepackage{nameref}
\makeatletter
\newcommand*{\currentname}{\@currentlabelname}
\makeatother

\titleformat{\section}
  {\normalfont\scshape\Large\bfseries}
  {\thesection}
  {0.75em}
  {}

\titleformat{\subsection}
  {\normalfont\scshape\large\bfseries}
  {\thesubsection}
  {0.75em}
  {}

\titleformat{\subsubsection}
  {\normalfont\scshape\normalsize\bfseries}
  {\thesubsubsection}
  {0.75em}
  {}

% Formula
\newcounter{formula}[section]
\newenvironment{formula}[1]{
  \stepcounter{formula}
  \begin{tcolorbox}[
    standard jigsaw, % Allows opacity
    colframe={fg},
    boxrule=1px,
    colback=bg,
    opacityback=0,
    sharp corners,
    sidebyside,
    righthand width=18px,
    coltext={fg}
  ]
  \centering
  \textbf{\uline{#1}}
}{
  \tcblower
  \textbf{\thesection.\theformula}
  \end{tcolorbox}
}

% Definition
\newcounter{definition}[section]

\newenvironment{definition*}[1]{
  \begin{tcolorbox}[
    standard jigsaw, % Allows opacity
    colframe={fg},
    boxrule=1px,
    colback=bg,
    opacityback=0,
    sharp corners,
    coltext={fg}
  ]
  \textbf{#1 \hfill}
  \vspace{5px}
  \hrule
  \vspace{5px}
  \noindent
}{
  \end{tcolorbox}
}

\newenvironment{definition}[1]{
  \stepcounter{definition}
  \begin{tcolorbox}[
    standard jigsaw, % Allows opacity
    colframe={fg},
    boxrule=1px,
    colback=bg,
    opacityback=0,
    sharp corners,
    coltext={fg}
  ]
  \textbf{#1 \hfill \thesection.\thedefinition}
  \vspace{5px}
  \hrule
  \vspace{5px}
  \noindent
}{
  \end{tcolorbox}
}

% Example Problem
\newcounter{example}[section]
\newenvironment{example}{
  \stepcounter{example}
  \begin{tcolorbox}[
    standard jigsaw, % Allows opacity
    colframe={fg},
    boxrule=1px,
    colback=bg,
    opacityback=0,
    sharp corners,
    coltext={fg}
  ]
  \textbf{Example \hfill \thesection.\theexample}
  \vspace{5px}
  \hrule
  \vspace{5px}
  \noindent
}{
  \end{tcolorbox}
}

\tikzset{
  cubeBorder/.style=fg,
  cubeFilling/.style={fg!20!bg, opacity=0.25},
  gridLine/.style={very thin, gray},
  graphLine/.style={-latex, thick, fg},
}

\pgfplotsset{
  basicAxis/.style={
    grid,
    major grid style={line width=.2pt,draw=fg!50!bg},
    axis lines = box,
    axis line style = {line width = 1px},
  }
}

%%%% REFERENCES %%%%
\usepackage{hyperref}
\hypersetup{
  colorlinks  = true,
  linkcolor   = pi,
  anchorcolor = pi,
  citecolor   = pi,
  filecolor   = pi,
  menucolor   = pi,
  runcolor    = pi,
  urlcolor    = pi,
}

\author{Ethan Anthony}


\title{Lecture 005}
\date{October 15, 2024}

\begin{document}

\newpage
\section{Adders and Comparators}
\label{sec:addersAndComparators}

\subsection{Adder}
\label{ssec:adder}

\begin{wrapfigure}[10]{r}{0.2\textwidth}
  \centering
  \begin{tblr}{cc|c|c}
    \toprule
    $x$ & $y$ & $S$ & $C$ \\
    \midrule
    0 & 0 & 0 & 0 \\
    0 & 1 & 1 & 0 \\
    1 & 0 & 1 & 0 \\
    1 & 1 & 0 & 1 \\
    \bottomrule
  \end{tblr}
  \caption{Half Adder Truth Table}
  \label{fig:halfAdderTruthTable}
\end{wrapfigure}

An \textbf{Adder} performs addition of bits using the rules for addition defined in the
binary number system. With an adder, subtraction (addition of the complement), multiplication
(repeated addition), and division (repeated addition of the complement) can all be performed.

Given two inputs, a \textbf{Half Adder} combines the bits of each input to produce two outputs:
the sum bit ($S$) and the carry bit ($C$). With any two bits, the sum will only be $1$ if
one of the two is a $1$, thus a XOR gate would be used for the sum. The carry will only
result in a $1$ if both inputs are $1$, thus an AND gate would be used. The circuit is
shown in Figure \ref{fig:013}.

\begin{figure}[H]
  \centering
  \includestandalone{figures/fig_013}
  \caption{Half Adder}
  \label{fig:013}
\end{figure}

A \textbf{Full Adder} performs the addition of three digits. Conceptually, adders can be
linked in a chain such that each adder has three inputs: two unique inputs and the carry
bit from the previous link in the chain. Thus, a full adder must accommodate three inputs.
A full adder can be seen in Figure \ref{fig:014}.

The Karnaugh Map representation of each term is as in Figure \ref{fig:adderKarnaugh}. Two
different outputs are defined: one for the sum bit ($S$) and one for the carry bit ($C$).
Based on these K-Maps: $S = c \oplus (x \oplus y)$ and $C = xy+yc+cx$.

\begin{figure}[H]
  \begin{subfigure}[H]{0.45\textwidth}
    \centering
    \includestandalone{figures/fig_015}
    \caption{Sum Bit}
    \label{fig:015}
  \end{subfigure}
  \begin{subfigure}[H]{0.45\textwidth}
    \centering
    \includestandalone{figures/fig_016}
    \caption{Carry Bit}
    \label{fig:016}
  \end{subfigure}
  \caption{Karnaugh Maps of the Sum and Carry Bits}
  \label{fig:adderKarnaugh}
\end{figure}

\newpage
\vfill
\begin{figure}[H]
  \centering
  \includestandalone{figures/fig_014}
  \caption{Full Adder}
  \label{fig:014}
\end{figure}
\vfill
\newpage



\end{document}
