\documentclass[12pt]{article}

%%%% GRAPHICS %%%%
\usepackage{tikz}
\usepackage{circuitikz}
\usetikzlibrary{arrows.meta}
\usepackage{tikz-3dplot}
\usepackage{graphicx}
\usepackage{pgfplots}
  \pgfplotsset{compat=1.18}
\usetikzlibrary{arrows}
\newcommand{\midarrow}{\tikz \draw[-triangle 90] (0,0) -- +(.1,0);}

%%%% FIGURES %%%%
\usepackage{subcaption}
\usepackage{wrapfig}
\usepackage{float}
\usepackage[skip=5pt, font=footnotesize]{caption}

%%%% FORMATTING %%%%
\usepackage{parskip}
\usepackage{tcolorbox}
\usepackage{ulem}
% \usepackage{fancyhdr}

%%%% TABLE FORMATTING %%%%
\usepackage{tabularray}
\UseTblrLibrary{booktabs}

%%%% MATH AND LOGIC %%%%
\usepackage{xifthen}
\usepackage{amsmath}
\usepackage{amssymb}
\usepackage{amsfonts}

%%%% TEXT AND SYMBOLS %%%%
\usepackage[T1]{fontenc}
\usepackage{textcomp}
\usepackage{gensymb}

%%%% OTHER %%%%
\usepackage{standalone}

%%%% LOGIC SYMBOLS %%%%
\newcommand*\xor{\oplus}

%%%% STYLES %%%%

% Packages
\usepackage{fullpage}
\usepackage{titlesec}
\usepackage[rgb]{xcolor}
\selectcolormodel{natural}
\usepackage{ninecolors}
\selectcolormodel{rgb}

% Colors
\definecolor{pg}{HTML}{24273A}
\definecolor{fg}{HTML}{FFFFFF}
\definecolor{bg}{HTML}{24273A}
\definecolor{re}{HTML}{d20f39}
\definecolor{gr}{HTML}{40a02b}
\definecolor{ye}{HTML}{df8e1d}
\definecolor{or}{HTML}{fe640b}
\definecolor{bl}{HTML}{1e66f5}
\definecolor{ma}{HTML}{8839ef}
\definecolor{cy}{HTML}{179299}
\definecolor{pi}{HTML}{ea76cb}

\usepackage{nameref}
\makeatletter
\newcommand*{\currentname}{\@currentlabelname}
\makeatother

\titleformat{\section}
  {\normalfont\scshape\Large\bfseries}
  {\thesection}
  {0.75em}
  {}

\titleformat{\subsection}
  {\normalfont\scshape\large\bfseries}
  {\thesubsection}
  {0.75em}
  {}

\titleformat{\subsubsection}
  {\normalfont\scshape\normalsize\bfseries}
  {\thesubsubsection}
  {0.75em}
  {}

% Formula
\newcounter{formula}[section]
\newenvironment{formula}[1]{
  \stepcounter{formula}
  \begin{tcolorbox}[
    standard jigsaw, % Allows opacity
    colframe={fg},
    boxrule=1px,
    colback=bg,
    opacityback=0,
    sharp corners,
    sidebyside,
    righthand width=18px,
    coltext={fg}
  ]
  \centering
  \textbf{\uline{#1}}
}{
  \tcblower
  \textbf{\thesection.\theformula}
  \end{tcolorbox}
}

% Definition
\newcounter{definition}[section]

\newenvironment{definition*}[1]{
  \begin{tcolorbox}[
    standard jigsaw, % Allows opacity
    colframe={fg},
    boxrule=1px,
    colback=bg,
    opacityback=0,
    sharp corners,
    coltext={fg}
  ]
  \textbf{#1 \hfill}
  \vspace{5px}
  \hrule
  \vspace{5px}
  \noindent
}{
  \end{tcolorbox}
}

\newenvironment{definition}[1]{
  \stepcounter{definition}
  \begin{tcolorbox}[
    standard jigsaw, % Allows opacity
    colframe={fg},
    boxrule=1px,
    colback=bg,
    opacityback=0,
    sharp corners,
    coltext={fg}
  ]
  \textbf{#1 \hfill \thesection.\thedefinition}
  \vspace{5px}
  \hrule
  \vspace{5px}
  \noindent
}{
  \end{tcolorbox}
}

% Example Problem
\newcounter{example}[section]
\newenvironment{example}{
  \stepcounter{example}
  \begin{tcolorbox}[
    standard jigsaw, % Allows opacity
    colframe={fg},
    boxrule=1px,
    colback=bg,
    opacityback=0,
    sharp corners,
    coltext={fg}
  ]
  \textbf{Example \hfill \thesection.\theexample}
  \vspace{5px}
  \hrule
  \vspace{5px}
  \noindent
}{
  \end{tcolorbox}
}

\tikzset{
  cubeBorder/.style=fg,
  cubeFilling/.style={fg!20!bg, opacity=0.25},
  gridLine/.style={very thin, gray},
  graphLine/.style={-latex, thick, fg},
}

\pgfplotsset{
  basicAxis/.style={
    grid,
    major grid style={line width=.2pt,draw=fg!50!bg},
    axis lines = box,
    axis line style = {line width = 1px},
  }
}

%%%% REFERENCES %%%%
\usepackage{hyperref}
\hypersetup{
  colorlinks  = true,
  linkcolor   = pi,
  anchorcolor = pi,
  citecolor   = pi,
  filecolor   = pi,
  menucolor   = pi,
  runcolor    = pi,
  urlcolor    = pi,
}

\author{Ethan Anthony}


\title{Lecture 004}
\date{October 05, 2024}

\begin{document}

\newpage
\section{Sequences}
\label{sec:sequences}

\subsection{Sequence Definition}
\label{ssec:sequenceDefinition}

\begin{definition}{Sequence}
  A sequence is an \textbf{ordered} list of elements. The elements in a sequence are derived
  from a subset of integers, but aren't necessarily integers themselves.
\end{definition}

Some sequences, such as the Fibonacci Sequence, are defined recursively. That is each term
in the sequence is defined in terms of some previous terms. The Fibonacci Sequence, for example,
defines each term as the sum of the two previous terms. These recursively defined sequences
are said to have \textbf{Recursive Relations} between their terms.

\subsection{Types of Sequences}
\label{ssec:typesOfSequences}

\begin{center}
  {\Large \textbf{Geometric Progression}}
\end{center}

\begin{figure}[H]
  \begin{subfigure}[H]{0.55\textwidth}
    A geometric progression is a sequence of the form:
    \begin{equation*}
      ar^0, ar^1, ar^2, ar^3, ..., ar^n
    \end{equation*}
    where the \textit{\textbf{initial term a}} and the \textbf{\textit{common ratio r}} are both
    real numbers.
  \end{subfigure}
  \begin{subfigure}[H]{0.4\textwidth}
    \begin{center}
      \begin{tblr}{ccl}
        \toprule
        \textbf{a} & \textbf{r} & \textbf{Sequence} \\
        \midrule
        $1$ & $-1$ & $\{1, -1, 1, -1, 1, -1, ...\}$ \\
        $2$ & $5$  & $\{2, 10, 50, 250, 1250, ...\}$ \\
        $6$ & $\frac{1}{3}$ & $\{6, 2, \frac{2}{3}, \frac{2}{9}, \frac{2}{27}, \frac{2}{81}, \frac{2}{243}, ...\}$ \\
        \bottomrule
      \end{tblr}
    \end{center}
  \end{subfigure}
\end{figure}

\begin{center}
  {\Large \textbf{Artithmetic Progression}}
\end{center}

\begin{figure}[H]
  \begin{subfigure}[H]{0.4\textwidth}
    \begin{center}
      \begin{tblr}{ccl}
        \toprule
        \textbf{a} & \textbf{d} & \textbf{Sequence} \\
        \midrule
        $1$ & $-1$ & $\{1, 0, -1, -2, -3, ...\}$ \\
        $2$ & $5$  & $\{2, 7, 12, 17, 22, 27, ...\}$ \\
        $6$ & $\frac{1}{3}$ & $\{6, \frac{19}{3}, \frac{20}{3}, \frac{21}{3}, \frac{22}{3}, \frac{23}{3}, ...\}$ \\
        \bottomrule
      \end{tblr}
    \end{center}
  \end{subfigure}
  \begin{subfigure}[H]{0.55\textwidth}
    A arithmetic progression is a sequence of the form:
    \begin{equation*}
      a + 0d, a + 1d, a + 2d, a + 3d, ..., a + nd
    \end{equation*}
    where the \textit{\textbf{initial term a}} and the \textbf{\textit{common difference d}} are both
    real numbers.
  \end{subfigure}
\end{figure}

\begin{center}
  {\Large \textbf{Fibonacci Sequence}}
\end{center}
The Fibonacci Sequence defined the $n^{th}$ term as the sum of the $n-1^{th}$ term and
the $n-2^{th}$ term. The first two terms cannot be defined recursively, and are usually
defined as $F_0 = 0$ and $F_1 = 1$.
\begin{equation*}
  F_0, F_1, (F_0+F_1), (F_1+F_2), (F_2+F_3), (F_3+F_4), ...
\end{equation*}
The Fibonacci Sequence with $F_0 = 0$ and $F_1 = 1$ is as follows:
\begin{equation*}
  0, 1, 1, 2, 3, 5, 8, 13, 21, 34, ...
\end{equation*}


\end{document}
