\documentclass[12pt]{article}

\input{../../xlatex/imports/preamble}

\title{Lecture 002}
\date{September 19, 2024}

\begin{document}

\newpage
\section{Sets}
\label{sec:sets}

\subsection{Introduction to Sets}
\label{ssec:introductionToSets}

\begin{definition}{Set}
  A collection of objects defined either by listing its members or by defining a pattern.
\end{definition}

The ways to specifically define certain elements as part of a set is as follows:
\begin{center}
  \begin{tblr}{cc}
    \toprule
    Notation & Meaning \\
    \midrule
    $5 \in A$ & $5$ is a member of set $A$ \\
    $3 \notin B$ & $3$ is not a member of set $B$ \\
    $C = \{1, 2, 3\}$ & Set $C$ contains $1$, $2$, and $3$ \\
    \bottomrule
  \end{tblr}
\end{center}

Additionally, set builder notation allows you to construct a set based on a defined pattern:
\begin{figure}[H]
  \vspace{-20pt}
  \begin{align*}
    A &= \{2k+1|k \in \mathbb{Z}\} \\
    A &= \{n\ |\ \exists k \in \mathbb{Z} \textup{ for which } n = 2k + 1\}
  \end{align*}
  \caption{Defines Set $A$ as all odd numbers}
  \label{fig:setA}
\end{figure}
Sets can also be defined to be empty. $A = \{\}$, $A = \varnothing$, and $A = \{\varnothing\}$
all define the set $A$ to be empty. Additionally, this can be done through set builder notation:
\begin{equation*}
  A = \{x\ |\ \forall x, x \ \exists !\  U\}
\end{equation*}

\hrule
\vspace{12pt}

\subsection{Set Operations}
\label{ssec:setOperations}

Once a set exists, operations can be performed on the set at large. Four basic operations
are the union of two sets ($\cup$), the intersection of two sets ($\cap$), the complement
of a set ($\complement$), and the difference of two sets (\textbf{\textbackslash}).

\begin{figure}[H]
  \centering
  \begin{subfigure}[H]{0.45\textwidth}
    \centering
    \includestandalone{figures/fig_001}
    \caption{$A \cup B$}
    \label{fig:setUnion}
  \end{subfigure}
  \begin{subfigure}[H]{0.45\textwidth}
    \centering
    \includestandalone{figures/fig_002}
    \caption{$A \cap B$}
    \label{fig:setIntersection}
  \end{subfigure}
\end{figure}
\begin{figure}[H]
  \centering
  \begin{subfigure}[H]{0.45\textwidth}
    \centering
    \includestandalone{figures/fig_003}
    \caption{$A^{\mathbb{C}}$}
    \label{fig:setComplement}
  \end{subfigure}
  \begin{subfigure}[H]{0.45\textwidth}
    \centering
    \includestandalone{figures/fig_004}
    \caption{$A \textbf{\textbackslash} B$}
    \label{fig:setDifference}
  \end{subfigure}
  \caption{Set Operations}
  \label{fig:setOperations}
\end{figure}

\subsection{Subsets and Supersets}
\label{ssec:subsetsAndSupersets}

\begin{definition}{Subset}
  
\end{definition}

\begin{definition}{Superset}
  
\end{definition}

\subsection{Powersets and Cartesian Products}
\label{ssec:powersetsAndCartesianProducts}

\begin{definition}{Powerset}
  
\end{definition}

\begin{definition}{Cartesian Product}
  
\end{definition}

\subsection{Set Identities}
\label{ssec:setIdentities}

\subsection{Set Notation}
\label{ssec:setNotation}

\begin{figure}[H]
  \centering
  \begin{subfigure}[H]{0.45\textwidth}
    \centering
    \begin{tblr}{cc}
      \toprule
      Notation & Meaning \\
      \midrule
      $\varnothing$ & Empty Set \\
      $\#$ & Size of a Set \\
      $\in$ & Is a member of \\
      $\notin$ & Is not a member of \\
      $\subset$ & Is a subset of \\
      $\supset$ & Is a superset of \\
      $\supset$ & Is a superset of \\
      $\cup$ & Set union \\
      $\cap$ & Set intersection \\
      \textbf{\textbackslash} & Set difference \\
      $\complement$ & Set complement \\
      \bottomrule
    \end{tblr}
    \caption{Set Theory}
    \label{fig:setTheory}
  \end{subfigure}
  \begin{subfigure}[H]{0.45\textwidth}
    \centering
    \begin{tblr}{cc}
      \toprule
      Notation & Meaning \\
      \midrule
      $\neg$ & Logical Negation \\
      $\land$ & Logical And \\
      $\lor$ & Logical Or \\
      $\oplus$ & Logical Xor \\
      $\forall$ & Universal Quantification \\
      $\exists$ & Existential Quantification \\
      $\exists !$ & Uniqueness Quantification \\
      $\Rightarrow$ & Material Conditional \\
      $\Leftrightarrow$ & Logical Equivalence \\
      $\bot$ & False Predicate \\
      $\top$ & True  Predicate \\
      \bottomrule
    \end{tblr}
    \caption{Formal Logic}
    \label{fig:formalLogic}
  \end{subfigure}

  \vspace{12pt}
  \begin{subfigure}[H]{0.45\textwidth}
    \centering
    \begin{tblr}{cc}
      \toprule
      Notation & Meaning \\
      \midrule
      $\mathbb{N}$ & Natural Numbers \\
      $\mathbb{Z}$ & Integers \\
      $\mathbb{Q}$ & Rational Numbers \\
      $\mathbb{R}$ & Real Numbers \\
      $\mathbb{C}$ & Complex Numbers \\
      \bottomrule
    \end{tblr}
    \caption{Number Systems}
    \label{fig:numberSystems}
  \end{subfigure}
  \caption{Logic and Set Theory Notation}
  \label{fig:logicAndSetTheoryNotation}
\end{figure}

\end{document}
