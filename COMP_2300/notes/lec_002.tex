\documentclass[12pt]{article}

%%%% GRAPHICS %%%%
\usepackage{tikz}
\usepackage{circuitikz}
\usetikzlibrary{arrows.meta}
\usepackage{tikz-3dplot}
\usepackage{graphicx}
\usepackage{pgfplots}
  \pgfplotsset{compat=1.18}
\usetikzlibrary{arrows}
\newcommand{\midarrow}{\tikz \draw[-triangle 90] (0,0) -- +(.1,0);}

%%%% FIGURES %%%%
\usepackage{subcaption}
\usepackage{wrapfig}
\usepackage{float}
\usepackage[skip=5pt, font=footnotesize]{caption}

%%%% FORMATTING %%%%
\usepackage{parskip}
\usepackage{tcolorbox}
\usepackage{ulem}
% \usepackage{fancyhdr}

%%%% TABLE FORMATTING %%%%
\usepackage{tabularray}
\UseTblrLibrary{booktabs}

%%%% MATH AND LOGIC %%%%
\usepackage{xifthen}
\usepackage{amsmath}
\usepackage{amssymb}
\usepackage{amsfonts}

%%%% TEXT AND SYMBOLS %%%%
\usepackage[T1]{fontenc}
\usepackage{textcomp}
\usepackage{gensymb}

%%%% OTHER %%%%
\usepackage{standalone}

%%%% LOGIC SYMBOLS %%%%
\newcommand*\xor{\oplus}

%%%% STYLES %%%%

% Packages
\usepackage{fullpage}
\usepackage{titlesec}
\usepackage[rgb]{xcolor}
\selectcolormodel{natural}
\usepackage{ninecolors}
\selectcolormodel{rgb}

% Colors
\definecolor{pg}{HTML}{24273A}
\definecolor{fg}{HTML}{FFFFFF}
\definecolor{bg}{HTML}{24273A}
\definecolor{re}{HTML}{d20f39}
\definecolor{gr}{HTML}{40a02b}
\definecolor{ye}{HTML}{df8e1d}
\definecolor{or}{HTML}{fe640b}
\definecolor{bl}{HTML}{1e66f5}
\definecolor{ma}{HTML}{8839ef}
\definecolor{cy}{HTML}{179299}
\definecolor{pi}{HTML}{ea76cb}

\usepackage{nameref}
\makeatletter
\newcommand*{\currentname}{\@currentlabelname}
\makeatother

\titleformat{\section}
  {\normalfont\scshape\Large\bfseries}
  {\thesection}
  {0.75em}
  {}

\titleformat{\subsection}
  {\normalfont\scshape\large\bfseries}
  {\thesubsection}
  {0.75em}
  {}

\titleformat{\subsubsection}
  {\normalfont\scshape\normalsize\bfseries}
  {\thesubsubsection}
  {0.75em}
  {}

% Formula
\newcounter{formula}[section]
\newenvironment{formula}[1]{
  \stepcounter{formula}
  \begin{tcolorbox}[
    standard jigsaw, % Allows opacity
    colframe={fg},
    boxrule=1px,
    colback=bg,
    opacityback=0,
    sharp corners,
    sidebyside,
    righthand width=18px,
    coltext={fg}
  ]
  \centering
  \textbf{\uline{#1}}
}{
  \tcblower
  \textbf{\thesection.\theformula}
  \end{tcolorbox}
}

% Definition
\newcounter{definition}[section]

\newenvironment{definition*}[1]{
  \begin{tcolorbox}[
    standard jigsaw, % Allows opacity
    colframe={fg},
    boxrule=1px,
    colback=bg,
    opacityback=0,
    sharp corners,
    coltext={fg}
  ]
  \textbf{#1 \hfill}
  \vspace{5px}
  \hrule
  \vspace{5px}
  \noindent
}{
  \end{tcolorbox}
}

\newenvironment{definition}[1]{
  \stepcounter{definition}
  \begin{tcolorbox}[
    standard jigsaw, % Allows opacity
    colframe={fg},
    boxrule=1px,
    colback=bg,
    opacityback=0,
    sharp corners,
    coltext={fg}
  ]
  \textbf{#1 \hfill \thesection.\thedefinition}
  \vspace{5px}
  \hrule
  \vspace{5px}
  \noindent
}{
  \end{tcolorbox}
}

% Example Problem
\newcounter{example}[section]
\newenvironment{example}{
  \stepcounter{example}
  \begin{tcolorbox}[
    standard jigsaw, % Allows opacity
    colframe={fg},
    boxrule=1px,
    colback=bg,
    opacityback=0,
    sharp corners,
    coltext={fg}
  ]
  \textbf{Example \hfill \thesection.\theexample}
  \vspace{5px}
  \hrule
  \vspace{5px}
  \noindent
}{
  \end{tcolorbox}
}

\tikzset{
  cubeBorder/.style=fg,
  cubeFilling/.style={fg!20!bg, opacity=0.25},
  gridLine/.style={very thin, gray},
  graphLine/.style={-latex, thick, fg},
}

\pgfplotsset{
  basicAxis/.style={
    grid,
    major grid style={line width=.2pt,draw=fg!50!bg},
    axis lines = box,
    axis line style = {line width = 1px},
  }
}

%%%% REFERENCES %%%%
\usepackage{hyperref}
\hypersetup{
  colorlinks  = true,
  linkcolor   = pi,
  anchorcolor = pi,
  citecolor   = pi,
  filecolor   = pi,
  menucolor   = pi,
  runcolor    = pi,
  urlcolor    = pi,
}

\author{Ethan Anthony}


\title{Lecture 002}
\date{September 19, 2024}

\begin{document}

\newpage
\section{Sets}
\label{sec:sets}

\subsection{Introduction to Sets}
\label{ssec:introductionToSets}

\begin{definition}{Set}
  A collection of objects defined either by listing its members or by defining a pattern.
\end{definition}

The ways to specifically define certain elements as part of a set is as follows:
\begin{center}
  \begin{tblr}{cc}
    \toprule
    Notation & Meaning \\
    \midrule
    $5 \in A$ & $5$ is a member of set $A$ \\
    $3 \notin B$ & $3$ is not a member of set $B$ \\
    $C = \{1, 2, 3\}$ & Set $C$ contains $1$, $2$, and $3$ \\
    \bottomrule
  \end{tblr}
\end{center}

Additionally, set builder notation allows you to construct a set based on a defined pattern:
\begin{figure}[H]
  \vspace{-20pt}
  \begin{align*}
    A &= \{2k+1|k \in \mathbb{Z}\} \\
    A &= \{n\ |\ \exists k \in \mathbb{Z} \textup{ for which } n = 2k + 1\}
  \end{align*}
  \caption{Defines Set $A$ as all odd numbers}
  \label{fig:setA}
\end{figure}
Sets can also be defined to be empty. $A = \{\}$, $A = \varnothing$, and $A = \{\varnothing\}$
all define the set $A$ to be empty. Additionally, this can be done through set builder notation:
\begin{equation*}
  A = \{x\ |\ \forall x, x \ \exists !\  U\}
\end{equation*}

\hrule
\vspace{12pt}

\subsection{Set Operations}
\label{ssec:setOperations}

Once a set exists, operations can be performed on the set at large. Four basic operations
are the union of two sets ($\cup$), the intersection of two sets ($\cap$), the complement
of a set ($\complement$), and the difference of two sets (\textbf{\textbackslash}).

\begin{figure}[H]
  \centering
  \begin{subfigure}[H]{0.45\textwidth}
    \centering
    \includestandalone{figures/fig_001}
    \caption{$A \cup B$}
    \label{fig:setUnion}
  \end{subfigure}
  \begin{subfigure}[H]{0.45\textwidth}
    \centering
    \includestandalone{figures/fig_002}
    \caption{$A \cap B$}
    \label{fig:setIntersection}
  \end{subfigure}
\end{figure}
\begin{figure}[H]
  \centering
  \begin{subfigure}[H]{0.45\textwidth}
    \centering
    \includestandalone{figures/fig_003}
    \caption{$A^{\mathbb{C}}$}
    \label{fig:setComplement}
  \end{subfigure}
  \begin{subfigure}[H]{0.45\textwidth}
    \centering
    \includestandalone{figures/fig_004}
    \caption{$A \textbf{\textbackslash} B$}
    \label{fig:setDifference}
  \end{subfigure}
  \caption{Set Operations}
  \label{fig:setOperations}
\end{figure}

\subsection{Subsets and Supersets}
\label{ssec:subsetsAndSupersets}

\begin{definition}{Subset}
  
\end{definition}

\begin{definition}{Superset}
  
\end{definition}

\subsection{Powersets and Cartesian Products}
\label{ssec:powersetsAndCartesianProducts}

\begin{definition}{Powerset}
  
\end{definition}

\begin{definition}{Cartesian Product}
  
\end{definition}

\subsection{Set Identities}
\label{ssec:setIdentities}

\subsection{Set Notation}
\label{ssec:setNotation}

\begin{figure}[H]
  \centering
  \begin{subfigure}[H]{0.45\textwidth}
    \centering
    \begin{tblr}{cc}
      \toprule
      Notation & Meaning \\
      \midrule
      $\varnothing$ & Empty Set \\
      $\#$ & Size of a Set \\
      $\in$ & Is a member of \\
      $\notin$ & Is not a member of \\
      $\subset$ & Is a subset of \\
      $\supset$ & Is a superset of \\
      $\supset$ & Is a superset of \\
      $\cup$ & Set union \\
      $\cap$ & Set intersection \\
      \textbf{\textbackslash} & Set difference \\
      $\complement$ & Set complement \\
      \bottomrule
    \end{tblr}
    \caption{Set Theory}
    \label{fig:setTheory}
  \end{subfigure}
  \begin{subfigure}[H]{0.45\textwidth}
    \centering
    \begin{tblr}{cc}
      \toprule
      Notation & Meaning \\
      \midrule
      $\neg$ & Logical Negation \\
      $\land$ & Logical And \\
      $\lor$ & Logical Or \\
      $\oplus$ & Logical Xor \\
      $\forall$ & Universal Quantification \\
      $\exists$ & Existential Quantification \\
      $\exists !$ & Uniqueness Quantification \\
      $\Rightarrow$ & Material Conditional \\
      $\Leftrightarrow$ & Logical Equivalence \\
      $\bot$ & False Predicate \\
      $\top$ & True  Predicate \\
      \bottomrule
    \end{tblr}
    \caption{Formal Logic}
    \label{fig:formalLogic}
  \end{subfigure}

  \vspace{12pt}
  \begin{subfigure}[H]{0.45\textwidth}
    \centering
    \begin{tblr}{cc}
      \toprule
      Notation & Meaning \\
      \midrule
      $\mathbb{N}$ & Natural Numbers \\
      $\mathbb{Z}$ & Integers \\
      $\mathbb{Q}$ & Rational Numbers \\
      $\mathbb{R}$ & Real Numbers \\
      $\mathbb{C}$ & Complex Numbers \\
      \bottomrule
    \end{tblr}
    \caption{Number Systems}
    \label{fig:numberSystems}
  \end{subfigure}
  \caption{Logic and Set Theory Notation}
  \label{fig:logicAndSetTheoryNotation}
\end{figure}

\end{document}
