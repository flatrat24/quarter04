\documentclass[12pt]{article}

\input{../../xlatex/imports/preamble}

\title{Lecture 005}
\date{October 08, 2024}

\begin{document}
\newpage
\section{Number Theory}
\label{sec:numberTheory}

\subsection{Last-Digit Algebra}
\label{ssec:lastDigitAlgebra}

\begin{wrapfigure}[19]{l}{0.35\textwidth}
  \centering
  \begin{tblr}{cc|c}
    \toprule
    $i$ & $n \cdot i$ & $n \% 10$ \\
    \midrule
    0  & 0  & 0 \\
    1  & 7  & 7 \\
    2  & 14 & 4 \\
    3  & 21 & 1 \\
    4  & 28 & 8 \\
    5  & 35 & 5 \\
    6  & 42 & 2 \\
    7  & 49 & 9 \\
    8  & 56 & 6 \\
    9  & 63 & 3 \\
    10 & 70 & 0 \\
    11 & 77 & 7 \\
    12 & 84 & 4 \\
    13 & 91 & 1 \\
    \bottomrule
  \end{tblr}
\end{wrapfigure}

Adding $7$ to itself infinitely will result in an infinitely increasing sequence of integers.
By focusing only on the last digit (the one's digit) of each item in the sequence, a pattern
emerges: $0,7,4,1,8,5,2,9,6,3,0,7,4,1,...$.

This sequence is showing that when $7$ is added to a number ending in a $0$, the sum will
end in a $7$; when $7$ is added to a number ending in a $7$, the sum will end in a $4$; etc.
This idea can be stated as:
\begin{align*}
  (7 \cdot 1) \% 10 &= 7 \\
  (7 \cdot 2) \% 10 &= 4
\end{align*}
Any number ending in a $7$ can replace the seven in these equations. Thus:
\begin{align*}
  \forall x\ | \ x \% 10=7 \Rightarrow (x \cdot 1) \% 10 &= 7 \\
  \forall x\ | \ x \% 10=7 \Rightarrow (x \cdot 2) \% 10 &= 4
\end{align*}
By considering only the last digit of a number, rules can be derived about the result of
some arbitrary algebraic operation.

\subsection{Divisibility and Modular Arithmetic}
\label{ssec:divisibility}

\begin{definition}{Divides}
  $a$ is said to \textit{divide} $b$ if $\frac{b}{a} \in \mathbb{Z}$. In other words,
  $a$ divides $b$ if there exists some integer $c$ such that $b = c \cdot a$. Denoted as
  $a | b$.
\end{definition}
There are three \textbf{basic properties of divisibility}. Considering all \textit{integer} values of $a$,
$b$, and $c$ where $a \neq 0$, then:
\begin{align*}
  a|b \land a|c &\Rightarrow a|(b+c) \\
  a|b           &\Rightarrow a|bc \\
  a|b \land b|c &\Rightarrow a|c
\end{align*}
\end{document}
