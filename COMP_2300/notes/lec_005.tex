\documentclass[12pt]{article}

%%%% GRAPHICS %%%%
\usepackage{tikz}
\usepackage{circuitikz}
\usetikzlibrary{arrows.meta}
\usepackage{tikz-3dplot}
\usepackage{graphicx}
\usepackage{pgfplots}
  \pgfplotsset{compat=1.18}
\usetikzlibrary{arrows}
\newcommand{\midarrow}{\tikz \draw[-triangle 90] (0,0) -- +(.1,0);}

%%%% FIGURES %%%%
\usepackage{subcaption}
\usepackage{wrapfig}
\usepackage{float}
\usepackage[skip=5pt, font=footnotesize]{caption}

%%%% FORMATTING %%%%
\usepackage{parskip}
\usepackage{tcolorbox}
\usepackage{ulem}
% \usepackage{fancyhdr}

%%%% TABLE FORMATTING %%%%
\usepackage{tabularray}
\UseTblrLibrary{booktabs}

%%%% MATH AND LOGIC %%%%
\usepackage{xifthen}
\usepackage{amsmath}
\usepackage{amssymb}
\usepackage{amsfonts}

%%%% TEXT AND SYMBOLS %%%%
\usepackage[T1]{fontenc}
\usepackage{textcomp}
\usepackage{gensymb}

%%%% OTHER %%%%
\usepackage{standalone}

%%%% LOGIC SYMBOLS %%%%
\newcommand*\xor{\oplus}

%%%% STYLES %%%%

% Packages
\usepackage{fullpage}
\usepackage{titlesec}
\usepackage[rgb]{xcolor}
\selectcolormodel{natural}
\usepackage{ninecolors}
\selectcolormodel{rgb}

% Colors
\definecolor{pg}{HTML}{24273A}
\definecolor{fg}{HTML}{FFFFFF}
\definecolor{bg}{HTML}{24273A}
\definecolor{re}{HTML}{d20f39}
\definecolor{gr}{HTML}{40a02b}
\definecolor{ye}{HTML}{df8e1d}
\definecolor{or}{HTML}{fe640b}
\definecolor{bl}{HTML}{1e66f5}
\definecolor{ma}{HTML}{8839ef}
\definecolor{cy}{HTML}{179299}
\definecolor{pi}{HTML}{ea76cb}

\usepackage{nameref}
\makeatletter
\newcommand*{\currentname}{\@currentlabelname}
\makeatother

\titleformat{\section}
  {\normalfont\scshape\Large\bfseries}
  {\thesection}
  {0.75em}
  {}

\titleformat{\subsection}
  {\normalfont\scshape\large\bfseries}
  {\thesubsection}
  {0.75em}
  {}

\titleformat{\subsubsection}
  {\normalfont\scshape\normalsize\bfseries}
  {\thesubsubsection}
  {0.75em}
  {}

% Formula
\newcounter{formula}[section]
\newenvironment{formula}[1]{
  \stepcounter{formula}
  \begin{tcolorbox}[
    standard jigsaw, % Allows opacity
    colframe={fg},
    boxrule=1px,
    colback=bg,
    opacityback=0,
    sharp corners,
    sidebyside,
    righthand width=18px,
    coltext={fg}
  ]
  \centering
  \textbf{\uline{#1}}
}{
  \tcblower
  \textbf{\thesection.\theformula}
  \end{tcolorbox}
}

% Definition
\newcounter{definition}[section]

\newenvironment{definition*}[1]{
  \begin{tcolorbox}[
    standard jigsaw, % Allows opacity
    colframe={fg},
    boxrule=1px,
    colback=bg,
    opacityback=0,
    sharp corners,
    coltext={fg}
  ]
  \textbf{#1 \hfill}
  \vspace{5px}
  \hrule
  \vspace{5px}
  \noindent
}{
  \end{tcolorbox}
}

\newenvironment{definition}[1]{
  \stepcounter{definition}
  \begin{tcolorbox}[
    standard jigsaw, % Allows opacity
    colframe={fg},
    boxrule=1px,
    colback=bg,
    opacityback=0,
    sharp corners,
    coltext={fg}
  ]
  \textbf{#1 \hfill \thesection.\thedefinition}
  \vspace{5px}
  \hrule
  \vspace{5px}
  \noindent
}{
  \end{tcolorbox}
}

% Example Problem
\newcounter{example}[section]
\newenvironment{example}{
  \stepcounter{example}
  \begin{tcolorbox}[
    standard jigsaw, % Allows opacity
    colframe={fg},
    boxrule=1px,
    colback=bg,
    opacityback=0,
    sharp corners,
    coltext={fg}
  ]
  \textbf{Example \hfill \thesection.\theexample}
  \vspace{5px}
  \hrule
  \vspace{5px}
  \noindent
}{
  \end{tcolorbox}
}

\tikzset{
  cubeBorder/.style=fg,
  cubeFilling/.style={fg!20!bg, opacity=0.25},
  gridLine/.style={very thin, gray},
  graphLine/.style={-latex, thick, fg},
}

\pgfplotsset{
  basicAxis/.style={
    grid,
    major grid style={line width=.2pt,draw=fg!50!bg},
    axis lines = box,
    axis line style = {line width = 1px},
  }
}

%%%% REFERENCES %%%%
\usepackage{hyperref}
\hypersetup{
  colorlinks  = true,
  linkcolor   = pi,
  anchorcolor = pi,
  citecolor   = pi,
  filecolor   = pi,
  menucolor   = pi,
  runcolor    = pi,
  urlcolor    = pi,
}

\author{Ethan Anthony}


\title{Lecture 005}
\date{October 08, 2024}

\begin{document}
\newpage
\section{Number Theory}
\label{sec:numberTheory}

\subsection{Last-Digit Algebra}
\label{ssec:lastDigitAlgebra}

\begin{wrapfigure}[19]{l}{0.35\textwidth}
  \centering
  \begin{tblr}{cc|c}
    \toprule
    $i$ & $n \cdot i$ & $n \% 10$ \\
    \midrule
    0  & 0  & 0 \\
    1  & 7  & 7 \\
    2  & 14 & 4 \\
    3  & 21 & 1 \\
    4  & 28 & 8 \\
    5  & 35 & 5 \\
    6  & 42 & 2 \\
    7  & 49 & 9 \\
    8  & 56 & 6 \\
    9  & 63 & 3 \\
    10 & 70 & 0 \\
    11 & 77 & 7 \\
    12 & 84 & 4 \\
    13 & 91 & 1 \\
    \bottomrule
  \end{tblr}
\end{wrapfigure}

Adding $7$ to itself infinitely will result in an infinitely increasing sequence of integers.
By focusing only on the last digit (the one's digit) of each item in the sequence, a pattern
emerges: $0,7,4,1,8,5,2,9,6,3,0,7,4,1,...$.

This sequence is showing that when $7$ is added to a number ending in a $0$, the sum will
end in a $7$; when $7$ is added to a number ending in a $7$, the sum will end in a $4$; etc.
This idea can be stated as:
\begin{align*}
  (7 \cdot 1) \% 10 &= 7 \\
  (7 \cdot 2) \% 10 &= 4
\end{align*}
Any number ending in a $7$ can replace the seven in these equations. Thus:
\begin{align*}
  \forall x\ | \ x \% 10=7 \Rightarrow (x \cdot 1) \% 10 &= 7 \\
  \forall x\ | \ x \% 10=7 \Rightarrow (x \cdot 2) \% 10 &= 4
\end{align*}
By considering only the last digit of a number, rules can be derived about the result of
some arbitrary algebraic operation.

\subsection{Divisibility and Modular Arithmetic}
\label{ssec:divisibility}

\begin{definition}{Divides}
  $a$ is said to \textit{divide} $b$ if $\frac{b}{a} \in \mathbb{Z}$. In other words,
  $a$ divides $b$ if there exists some integer $c$ such that $b = c \cdot a$. Denoted as
  $a | b$.
\end{definition}
There are three \textbf{basic properties of divisibility}. Considering all \textit{integer} values of $a$,
$b$, and $c$ where $a \neq 0$, then:
\begin{align*}
  a|b \land a|c &\Rightarrow a|(b+c) \\
  a|b           &\Rightarrow a|bc \\
  a|b \land b|c &\Rightarrow a|c
\end{align*}
\end{document}
