\documentclass[12pt]{article}

%%%% GRAPHICS %%%%
\usepackage{tikz}
\usepackage{circuitikz}
\usetikzlibrary{arrows.meta}
\usepackage{tikz-3dplot}
\usepackage{graphicx}
\usepackage{pgfplots}
  \pgfplotsset{compat=1.18}
\usetikzlibrary{arrows}
\newcommand{\midarrow}{\tikz \draw[-triangle 90] (0,0) -- +(.1,0);}

%%%% FIGURES %%%%
\usepackage{subcaption}
\usepackage{wrapfig}
\usepackage{float}
\usepackage[skip=5pt, font=footnotesize]{caption}

%%%% FORMATTING %%%%
\usepackage{parskip}
\usepackage{tcolorbox}
\usepackage{ulem}
% \usepackage{fancyhdr}

%%%% TABLE FORMATTING %%%%
\usepackage{tabularray}
\UseTblrLibrary{booktabs}

%%%% MATH AND LOGIC %%%%
\usepackage{xifthen}
\usepackage{amsmath}
\usepackage{amssymb}
\usepackage{amsfonts}

%%%% TEXT AND SYMBOLS %%%%
\usepackage[T1]{fontenc}
\usepackage{textcomp}
\usepackage{gensymb}

%%%% OTHER %%%%
\usepackage{standalone}

%%%% LOGIC SYMBOLS %%%%
\newcommand*\xor{\oplus}

%%%% STYLES %%%%

% Packages
\usepackage{fullpage}
\usepackage{titlesec}
\usepackage[rgb]{xcolor}
\selectcolormodel{natural}
\usepackage{ninecolors}
\selectcolormodel{rgb}

% Colors
\definecolor{pg}{HTML}{24273A}
\definecolor{fg}{HTML}{FFFFFF}
\definecolor{bg}{HTML}{24273A}
\definecolor{re}{HTML}{d20f39}
\definecolor{gr}{HTML}{40a02b}
\definecolor{ye}{HTML}{df8e1d}
\definecolor{or}{HTML}{fe640b}
\definecolor{bl}{HTML}{1e66f5}
\definecolor{ma}{HTML}{8839ef}
\definecolor{cy}{HTML}{179299}
\definecolor{pi}{HTML}{ea76cb}

\usepackage{nameref}
\makeatletter
\newcommand*{\currentname}{\@currentlabelname}
\makeatother

\titleformat{\section}
  {\normalfont\scshape\Large\bfseries}
  {\thesection}
  {0.75em}
  {}

\titleformat{\subsection}
  {\normalfont\scshape\large\bfseries}
  {\thesubsection}
  {0.75em}
  {}

\titleformat{\subsubsection}
  {\normalfont\scshape\normalsize\bfseries}
  {\thesubsubsection}
  {0.75em}
  {}

% Formula
\newcounter{formula}[section]
\newenvironment{formula}[1]{
  \stepcounter{formula}
  \begin{tcolorbox}[
    standard jigsaw, % Allows opacity
    colframe={fg},
    boxrule=1px,
    colback=bg,
    opacityback=0,
    sharp corners,
    sidebyside,
    righthand width=18px,
    coltext={fg}
  ]
  \centering
  \textbf{\uline{#1}}
}{
  \tcblower
  \textbf{\thesection.\theformula}
  \end{tcolorbox}
}

% Definition
\newcounter{definition}[section]

\newenvironment{definition*}[1]{
  \begin{tcolorbox}[
    standard jigsaw, % Allows opacity
    colframe={fg},
    boxrule=1px,
    colback=bg,
    opacityback=0,
    sharp corners,
    coltext={fg}
  ]
  \textbf{#1 \hfill}
  \vspace{5px}
  \hrule
  \vspace{5px}
  \noindent
}{
  \end{tcolorbox}
}

\newenvironment{definition}[1]{
  \stepcounter{definition}
  \begin{tcolorbox}[
    standard jigsaw, % Allows opacity
    colframe={fg},
    boxrule=1px,
    colback=bg,
    opacityback=0,
    sharp corners,
    coltext={fg}
  ]
  \textbf{#1 \hfill \thesection.\thedefinition}
  \vspace{5px}
  \hrule
  \vspace{5px}
  \noindent
}{
  \end{tcolorbox}
}

% Example Problem
\newcounter{example}[section]
\newenvironment{example}{
  \stepcounter{example}
  \begin{tcolorbox}[
    standard jigsaw, % Allows opacity
    colframe={fg},
    boxrule=1px,
    colback=bg,
    opacityback=0,
    sharp corners,
    coltext={fg}
  ]
  \textbf{Example \hfill \thesection.\theexample}
  \vspace{5px}
  \hrule
  \vspace{5px}
  \noindent
}{
  \end{tcolorbox}
}

\tikzset{
  cubeBorder/.style=fg,
  cubeFilling/.style={fg!20!bg, opacity=0.25},
  gridLine/.style={very thin, gray},
  graphLine/.style={-latex, thick, fg},
}

\pgfplotsset{
  basicAxis/.style={
    grid,
    major grid style={line width=.2pt,draw=fg!50!bg},
    axis lines = box,
    axis line style = {line width = 1px},
  }
}

%%%% REFERENCES %%%%
\usepackage{hyperref}
\hypersetup{
  colorlinks  = true,
  linkcolor   = pi,
  anchorcolor = pi,
  citecolor   = pi,
  filecolor   = pi,
  menucolor   = pi,
  runcolor    = pi,
  urlcolor    = pi,
}

\author{Ethan Anthony}


\title{Lecture 003}
\date{September 26, 2024}

\begin{document}

\newpage
\section{Functions}
\label{sec:functions}

\subsection{Function Definition}
\label{ssec:functionDefinition}

A function is defined as $f: A \rightarrow B$, where each element in array $A$ is assigned
to exactly one element in array $B$. $f(a) = b$ denotes a single element from set $B$ as the
assignment from an element from set $A$.

A function $f: A \rightarrow B$ can also be defined as a subset of $A \times B$. This is because
a function $f: A \rightarrow B$ contains one, and only one, ordered pair for each element
in $A$.
\begin{equation*}
  \forall x [ x \in A \rightarrow \exists y [ y \in B \land (x,y) \in f]]
\end{equation*}

\begin{figure}[H]
  \centering
  \begin{subfigure}[H]{0.24\textwidth}
    \centering
    \includestandalone{figures/fig_005}
    \caption{Not a Function}
    \label{fig:005}
  \end{subfigure}
  \begin{subfigure}[H]{0.24\textwidth}
    \centering
    \includestandalone{figures/fig_006}
    \caption{Function}
    \label{fig:006}
  \end{subfigure}
  \begin{subfigure}[H]{0.24\textwidth}
    \centering
    \includestandalone{figures/fig_007}
    \caption{Function}
    \label{fig:007}
  \end{subfigure}
  \begin{subfigure}[H]{0.24\textwidth}
    \centering
    \includestandalone{figures/fig_008}
    \caption{Not a Function}
    \label{fig:008}
  \end{subfigure}
\end{figure}

In the working definition of a function as $f: A \rightarrow B$, the \textbf{domain} of the
function is $A$, and the \textbf{codomain} is $B$. In terms of $f(a) = b$, $a$ is the \textbf{preimage}
and $b$ is the \textbf{image}.

\begin{definition}{Range of a Function}
  The range of a function is the set of all images of the function.
  \begin{equation*}
    \textup{Range of a Set} = \{x|\forall a \in A, f(a) = x \land x \in B\}
  \end{equation*}
\end{definition}

\subsection{Properties of Functions}
\label{ssec:propertiesOfFunctions}

\begin{definition}{Injection (One to one)}
  A function is said to be \textit{injective} if and only if $f(a) = f(b) \rightarrow a = b$.
  An injective function has only one input for each element in its range.
\end{definition}

\begin{definition}{Surjection (Onto)}
  A function is said to be \textit{surjective} if and only if for every element $b$ in $B$, there
  is an element $a$ in $A$ such that $f(a) = b$. In other words, the range of the function
  must be equivalent to the codomain.
\end{definition}

\begin{definition}{Bijection}
  A function is said to be bijective if and only if it is both \textit{injective} and \textit{surjective}.
\end{definition}

\begin{figure}[H]
  \centering
  \begin{subfigure}[H]{0.24\textwidth}
    \centering
    \includestandalone{figures/fig_009}
    \caption{General Function}
    \label{fig:009}
  \end{subfigure}
  \begin{subfigure}[H]{0.24\textwidth}
    \centering
    \includestandalone{figures/fig_010}
    \caption{Injective}
    \label{fig:010}
  \end{subfigure}
  \begin{subfigure}[H]{0.24\textwidth}
    \centering
    \includestandalone{figures/fig_011}
    \caption{Surjective}
    \label{fig:011}
  \end{subfigure}
  \begin{subfigure}[H]{0.24\textwidth}
    \centering
    \includestandalone{figures/fig_012}
    \caption{Bijective}
    \label{fig:012}
  \end{subfigure}
  \caption{Examples of Functions}
  \label{fig:examplesOfFunctions}
\end{figure}

Functions can also be inversed. If there is a function $f$, the inverse is denoted as $f^{-1}$.
A function can only have an inverse if it is a \textbf{bijective} function. This is because,
if the function were to be injective, but not surjective, there would be elements in the domain of $f^{-1}$
that don't map onto the codomain of $f^{-1}$. Similarly, if the function were to be surjective,
but not injective, elements in the domain of $f^{-1}$ would map onto multiple values.

\end{document}
