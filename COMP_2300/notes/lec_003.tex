\documentclass[12pt]{article}

\input{../../xlatex/imports/preamble}

\title{Lecture 003}
\date{September 26, 2024}

\begin{document}

\newpage
\section{Functions}
\label{sec:functions}

\subsection{Function Definition}
\label{ssec:functionDefinition}

A function is defined as $f: A \rightarrow B$, where each element in array $A$ is assigned
to exactly one element in array $B$. $f(a) = b$ denotes a single element from set $B$ as the
assignment from an element from set $A$.

A function $f: A \rightarrow B$ can also be defined as a subset of $A \times B$. This is because
a function $f: A \rightarrow B$ contains one, and only one, ordered pair for each element
in $A$.
\begin{equation*}
  \forall x [ x \in A \rightarrow \exists y [ y \in B \land (x,y) \in f]]
\end{equation*}

\begin{figure}[H]
  \centering
  \begin{subfigure}[H]{0.24\textwidth}
    \centering
    \includestandalone{figures/fig_005}
    \caption{Not a Function}
    \label{fig:005}
  \end{subfigure}
  \begin{subfigure}[H]{0.24\textwidth}
    \centering
    \includestandalone{figures/fig_006}
    \caption{Function}
    \label{fig:006}
  \end{subfigure}
  \begin{subfigure}[H]{0.24\textwidth}
    \centering
    \includestandalone{figures/fig_007}
    \caption{Function}
    \label{fig:007}
  \end{subfigure}
  \begin{subfigure}[H]{0.24\textwidth}
    \centering
    \includestandalone{figures/fig_008}
    \caption{Not a Function}
    \label{fig:008}
  \end{subfigure}
\end{figure}

In the working definition of a function as $f: A \rightarrow B$, the \textbf{domain} of the
function is $A$, and the \textbf{codomain} is $B$. In terms of $f(a) = b$, $a$ is the \textbf{preimage}
and $b$ is the \textbf{image}.

\begin{definition}{Range of a Function}
  The range of a function is the set of all images of the function.
  \begin{equation*}
    \textup{Range of a Set} = \{x|\forall a \in A, f(a) = x \land x \in B\}
  \end{equation*}
\end{definition}

\subsection{Properties of Functions}
\label{ssec:propertiesOfFunctions}

\begin{definition}{Injection (One to one)}
  A function is said to be \textit{injective} if and only if $f(a) = f(b) \rightarrow a = b$.
  An injective function has only one input for each element in its range.
\end{definition}

\begin{definition}{Surjection (Onto)}
  A function is said to be \textit{surjective} if and only if for every element $b$ in $B$, there
  is an element $a$ in $A$ such that $f(a) = b$. In other words, the range of the function
  must be equivalent to the codomain.
\end{definition}

\begin{definition}{Bijection}
  A function is said to be bijective if and only if it is both \textit{injective} and \textit{surjective}.
\end{definition}

\begin{figure}[H]
  \centering
  \begin{subfigure}[H]{0.24\textwidth}
    \centering
    \includestandalone{figures/fig_009}
    \caption{General Function}
    \label{fig:009}
  \end{subfigure}
  \begin{subfigure}[H]{0.24\textwidth}
    \centering
    \includestandalone{figures/fig_010}
    \caption{Injective}
    \label{fig:010}
  \end{subfigure}
  \begin{subfigure}[H]{0.24\textwidth}
    \centering
    \includestandalone{figures/fig_011}
    \caption{Surjective}
    \label{fig:011}
  \end{subfigure}
  \begin{subfigure}[H]{0.24\textwidth}
    \centering
    \includestandalone{figures/fig_012}
    \caption{Bijective}
    \label{fig:012}
  \end{subfigure}
  \caption{Examples of Functions}
  \label{fig:examplesOfFunctions}
\end{figure}

Functions can also be inversed. If there is a function $f$, the inverse is denoted as $f^{-1}$.
A function can only have an inverse if it is a \textbf{bijective} function. This is because,
if the function were to be injective, but not surjective, there would be elements in the domain of $f^{-1}$
that don't map onto the codomain of $f^{-1}$. Similarly, if the function were to be surjective,
but not injective, elements in the domain of $f^{-1}$ would map onto multiple values.

\end{document}
