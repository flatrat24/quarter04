\documentclass[12pt]{article}

\input{../../xlatex/imports/preamble}

\title{Lecture 004}
\date{September 13, 2024}

\begin{document}

\section{2D Moments}
\label{sec:2DMoments}

\subsection{Defining Moments}
\label{ssec:definingMoments}

\begin{definition}{Moment}
  A \textbf{vector quantity} used to describe the "turning effect" of a force about a point
  perpendicular to the force. Calculated using the vector product:
  \begin{equation*}
    \overrightarrow{M}_O = \overrightarrow{r} \times \overrightarrow{F}
  \end{equation*}
\end{definition}

To calculate the moment of a force about a point, we need to calculate vector products. The
cross product of two vectors ($\overrightarrow{P} \times \overrightarrow{Q}$) results in a third
vector ($\overrightarrow{V}$):

\begin{formula}{Vector Product}
  \begin{equation*}
    \overrightarrow{V} = \overrightarrow{P} \times \overrightarrow{Q}
  \end{equation*}
  This results in a new vector with
  \begin{itemize}
    \itemsep0em
    \item Magnitude $M_V$ determined by $|\overrightarrow{V}|=PQsin(\theta)$
    \item Direction $\hat{u}_V$ with a \textit{line of action} perpendicular to both original
      vectors and \textit{sense} determined by the right hand rule
  \end{itemize}
\end{formula}

To calculate the vector product of two vectors, the following process is used:


\begin{definition}{Principle of Transmissibility}
  Any force $\overrightarrow{F}$ acting on a rigid body at a given point can be replaced by
  an \textbf{equivalent force} $\overrightarrow{F}'$ provided that $\overrightarrow{F}$ and
  $\overrightarrow{F}'$ have the \uline{same magnitude}, \uline{same direction}, and
  \uline{same line of action}.
\end{definition}

\subsection{General Process}
\label{ssec:generalProcess}

%TODO

\newpage

\begin{example}
  \begin{wrapfigure}[]{r}{0.3\textwidth}
    \centering
    \includestandalone{figures/fig_006}
  \end{wrapfigure}

  A gardener is moving soil with a wheelbarrow. The length of the handle from $A$ to $B$ is
  $45in$, and in the position shows, $\theta$ is $25 \degree$. Find the moment created about
  the axle $A$ when she applies a verticle $30$ lb force at $B$ by resolving the force $F$
  into components parallel and perpendicular to the handle.

  \vspace{22pt}
  \hrule
  \vspace{12pt}

  \begin{figure}[H]
    \centering
    \begin{subfigure}[t]{0.2\textwidth}
      \centering
      \uline{Find:}
      \begin{equation*}
        \overrightarrow{M}_A
      \end{equation*}
    \end{subfigure}
    \vrule
    \begin{subfigure}[t]{0.3\textwidth}
      \centering
      \uline{Given:}
      \begin{align*}
        F &= 30 \textup{lb} \\
        l_{ab} &= 45 \textup{in} \\
        \theta &= 25 \degree
      \end{align*}
    \end{subfigure}
    \vrule
    \begin{subfigure}[t]{0.4\textwidth}
      \centering
      \uline{Knowns:}
      \begin{align*}
        \overrightarrow{M}_A &= \overrightarrow{r}_{b/a} \times \overrightarrow{F} \\
        M_A &= r_{b/a} F sin(\theta)
      \end{align*}
    \end{subfigure}
  \end{figure}

  \hrule
  \vspace{12pt}

  \textbf{1)} Define the reference frame ($x, y, z$ and $\hat{i},\hat{j},\hat{k}$)
  \begin{figure}[H]
    \centering
    \includestandalone{figures/fig_007}
  \end{figure}

  \textbf{2)} Find $\overrightarrow{r}_{b/a}$
  \begin{align*}
    \overrightarrow{r}_{b/a} &= \overrightarrow{r}_x\hat{i} + \overrightarrow{r}_y\hat{j} + \overrightarrow{r}_z\hat{k} \\
                             &= l_{ab}cos(\theta)\hat{i} + l_{ab}sin(\theta)\hat{j} + 0\hat{k}
  \end{align*}

  \textbf{3)} Find $\overrightarrow{F}$
  \begin{align*}
    \overrightarrow{F} &= \overrightarrow{F}_x\hat{i} + \overrightarrow{F}_y\hat{j} + \overrightarrow{F}_z\hat{k} \\
                       &= 0\hat{i} + |\overrightarrow{F}|\hat{j} + 0\hat{k}
  \end{align*}

  \textit{\hfill continued...}


\end{example}

\newpage
\begin{tcolorbox}[
  standard jigsaw, % Allows opacity
  colframe=fg,
  boxrule=1px,
  colback=black,
  opacityback=0,
  sharp corners,
  coltext=fg
  ]
  \textbf{Example Continued \hfill \thesection.\theexample}
  \vspace{5px}
  \hrule
  \vspace{5px}
  \noindent

  \textbf{4)} Evaluate $\overrightarrow{M}_A = \overrightarrow{r}_{b/a} \times \overrightarrow{F}$
  \begin{align*}
    \overrightarrow{M}_A &=
    \begin{vmatrix}
      \hat{i} & \hat{j} & \hat{k} \\
      \overrightarrow{r}_x & \overrightarrow{r}_y & \overrightarrow{r}_z \\
      \overrightarrow{F}_x & \overrightarrow{F}_y & \overrightarrow{F}_z
    \end{vmatrix}
    =
    \begin{vmatrix}
      \hat{i} & \hat{j} & \hat{k} \\
      l_{ab}cos(\theta) & l_{ab}sin(\theta) & 0 \\
      0 & |\overrightarrow{F}| & 0
    \end{vmatrix} \\
    &= \hat{i}
    \begin{vmatrix}
      l_{ab}sin(\theta) & 0 \\
      |\overrightarrow{F}| & 0
    \end{vmatrix}
    - \hat{j}
    \begin{vmatrix}
      l_{ab}cos(\theta) & 0 \\
      0 & 0
    \end{vmatrix}
    + \hat{k}
    \begin{vmatrix}
      l_{ab}cos(\theta) & l_{ab}sin(\theta) \\
      0 & |\overrightarrow{F}|
    \end{vmatrix} \\
    &= \hat{i}(l_{ab}cos(\theta) \cdot 0 - |\overrightarrow{F}| \cdot 0) - \hat{j}(l_{ab}cos(\theta) \cdot 0 - 0) + \hat{k}(l_{ab}cos(\theta) \cdot |\overrightarrow{F}| - l_{ab}sin(\theta) \cdot 0) \\
    &= (0)\hat{i} - (0)\hat{j} + (l_{ab}cos(\theta) \cdot |\overrightarrow{F}|)\hat{k} \\
    &= (0)\hat{i} - (0)\hat{j} + (45\textup{in}cos(25\degree) \cdot |30\textup{lb}|)\hat{k} \\
    &= (0)\hat{i} - (0)\hat{j} + (40.78\textup{in}) \cdot 30\textup{lb})\hat{k} \\
    &= (0)\hat{i} - (0)\hat{j} + (1223.51\textup{in} \cdot \textup{lb})\hat{k}
  \end{align*}

  \textbf{5)} Find the effective moment arm
  \begin{align*}
    \textup{M} &= \frac{\textup{Fr}sin(\beta)}{\textup{F}} \\
               &= \frac{30 \textup{lb} \cdot 45 \textup{in} \cdot sin(115 \degree)}{30 \textup{lb}} \\
               &= \frac{1223.5 \textup{lb} \cdot \textup{in}}{30 \textup{lb}} \\
               &= 40.9 \textup{in}
  \end{align*}
\end{tcolorbox}

\end{document}
