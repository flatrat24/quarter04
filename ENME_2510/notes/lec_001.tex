\documentclass[12pt]{article}

%%%% GRAPHICS %%%%
\usepackage{tikz}
\usepackage{circuitikz}
\usetikzlibrary{arrows.meta}
\usepackage{tikz-3dplot}
\usepackage{graphicx}
\usepackage{pgfplots}
  \pgfplotsset{compat=1.18}
\usetikzlibrary{arrows}
\newcommand{\midarrow}{\tikz \draw[-triangle 90] (0,0) -- +(.1,0);}

%%%% FIGURES %%%%
\usepackage{subcaption}
\usepackage{wrapfig}
\usepackage{float}
\usepackage[skip=5pt, font=footnotesize]{caption}

%%%% FORMATTING %%%%
\usepackage{parskip}
\usepackage{tcolorbox}
\usepackage{ulem}
% \usepackage{fancyhdr}

%%%% TABLE FORMATTING %%%%
\usepackage{tabularray}
\UseTblrLibrary{booktabs}

%%%% MATH AND LOGIC %%%%
\usepackage{xifthen}
\usepackage{amsmath}
\usepackage{amssymb}
\usepackage{amsfonts}

%%%% TEXT AND SYMBOLS %%%%
\usepackage[T1]{fontenc}
\usepackage{textcomp}
\usepackage{gensymb}

%%%% OTHER %%%%
\usepackage{standalone}

%%%% LOGIC SYMBOLS %%%%
\newcommand*\xor{\oplus}

%%%% STYLES %%%%

% Packages
\usepackage{fullpage}
\usepackage{titlesec}
\usepackage[rgb]{xcolor}
\selectcolormodel{natural}
\usepackage{ninecolors}
\selectcolormodel{rgb}

% Colors
\definecolor{pg}{HTML}{24273A}
\definecolor{fg}{HTML}{FFFFFF}
\definecolor{bg}{HTML}{24273A}
\definecolor{re}{HTML}{d20f39}
\definecolor{gr}{HTML}{40a02b}
\definecolor{ye}{HTML}{df8e1d}
\definecolor{or}{HTML}{fe640b}
\definecolor{bl}{HTML}{1e66f5}
\definecolor{ma}{HTML}{8839ef}
\definecolor{cy}{HTML}{179299}
\definecolor{pi}{HTML}{ea76cb}

\usepackage{nameref}
\makeatletter
\newcommand*{\currentname}{\@currentlabelname}
\makeatother

\titleformat{\section}
  {\normalfont\scshape\Large\bfseries}
  {\thesection}
  {0.75em}
  {}

\titleformat{\subsection}
  {\normalfont\scshape\large\bfseries}
  {\thesubsection}
  {0.75em}
  {}

\titleformat{\subsubsection}
  {\normalfont\scshape\normalsize\bfseries}
  {\thesubsubsection}
  {0.75em}
  {}

% Formula
\newcounter{formula}[section]
\newenvironment{formula}[1]{
  \stepcounter{formula}
  \begin{tcolorbox}[
    standard jigsaw, % Allows opacity
    colframe={fg},
    boxrule=1px,
    colback=bg,
    opacityback=0,
    sharp corners,
    sidebyside,
    righthand width=18px,
    coltext={fg}
  ]
  \centering
  \textbf{\uline{#1}}
}{
  \tcblower
  \textbf{\thesection.\theformula}
  \end{tcolorbox}
}

% Definition
\newcounter{definition}[section]

\newenvironment{definition*}[1]{
  \begin{tcolorbox}[
    standard jigsaw, % Allows opacity
    colframe={fg},
    boxrule=1px,
    colback=bg,
    opacityback=0,
    sharp corners,
    coltext={fg}
  ]
  \textbf{#1 \hfill}
  \vspace{5px}
  \hrule
  \vspace{5px}
  \noindent
}{
  \end{tcolorbox}
}

\newenvironment{definition}[1]{
  \stepcounter{definition}
  \begin{tcolorbox}[
    standard jigsaw, % Allows opacity
    colframe={fg},
    boxrule=1px,
    colback=bg,
    opacityback=0,
    sharp corners,
    coltext={fg}
  ]
  \textbf{#1 \hfill \thesection.\thedefinition}
  \vspace{5px}
  \hrule
  \vspace{5px}
  \noindent
}{
  \end{tcolorbox}
}

% Example Problem
\newcounter{example}[section]
\newenvironment{example}{
  \stepcounter{example}
  \begin{tcolorbox}[
    standard jigsaw, % Allows opacity
    colframe={fg},
    boxrule=1px,
    colback=bg,
    opacityback=0,
    sharp corners,
    coltext={fg}
  ]
  \textbf{Example \hfill \thesection.\theexample}
  \vspace{5px}
  \hrule
  \vspace{5px}
  \noindent
}{
  \end{tcolorbox}
}

\tikzset{
  cubeBorder/.style=fg,
  cubeFilling/.style={fg!20!bg, opacity=0.25},
  gridLine/.style={very thin, gray},
  graphLine/.style={-latex, thick, fg},
}

\pgfplotsset{
  basicAxis/.style={
    grid,
    major grid style={line width=.2pt,draw=fg!50!bg},
    axis lines = box,
    axis line style = {line width = 1px},
  }
}

%%%% REFERENCES %%%%
\usepackage{hyperref}
\hypersetup{
  colorlinks  = true,
  linkcolor   = pi,
  anchorcolor = pi,
  citecolor   = pi,
  filecolor   = pi,
  menucolor   = pi,
  runcolor    = pi,
  urlcolor    = pi,
}

\author{Ethan Anthony}


\title{Lecture 001}
\date{September 09, 2024}

\begin{document}

\section{Introduction to Mechanics}
\label{sec:introductionToMechanics}

The basic concepts used in mechanics are \textit{space, time, mass, and force}. Together,
they come to form the general idea of what mechanics is.

\begin{definition}{Mechanics}
  The study of what happens to an object or system when forces are applied to it.
  There are three bodies of mechanics: rigid body mechanics, mechanics of materials,
  and fluid mechanics.
\end{definition}

Elementary mechanics is derived from six basic principles.

\begin{definition}{Six Basic Principles of Mechanics}
  \textbf{The Parallelogram Law for the Addition of Forces} - Two separate forces acting upon
  the same object can be combined into the \textit{resultant} force by drawing the diagonal
  inside a parallelogram formed by two of each of the original forces.

  \textbf{The Principle of Transimmibility} - The equilibrium of an object remains unchanged
  if a force acting upon that object is replaced with another identical force acting along
  the same line of action.

  \textbf{Newton's Three Laws of Motion}
  \begin{itemize}
    \itemsep0em
    \item \textbf{First Law} - An object's motion remains unchanged if the resultant force
      on that object is zero.
    \item \textbf{Second Law} - An object will have an acceleration proportional to the resultant
      force acting upon it. The proportionality is described as: $\overrightarrow{F}=m \overrightarrow{a}$.
    \item \textbf{Third Law} - The forces of action and reaction between bodies in contact are 
      equal in magnitude and line of action, and opposite in direction.
  \end{itemize}

  \textbf{Newton's Law of Gravitation} - Two particles of mass $M$ and $m$ are attracted
  by forces $F$ and $-F$ respectively given by: $F=G \frac{Mm}{r^2}$.
\end{definition}

\subsection{Units}
\label{ssec:units}

There are two systems of units: the \textbf{International System of Units (SI)} and the \textbf{U.S. Customary}.
These systems measure the four basic physical quantities (space, time, mass, and force).

The \textbf{SI Base} units are what every other unit is derived from. They are listed in full in
Figure \ref{tbl:siBaseUnits}.

\begin{figure}[H]
  \begin{center}
    \begin{tblr}{lcc}
      \toprule
      \textbf{Base Quantity}    & \textbf{Name} & \textbf{Symbol} \\
      \midrule
      Length                    & meter         & m               \\
      Mass                      & kilogram      & kg              \\
      Time                      & second        & s               \\
      Electric Current          & ampere        & A               \\
      Thermodynamic Temperature & kelvin        & K               \\
      Amount of Substance       & mole          & mol             \\
      Luminous Intensity        & candela       & cd              \\
      \bottomrule
    \end{tblr}
    \caption{SI Base Units}
    \label{tbl:siBaseUnits}
  \end{center}
\end{figure}

Using the SI Base Units, there is also a set of \textbf{SI Derived Units}, meaning that they are
standard units that are created through combining the SI Base Units in different ways. They
are listed in Figure \ref{tbl:siDerivedUnits}.

\begin{figure}[H]
  \begin{center}
    \begin{tblr}{lccc}
      \toprule
      \textbf{Derived Quantity} & \textbf{Name}  & \textbf{Symbol} & \textbf{Equivalent Base Units} \\
      \midrule
      Frequency                 & hertz          & Hz              & $s^{-1}$                       \\
      Force                     & newtown        & N               & $\frac{m \cdot kg}{s^2}$       \\
      Pressure                  & pascal         & Pa              & $\frac{N}{m^2}$                \\
      Energy                    & joule          & J               & $N \cdot m$                    \\
      Power                     & watt           & W               & $\frac{J}{s}$                  \\
      Electric Charge           & coulomb        & C               & $s \cdot A$                    \\
      Electric Potential        & volt           & V               & $\frac{W}{A}$                  \\
      Electric Resistance       & ohm            & $\Omega$        & $\frac{V}{A}$                  \\
      Celsius Temperature       & degree Celsius & $\celsius$      & $K$                            \\
      \bottomrule
    \end{tblr}
    \caption{SI Derived Units}
    \label{tbl:siDerivedUnits}
  \end{center}
\end{figure}

Converting between SI and U.S. Customary, there each physical quantity has a corresponding
unit in both systems that are linearly related. The conversions are shows in Figure \ref{tbl:conversion}.

\begin{figure}[H]
  \centering
  \begin{tblr}{lccc}
    \toprule
    \textbf{Quantity} & \textbf{U.S. Customary} & \textbf{Equals} & \textbf{SI} \\
    \midrule
    Force  & $1lb$   & = & $4.448N$ \\
    Mass   & $1slug$ & = & $14.59kg$ \\
    Length & $1ft $  & = & $0.3048m$ \\
    Volume & $1gal$  & = & $3.785l$ \\
    \bottomrule
  \end{tblr}
  \caption{Unit Conversions}
  \label{tbl:conversion}
\end{figure}

Lastly, there is a set of \textbf{SI Unit Prefixes}, used to modify the magnitude of a unit.
The range of prefixes is listed in Figure \ref{tbl:siPrefix}.

\vspace{-6pt}

\begin{figure}[H]
  \begin{center}
    \begin{tblr}{lccl}
      \toprule
      \textbf{Factor} & \textbf{Name} & \textbf{Symbol} & \textbf{Numerical Value} \\
      \midrule
      $10^{15}$       & peta          & P               & $1\ 000\ 000\ 000\ 000\ 000$  \\
      $10^{12}$       & tera          & T               & $1\ 000\ 000\ 000\ 000$       \\
      $10^{9}$        & giga          & G               & $1\ 000\ 000\ 000$            \\
      $10^{6}$        & mega          & M               & $1\ 000\ 000$                 \\
      $10^{3}$        & kilo          & k               & $1\ 000$                      \\
      $10^{2}$        & hecto         & h               & $100$                         \\
      $10^{1}$        & deka          & da              & $10$                          \\
      $10^{-1}$       & deci          & d               & $0.1$                         \\
      $10^{-2}$       & centi         & c               & $0.01$                        \\
      $10^{-3}$       & milli         & m               & $0.001$                       \\
      $10^{-6}$       & micro         & $\mu$           & $0.000\ 001$                  \\
      $10^{-9}$       & nano          & n               & $0.000\ 000\ 001$             \\
      $10^{-12}$      & pico          & p               & $0.000\ 000\ 000\ 001$        \\
      $10^{-15}$      & femto         & f               & $0.000\ 000\ 000\ 000\ 001$   \\
      \bottomrule
    \end{tblr}
    \caption{SI Unit Prefixes}
    \label{tbl:siPrefix}
  \end{center}
\end{figure}

\end{document}
