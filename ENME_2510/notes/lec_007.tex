\documentclass[12pt]{article}

\input{../../xlatex/imports/preamble}

\title{Lecture 007}
\date{October 07, 2024}

\begin{document}

\newpage
\section{Shear and Bending}
\label{sec:shearAndBending}

\subsection{Notation and Convention}
\label{ssec:notationAndConvention}

Thusfar, only external forces have been considered. However internal forces also exist.
\begin{definition}{Internal Loads}
  Act \textit{everywhere} inside the object/body. These forces hold the object/body together
  and adapt/change based on external loading conditions.
\end{definition}
There are three main types of internal loads:

\begin{wrapfigure}[3]{r}{0.3\textwidth}
  \vspace{-10pt}
  \centering
  \includestandalone{figures/fig_020}
  % \caption{Axial Forces}
  \label{fig:020}
\end{wrapfigure}
\begin{center}
  {\large \textbf{Axial Forces ($N^+$)}}
\end{center}
These internal loads act directly along the length of the axis of an object.

The sign convention of Axial Forces is such that:
\begin{itemize}
  \itemsep0em
  \item Positive refers to \textbf{tension}
  \item Negative refers to \textbf{compression}
\end{itemize}

\begin{wrapfigure}[3]{r}{0.3\textwidth}
  \vspace{-20pt}
  \centering
  \includestandalone{figures/fig_021}
  % \caption{Shear Forces}
  \label{fig:021}
\end{wrapfigure}
\begin{center}
  {\large \textbf{Shear Forces ($V^+$)}}
\end{center}
Force \textit{transverse} to the length of the member (\textit{parallel} to the cross section area.
This force will always create a moment about the $z$ axis of the member. 

The sign convention is such that:
\begin{itemize}
  \itemsep0em
  \item Positive refers to \textbf{clockwise rotation}
  \item Negative refers to \textbf{counterclockwise rotation}
\end{itemize}

\begin{wrapfigure}[3]{r}{0.3\textwidth}
  \vspace{-35pt}
  \centering
  \includestandalone{figures/fig_022}
  % \caption{Bending Moments}
  \label{fig:022}
\end{wrapfigure}
\begin{center}
  {\large \textbf{Bending Moments ($M^+$)}}
\end{center}
\textit{Rotational} load that bends the beam.

The sign convention is such that:
\begin{itemize}
  \itemsep0em
  \item Positive refers to \textbf{concave upward bending}
  \item Negative refers to \textbf{downward bending}
\end{itemize}

Notice that a force in the positive $\hat{i}$, negative $\hat{j}$, or positive $\hat{k}$
is considered a positive internal force when it is on the right side of an object.
Conversely, these forces are considered negative when on the left side of the object.

\begin{figure}[H]
  \vspace{-15pt}
  \centering
  \includestandalone{figures/fig_028}
  \caption{Sign Conventions of Internal Forces}
  \label{fig:028}
\end{figure}

\subsection{Solving for Internal Forces}
\label{ssec:solvingForInternalForces}

All the information thusfar in Statics will be important in solving for internal forces
of an object in static equilibrium. The only difference to the general process is in the
preparation. When solving for an internal force, \textit{cutting} the object at that point
of interest and treating it as two separate objects is how to find the internal stress.

\begin{figure}[H]
  \centering
  \begin{subfigure}[H]{0.45\textwidth}
    \centering
    \includestandalone{figures/fig_023}
    \caption{Original Object}
    \label{fig:originalObject}
  \end{subfigure}
  \begin{subfigure}[H]{0.45\textwidth}
    \centering
    \includestandalone{figures/fig_024}
    \caption{Cut of the Object}
    \label{fig:cutOfTheObject}
  \end{subfigure}
  \caption{Cutting an Object}
  \label{fig:023}
\end{figure}

In Figure \ref{fig:023}, a beam attached to a vertical support is experiencing two forces:
$F_1$ and $F_2$. To find the internal forces acting on the object at point $A$, one must
cut the beam into two imaginary parts along point $A$, the solve for the forces as $A$ as
usual remaining under the conditions of static equilibrium.

\begin{figure}[H]
  \centering
  \includestandalone{figures/fig_025}
  \caption{Initial Free Body Diagram}
  \label{fig:025}
\end{figure}

The original object is in static equilibrium, thus it is possible to find the reactionary
forces and moments about a point ($B$) in terms of the forces acting upon the object ($F_1$ and $F_2$).
\begin{align*}
  F_{Rx} &= \hat{i} \big(|cos(\theta)F_1|\big) \\
  F_{Ry} &= \hat{j} \big(|F_2| + |sin(\theta)F_1|\big) \\
  M_{R}  &= \hat{k} \big(|F_2L_1| + |F_1sin(\theta)L|\big)
\end{align*}
If we were to then cut the object at the line of $A$, the reaction forces would be in
equilibrium with the internal forces at $A$, as would the original forces of $F_1$ and $F_2$.

\begin{figure}[H]
  \centering
  \begin{subfigure}[b]{0.45\textwidth}
    \centering
    \includestandalone{figures/fig_027}
    \caption{Reaction Forces and Internal Forces}
    \label{fig:027}
  \end{subfigure}
  \begin{subfigure}[b]{0.45\textwidth}
    \centering
    \includestandalone{figures/fig_026}
    \caption{Internal Forces and Original Forces}
    \label{fig:026}
  \end{subfigure}
  \caption{Free Body Diagram of Each Segment}
  \label{fig:fbdOfEachSegment}
\end{figure}
From here, each segment can be solved as usual for the forces in each Free Body Diagram.
Figure \ref{fig:027} would balance the reaction forces with the internal forces. Figure \ref{fig:026}
would balance the original forces with the internal forces. In either situation, the
calculated internal forces should be the same.
\begin{figure}[H]
  \centering
  \begin{subfigure}[H]{0.45\textwidth}
    \centering
    \textbf{Left Segment}
    \begin{align*}
      F_{Ix} &= -\hat{i}|F_{Rx}| \\
      F_{Iy} &= -\hat{j}|F_{Ry}| \\
      M_{I}  &= -\hat{k}|M_{R}|
    \end{align*}
  \end{subfigure}
  \begin{subfigure}[H]{0.45\textwidth}
    \centering
    \textbf{Right Segment}
    \begin{align*}
      F_{Ix} &= \hat{i}|cos(\theta)F_1| \\
      F_{Iy} &= \hat{j}\big(|sin(\theta)F_1| + |F_2|\big) \\
      M_{I}  &= \hat{k}\big(|sin(\theta)F_1(L_1+L_2)| + |F_2L_2|\big)
    \end{align*}
  \end{subfigure}
  \begin{subfigure}[H]{\textwidth}
    \vspace{22pt}
    \centering
    \textbf{Finding Equilibrium in Both Sides}

    \begin{subfigure}[H]{0.45\textwidth}
      \centering
      \begin{align*}
        -\hat{i}|F_{Rx}|  &\rightarrow \hat{i}|cos(\theta)F_1| \\
        -|F_{Rx}|         &\rightarrow |cos(\theta)F_1| \\
        -|cos(\theta)F_1| &\rightarrow |cos(\theta)F_1|
      \end{align*}
      \caption{Balancing $\hat{i}$ Forces}
      \label{onetwothree}
    \end{subfigure}
    \centering
    \begin{subfigure}[H]{0.45\textwidth}
      \begin{align*}
        -\hat{j}|F_{Ry}|                    &\rightarrow \hat{j}\big(|sin(\theta)F_1| + |F_2|\big) \\
        -|F_{Ry}|                           &\rightarrow |sin(\theta)F_1| + |F_2| \\
        -\big(|sin(\theta)F_1| + |F_2|\big) &\rightarrow |sin(\theta)F_1| + |F_2|
      \end{align*}
      \caption{Balancing $\hat{j}$ Forces}
      \label{onetwfw}
    \end{subfigure}
    \begin{subfigure}[H]{\textwidth}
      \centering
      \begin{align*}
        -\hat{k}|M_{R}|                            &\rightarrow \hat{k}\big(|sin(\theta)F_1(L_1+L_2)| + |F_2L_2|\big) \\
        -|M_{R}|                                   &\rightarrow |sin(\theta)F_1(L_1+L_2)| + |F_2L_2| \\
        -|\big(|F_2L_1| + |F_1sin(\theta)L|\big)|  &\rightarrow |sin(\theta)F_1(L_1+L_2)| + |F_2L_2|
      \end{align*}
      \caption{Balancing $\hat{k}$ Moments}
      \label{onetwoth}
    \end{subfigure}
  \end{subfigure}
\end{figure}

\end{document}
