\documentclass[12pt]{article}

\input{../../xlatex/imports/preamble}

\title{Lecture 002}
\date{September 11, 2024}

\begin{document}

\section{Representing Forces}
\label{sec:representingForces}

\subsection{Force Vectors}
\label{ssec:forceVectors}

When describing a force, there are three components to that \textbf{force vector}:

\begin{figure}[h]
  \begin{subfigure}[H]{0.6\textwidth}
    \begin{enumerate}
      \itemsep0em
      \item Point of application
      \item Magnitude of the force
      \item Direction of the force
    \end{enumerate}
  \end{subfigure}
  \begin{subfigure}[H]{0.3\textwidth}
    \centering
    \includestandalone{figures/fig_001}
  \end{subfigure}
\end{figure}


\begin{definition}{Concurrent Forces}
  Two or more forces that are acting on the \textit{same point}.
\end{definition}

When dealing with concurrent forces, you can combine the two forces through vector addition
to create a single force acting on that same point. That force is called the resultant force.

\begin{definition}{Resultant Force}
  The vector sum of two or more concurrent forces.
  \begin{equation*}
    \sum_{}^{} \overrightarrow{F} = \overrightarrow{F}_R = \overrightarrow{F}_1 + \overrightarrow{F}_2 +...
  \end{equation*}
\end{definition}

\subsection{Vectors and Scalars}
\label{ssec:vectorsAndScalars}

\begin{figure}[H]
  \centering
  \includestandalone{figures/fig_002}
  \caption{Vector}
  \label{fig:002}
\end{figure}

\textbf{Vectors} should be represented with an arrow above the variable. For example, 
a force would be written as $\overrightarrow{\textup{F}}$.

\textbf{Magnitude} is written as an italic version of the vector. So, $|\overrightarrow{\textup{F}}| = F$.

Vector addition with can be done geometrically by creating a triangle, aligning the vectors
tip-to-tail.

\begin{figure}[H]
  \begin{subfigure}[H]{0.3\textwidth}
    \centering
    \includestandalone{figures/fig_003}
  \end{subfigure}
  \begin{subfigure}[H]{0.3\textwidth}
    \centering
    \includestandalone{figures/fig_004}
  \end{subfigure}
  \begin{subfigure}[H]{0.3\textwidth}
    \centering
    \includestandalone{figures/fig_005}
  \end{subfigure}
  \caption{Vector Addition}
  \label{fig:003}
\end{figure}

\textbf{Law of Sines} relates the lengths of the sides of any shaped triangle to the sines
of its angles.

\begin{formula}{Law of Sines}
  \begin{equation*}
    \frac{sin( \alpha )}{|\textup{\textbf{F}}_1|} = \frac{sin( \beta )}{|\textup{\textbf{F}}_2|} = \frac{sin( \gamma )}{|\textup{\textbf{F}}_R|}
  \end{equation*}
\end{formula}

\textbf{Law of Cosines} relates the lengths of the sides of any shaped triangle to the cosine
of one of its angles.

\begin{formula}{Law of Cosines}
  \begin{equation*}
    |\textup{\textbf{F}}_R|^2 = |\textup{\textbf{F}}_1|^2 + |\textup{\textbf{F}}_2|^2 - 2|\textup{\textbf{F}}_1||\textup{\textbf{F}}_2|cos( \gamma )
  \end{equation*}
\end{formula}

\subsection{Unit Vectors and Components of Vectors}
\label{ssec:unitVectorsAndComponentsOfVectors}

\textbf{Unit Vector} is a dimensionless vector with a magnitude equal to one. 

\begin{equation*}
  \hat{u} = \frac{u}{|u|}
\end{equation*}

When considering a $3D$ space going in the $x$, $y$, and $z$ directions, the corresponding
unit vectors are $\hat{i}$, $\hat{j}$, and $\hat{k}$ respectively.

When resolving a vector, you can take a force $F$, and break it into its $x$ and $y$ components.
However, $F_x$ has no direction, so it must be accompanied by a unit vector to specify its direction.
\begin{equation*}
  \overrightarrow{F} = F_x\hat{i} + F_y\hat{j} \\
\end{equation*}

To find the magnitude of a vector:

\begin{equation*}
  F = |\textup{F}| = \sqrt{(F_x)^2 + (F_y)^2}
\end{equation*}

To find its direction:

\begin{equation*}
  \hat{u}_F = \frac{F}{|F|} = \left(\frac{F_x}{F}\right)\hat{i} + \left(\frac{F_y}{F}\right)\hat{j}
\end{equation*}

\end{document}
