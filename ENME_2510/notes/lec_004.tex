\documentclass[12pt]{article}

%%%% GRAPHICS %%%%
\usepackage{tikz}
\usepackage{circuitikz}
\usetikzlibrary{arrows.meta}
\usepackage{tikz-3dplot}
\usepackage{graphicx}
\usepackage{pgfplots}
  \pgfplotsset{compat=1.18}
\usetikzlibrary{arrows}
\newcommand{\midarrow}{\tikz \draw[-triangle 90] (0,0) -- +(.1,0);}

%%%% FIGURES %%%%
\usepackage{subcaption}
\usepackage{wrapfig}
\usepackage{float}
\usepackage[skip=5pt, font=footnotesize]{caption}

%%%% FORMATTING %%%%
\usepackage{parskip}
\usepackage{tcolorbox}
\usepackage{ulem}
% \usepackage{fancyhdr}

%%%% TABLE FORMATTING %%%%
\usepackage{tabularray}
\UseTblrLibrary{booktabs}

%%%% MATH AND LOGIC %%%%
\usepackage{xifthen}
\usepackage{amsmath}
\usepackage{amssymb}
\usepackage{amsfonts}

%%%% TEXT AND SYMBOLS %%%%
\usepackage[T1]{fontenc}
\usepackage{textcomp}
\usepackage{gensymb}

%%%% OTHER %%%%
\usepackage{standalone}

%%%% LOGIC SYMBOLS %%%%
\newcommand*\xor{\oplus}

%%%% STYLES %%%%

% Packages
\usepackage{fullpage}
\usepackage{titlesec}
\usepackage[rgb]{xcolor}
\selectcolormodel{natural}
\usepackage{ninecolors}
\selectcolormodel{rgb}

% Colors
\definecolor{pg}{HTML}{24273A}
\definecolor{fg}{HTML}{FFFFFF}
\definecolor{bg}{HTML}{24273A}
\definecolor{re}{HTML}{d20f39}
\definecolor{gr}{HTML}{40a02b}
\definecolor{ye}{HTML}{df8e1d}
\definecolor{or}{HTML}{fe640b}
\definecolor{bl}{HTML}{1e66f5}
\definecolor{ma}{HTML}{8839ef}
\definecolor{cy}{HTML}{179299}
\definecolor{pi}{HTML}{ea76cb}

\usepackage{nameref}
\makeatletter
\newcommand*{\currentname}{\@currentlabelname}
\makeatother

\titleformat{\section}
  {\normalfont\scshape\Large\bfseries}
  {\thesection}
  {0.75em}
  {}

\titleformat{\subsection}
  {\normalfont\scshape\large\bfseries}
  {\thesubsection}
  {0.75em}
  {}

\titleformat{\subsubsection}
  {\normalfont\scshape\normalsize\bfseries}
  {\thesubsubsection}
  {0.75em}
  {}

% Formula
\newcounter{formula}[section]
\newenvironment{formula}[1]{
  \stepcounter{formula}
  \begin{tcolorbox}[
    standard jigsaw, % Allows opacity
    colframe={fg},
    boxrule=1px,
    colback=bg,
    opacityback=0,
    sharp corners,
    sidebyside,
    righthand width=18px,
    coltext={fg}
  ]
  \centering
  \textbf{\uline{#1}}
}{
  \tcblower
  \textbf{\thesection.\theformula}
  \end{tcolorbox}
}

% Definition
\newcounter{definition}[section]

\newenvironment{definition*}[1]{
  \begin{tcolorbox}[
    standard jigsaw, % Allows opacity
    colframe={fg},
    boxrule=1px,
    colback=bg,
    opacityback=0,
    sharp corners,
    coltext={fg}
  ]
  \textbf{#1 \hfill}
  \vspace{5px}
  \hrule
  \vspace{5px}
  \noindent
}{
  \end{tcolorbox}
}

\newenvironment{definition}[1]{
  \stepcounter{definition}
  \begin{tcolorbox}[
    standard jigsaw, % Allows opacity
    colframe={fg},
    boxrule=1px,
    colback=bg,
    opacityback=0,
    sharp corners,
    coltext={fg}
  ]
  \textbf{#1 \hfill \thesection.\thedefinition}
  \vspace{5px}
  \hrule
  \vspace{5px}
  \noindent
}{
  \end{tcolorbox}
}

% Example Problem
\newcounter{example}[section]
\newenvironment{example}{
  \stepcounter{example}
  \begin{tcolorbox}[
    standard jigsaw, % Allows opacity
    colframe={fg},
    boxrule=1px,
    colback=bg,
    opacityback=0,
    sharp corners,
    coltext={fg}
  ]
  \textbf{Example \hfill \thesection.\theexample}
  \vspace{5px}
  \hrule
  \vspace{5px}
  \noindent
}{
  \end{tcolorbox}
}

\tikzset{
  cubeBorder/.style=fg,
  cubeFilling/.style={fg!20!bg, opacity=0.25},
  gridLine/.style={very thin, gray},
  graphLine/.style={-latex, thick, fg},
}

\pgfplotsset{
  basicAxis/.style={
    grid,
    major grid style={line width=.2pt,draw=fg!50!bg},
    axis lines = box,
    axis line style = {line width = 1px},
  }
}

%%%% REFERENCES %%%%
\usepackage{hyperref}
\hypersetup{
  colorlinks  = true,
  linkcolor   = pi,
  anchorcolor = pi,
  citecolor   = pi,
  filecolor   = pi,
  menucolor   = pi,
  runcolor    = pi,
  urlcolor    = pi,
}

\author{Ethan Anthony}


\title{Lecture 005}
\date{September 19, 2024}

\begin{document}

\newpage
\section{Free Body Diagrams}
\label{sec:freeBodyDiagrams}

\subsection{Position Vectors}
\label{ssec:positionVectors}

\begin{definition}{Position Vector}
  A fixed vector which locates a point in space \textit{relative} to another point. Any point
  in space, given that a $xyz$ coordinate system has been established, can be described as
  a position vector.
\end{definition}

A position vector can also be used to describe the position of a point relative to \textit{another}
point. This can be done following these steps:
\begin{enumerate}
  \itemsep0em
  \item Identify the coordinates of each point
    \begin{align*}
      A &= (x_A, y_A, z_A) \\
      B &= (x_B, y_B, z_B)
    \end{align*}
  \item Determine the \textit{head} and \textit{tail} of the position vector
    \begin{center}
      \uline{from} point $A$ \uline{to} point $B$ $\Rightarrow$ $\overrightarrow{r}_{A/B}$ $\Rightarrow$ tail at $A$, head at $B$ \\
      \uline{from} point $B$ \uline{to} point $A$ $\Rightarrow$ $\overrightarrow{r}_{B/A}$ $\Rightarrow$ tail at $B$, head at $A$
    \end{center}
  \item Subtract \textit{tail} from \textit{head} coordinates
    \begin{align*}
      \overrightarrow{r}_{A/B} = (x_A - x_B)\hat{i} + (y_A - y_B)\hat{j} + (z_A - z_B)\hat{k} \\
      \overrightarrow{r}_{B/A} = (x_B - x_A)\hat{i} + (y_B - y_A)\hat{j} + (z_B - z_A)\hat{k}
    \end{align*}
\end{enumerate}

This process is the same process for finding the distance of some point $A$ from the origin
since the origin is located at ($0\hat{i},0\hat{j},0\hat{k}$).

\subsection{Free Body Diagrams}
\label{ssec:freeBodyDiagrams}

\begin{definition}{Free Body Diagram}
  A diagram modeling all forces and torques applied to a body at a given point in time.
\end{definition}

What is included in a \textbf{Free Body Diagram (FBD)}?
\begin{itemize}
  \itemsep0em
  \item The structure being analyzed (particle, rigid body, etc.)
  \item \textbf{\uline{ALL}} forces and moments acting on the body
  \item Relevant dimensions
\end{itemize}

Steps to drawing a FBD:
\begin{enumerate}
  \itemsep0em
  \item Draw the isolated particle
  \item Draw all forces acting on the particle
  \item Label all known force magnitudes and/or directions
  \item Label all unknown force magnitudes and/or directions as variables
\end{enumerate}


\end{document}
