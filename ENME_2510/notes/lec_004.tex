\documentclass[12pt]{article}

\input{../../xlatex/imports/preamble}

\title{Lecture 005}
\date{September 19, 2024}

\begin{document}

\newpage
\section{Free Body Diagrams}
\label{sec:freeBodyDiagrams}

\subsection{Position Vectors}
\label{ssec:positionVectors}

\begin{definition}{Position Vector}
  A fixed vector which locates a point in space \textit{relative} to another point. Any point
  in space, given that a $xyz$ coordinate system has been established, can be described as
  a position vector.
\end{definition}

A position vector can also be used to describe the position of a point relative to \textit{another}
point. This can be done following these steps:
\begin{enumerate}
  \itemsep0em
  \item Identify the coordinates of each point
    \begin{align*}
      A &= (x_A, y_A, z_A) \\
      B &= (x_B, y_B, z_B)
    \end{align*}
  \item Determine the \textit{head} and \textit{tail} of the position vector
    \begin{center}
      \uline{from} point $A$ \uline{to} point $B$ $\Rightarrow$ $\overrightarrow{r}_{A/B}$ $\Rightarrow$ tail at $A$, head at $B$ \\
      \uline{from} point $B$ \uline{to} point $A$ $\Rightarrow$ $\overrightarrow{r}_{B/A}$ $\Rightarrow$ tail at $B$, head at $A$
    \end{center}
  \item Subtract \textit{tail} from \textit{head} coordinates
    \begin{align*}
      \overrightarrow{r}_{A/B} = (x_A - x_B)\hat{i} + (y_A - y_B)\hat{j} + (z_A - z_B)\hat{k} \\
      \overrightarrow{r}_{B/A} = (x_B - x_A)\hat{i} + (y_B - y_A)\hat{j} + (z_B - z_A)\hat{k}
    \end{align*}
\end{enumerate}

This process is the same process for finding the distance of some point $A$ from the origin
since the origin is located at ($0\hat{i},0\hat{j},0\hat{k}$).

\subsection{Free Body Diagrams}
\label{ssec:freeBodyDiagrams}

\begin{definition}{Free Body Diagram}
  A diagram modeling all forces and torques applied to a body at a given point in time.
\end{definition}

What is included in a \textbf{Free Body Diagram (FBD)}?
\begin{itemize}
  \itemsep0em
  \item The structure being analyzed (particle, rigid body, etc.)
  \item \textbf{\uline{ALL}} forces and moments acting on the body
  \item Relevant dimensions
\end{itemize}

Steps to drawing a FBD:
\begin{enumerate}
  \itemsep0em
  \item Draw the isolated particle
  \item Draw all forces acting on the particle
  \item Label all known force magnitudes and/or directions
  \item Label all unknown force magnitudes and/or directions as variables
\end{enumerate}


\end{document}
