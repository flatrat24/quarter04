\documentclass[12pt]{article}

\input{../../xlatex/imports/preamble}

\title{Lecture 006}
\date{September 23, 2024}

\begin{document}

\section{Rigid Body Equilibrium}
\label{sec:rigidBodyEquilibrium}

\subsection{Defining Rigid Body Equilibrium}
\label{ssec:definingRigidBodyEquilibrium}

In Rigid Body Equilibrium, the sum of all forces and all moments acting on a body sums to
zero. In other words, the net force and net moment on the object is zero.
\begin{align*}
  \sum_{}^{} \overrightarrow{F} &= \overrightarrow{0} = \sum_{}^{} F_x \hat{i} + \sum_{}^{} F_y \hat{j} + \sum_{}^{} F_z \hat{k} = 0 \hat{i} + 0 \hat{j} + 0 \hat{k} \\
  \sum_{}^{} \overrightarrow{M} &= \overrightarrow{0} = \sum_{}^{} M_x \hat{i} + \sum_{}^{} M_y \hat{j} + \sum_{}^{} M_z \hat{k} = 0 \hat{i} + 0 \hat{j} + 0 \hat{k} \\
\end{align*}

\end{document}
