\documentclass[12pt]{article}

\input{../../xlatex/imports/preamble}

\title{Lecture 008}
\date{October 21, 2024}

\begin{document}
\newpage
\section{Trusses}
\label{sec:trusses}

\subsection{Terminology}
\label{ssec:terminology}

\begin{definition}{Structure}
  An assembly of individual bodies that are designed to support and/or transmit external
  and internal loads.
\end{definition}

There are multiple types of structures:
\begin{itemize}
  \itemsep0em
  \item \textbf{Trusses} - Stationary and composed of only two-force bodies, designed to \textit{support} loads.
  \item \textbf{Frames} - Stationary and contain at least multi-force body, designed to \textit{support} loads
  \item \textbf{Machines} - Contain moving parts and are designed to \textit{transmit} and/or \textit{alter} the effects of forces
\end{itemize}

For a \textbf{truss} specifically, there are a few characteristics that all of them share.
\begin{itemize}
  \itemsep0em
  \item Consists of straight, slender, and \textbf{two-force} members
  \item Each member is connected only by pin joints at their ends
  \item Each member is only loaded at the pin joints
  \item Each member \textbf{only} supports an \textit{axial} load
\end{itemize}

\subsection{Solving a Truss}
\label{ssec:solvingATruss}

\end{document}
