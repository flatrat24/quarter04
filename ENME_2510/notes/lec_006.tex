\documentclass[12pt]{article}

\input{../../xlatex/imports/preamble}
\graphicspath{ {./images/} }

\title{Lecture 007}
\date{September 30, 2024}

\begin{document}

\newpage
\section{Distributed Loads}
\label{sec:distributedLoads}

\begin{wrapfigure}[3]{r}{0.3\textwidth}
  \vspace{-30pt}
  \centering
  \includestandalone{figures/fig_010}
  \caption{Continuous Load}
  \label{fig:010}
\end{wrapfigure}

Up until now, the forces considered have been single lines acting on points of an object.
However, in nature, many forces act continuously along an entire segment of an object. These
forces are called \textbf{Distribute Loads}.

\begin{definition}{Distributed Loads}
  Forces that act continuously over a line or a surface.
  \begin{itemize}
    \itemsep0em
    \item Force per unit distance ($\frac{N}{m}$)
    \item Force per unit area ($\frac{N}{m^2}$ ; $Pa$)
  \end{itemize}
\end{definition}

\subsection{Centroids}
\label{ssec:centroids}

For any distributed load, the center of force of the load can be considered as the line
of action of the force. For example, in Figure \ref{fig:010}, the center of the orange object
would be the line of action of the force acting upon the beam. This center is called the \textbf{centroid}.

For simple shapes, finding the centroid of the force is simple. Figure \ref{fig:centroidsOfSimpleShapes}
shows the centroids for common shapes.

\begin{figure}[H]
  \centering
  \begin{subfigure}[H]{0.45\textwidth}
    \centering
    \begin{subfigure}[H]{0.45\textwidth}
      \centering
      \includestandalone{figures/fig_011}
    \end{subfigure}
    \begin{subfigure}[H]{0.45\textwidth}
      \begin{align*}
        C_x &= l \div 2 \\
        C_y &= l \div 2
      \end{align*}
    \end{subfigure}
    \caption{Square}
    \label{fig:011}
  \end{subfigure}
  \vrule
  \begin{subfigure}[H]{0.45\textwidth}
    \centering
    \begin{subfigure}[H]{0.45\textwidth}
      \centering
      \includestandalone{figures/fig_012}
    \end{subfigure}
    \begin{subfigure}[H]{0.45\textwidth}
      \begin{align*}
        C_x &= r_x \\
        C_y &= r_y
      \end{align*}
    \end{subfigure}
    \caption{Circle}
    \label{fig:012}
  \end{subfigure}

  \vspace{20pt}
  \hrule
  \vspace{20pt}

  \begin{subfigure}[H]{0.45\textwidth}
    \centering
    \begin{subfigure}[H]{0.45\textwidth}
      \centering
      \includestandalone{figures/fig_013}
    \end{subfigure}
    \begin{subfigure}[H]{0.45\textwidth}
      \begin{align*}
        C_x &= r_x \\
        C_y &= \frac{4r_y}{3 \pi}
      \end{align*}
    \end{subfigure}
    \caption{Semi-Circle}
    \label{fig:013}
  \end{subfigure}
  \vrule
  \begin{subfigure}[H]{0.45\textwidth}
    \centering
    \begin{subfigure}[H]{0.45\textwidth}
      \centering
      \includestandalone{figures/fig_014}
    \end{subfigure}
    \begin{subfigure}[H]{0.45\textwidth}
      \begin{align*}
        C_x &= \frac{2}{3}b \\
        C_y &= \frac{1}{3}h
      \end{align*}
    \end{subfigure}
    \caption{Right Triangle}
    \label{fig:014}
  \end{subfigure}
  \caption{Centroids of Simple Shapes}
  \label{fig:centroidsOfSimpleShapes}
\end{figure}

All this is doing is finding the average position of the object. For all of the mass in the object,
what is the average position of it. It is simple to see that the average position of an object
that is symmetrical over the $x$ and $y$ axis has a centroid in the exact middle of the 
object like the square or circle in Figure \ref{fig:centroidsOfSimpleShapes}.

How would the centroid of some non-symmetrical shape be found?

\begin{wrapfigure}[5]{r}{0.4\textwidth}
  \centering
  \includestandalone{figures/fig_016}
  \caption{Reimann's Like Subdivision}
  \label{fig:016}
\end{wrapfigure}

To find the average $x$ position of the shape (denoted as $\overline{x}$), the following
formula is used:
\begin{formula}{Centroid in One Axis}
  \begin{equation*}
    \overline{x} = \frac{\sum_{}^{} x_iA_i}{\sum_{}^{} A} = \frac{\int_{0}^{L} x\ dA \,}{\int_{0}^{L} dA \,}
  \end{equation*}
\end{formula}
The summation formula could be used for discrete quantities, such as finding the average
age among a group of people where each person and their age are integers. The integral is
used for more natural shapes, such as a continuous (distributed) load.

The formula essentially is dividing the shape into infinitesimally small portions, and finding
the product of their area and $x$-position. It takes the total of that and divides it by
the total area of a shape to find the weighted average, or the \textbf{centroid} of the shape.

\begin{figure}[H]
  \centering
  \begin{subfigure}[H]{0.45\textwidth}
    \centering
    \includestandalone{figures/fig_015}
  \end{subfigure}
  \begin{subfigure}[H]{0.5\textwidth}
    \centering
    \begin{align*}
      \overline{x} &= \frac{\int_{0}^{L} x\ dA \,}{\int_{0}^{L} dA \,} = \frac{\int_{0}^{3} (x)\left(\frac{1}{2}x\ dx\right) \,}{\int_{0}^{3} \frac{1}{2}x\ dx \,} \\
                   &= \frac{\frac{1}{2}\int_{0}^{3} x^2\ dx \,}{\frac{1}{2}\int_{0}^{3}x\ dx \,} = \frac{\frac{1}{2}\left[\frac{1}{3}x^3\right]_{0}^{3}}{\frac{1}{2}\left[\frac{1}{2}x^2\right]_{0}^{3}} \\
                   &= \frac{\frac{1}{2}\big[\left(\frac{1}{3}\cdot3^3\right)-\left(\frac{1}{3}\cdot0^3\right)\big]}{\frac{1}{2}\big[\left(\frac{1}{2}\cdot3^2\right)-\left(\frac{1}{2}\cdot0^2\right)\big]} \\
                   &= \frac{\frac{1}{2}\cdot(9-0)}{\frac{1}{2}\cdot(\frac{9}{2}-0)} = \frac{\frac{9}{2}}{\frac{9}{4}} = \frac{36}{18} = \frac{2}{1} = 2
    \end{align*}
  \end{subfigure}
\end{figure}

\subsection{Finding Total Force}
\label{ssec:findingTotalForce}

The second step to calculating distributed loads is to find the total force being applied.
With simple shapes, this can be done geometrically. In Figure \ref{fig:018}, the total force
would be:
\begin{equation*}
  \frac{1}{2} \cdot 4m \cdot 1.5N = 3Nm
\end{equation*}

\begin{wrapfigure}[10]{l}{0.45\textwidth}
  \centering
  \includestandalone{figures/fig_018}
  \caption{Finding Total Force}
  \label{fig:018}
\end{wrapfigure}

However, for more complex shapes, the total force can't simply be calculated geometrically.
In these cases, integrating is necessary (integrating can be used for simple shapes as well).

Still using Figure \ref{fig:018}, the total force can be calculated with integrals. First,
the magnitude of the force at an given point across the base of the triangle can be expressed
as a function of $x$, with $x$ being the distance along the base the point is.
\begin{equation*}
  w(x) = \left(\frac{w_b - w_a}{b}\right)x + w_a = \left(\frac{1.5 - 0}{4}\right)x + 0 = \frac{1.5}{4}x
\end{equation*}
Then, integrating this function along the length of the continuous force (from $0m$ to $4m$)
and evaluating will yield the total force being applied.
\begin{align*}
  W &= \int_{0}^{4} \frac{1.5}{4}x \,dx = \frac{1.5}{4}\left[\frac{x^2}{2}\right]_{0}^{4} = \frac{1.5}{4}\left[\left(\frac{4^2}{2}\right)-\left(\frac{0^2}{2}\right)\right] = \frac{1.5}{4} \cdot 8 = 3\ Nm
\end{align*}

\subsection{Calculating Distributed Loads}
\label{ssec:calculatingDistributedLoads}

Once the centroid and the total force of a distributed load is calculated, it can be treated
just as a force along the line of action that its centroid is on of a magnitude of its
total force.
\begin{figure}[H]
  \centering
  \includestandalone{figures/fig_017}
  \caption{Distributed Load}
  \label{fig:017}
\end{figure}
{\large \textbf{Finding the Centroid}}
\begin{align*}
  \overline{x} &= \frac{\int_{0}^{L} x \,dA}{\int_{0}^{L}  \,dA} = \frac{\int_{0}^{L} x \left(w_A + \frac{w_B - w_A}{L}x\right) \,dx}{\int_{0}^{L} {w_A + \frac{w_B - w_A}{L}x} \,dx} \Rightarrow \frac{\int_{0}^{L} x \left(1500 + \frac{4500 - 1500}{6}x\right) \,dx}{\int_{0}^{6} {1500 + \frac{4500 - 1500}{6}x} \,dx} = \frac{\int_{0}^{L} 1500x + \frac{3000}{6}x^2 \,dx}{\int_{0}^{6} {1500 + \frac{3000}{6}x} \,dx} \\
               &= \frac{\left[\frac{1500}{2}x^2 + \frac{3000}{18}x^3\right]_{0}^{6}}{\left[1500x + \frac{3000}{12}x^2\right]_{0}^{6}} = \frac{\left(\frac{1500}{2}\cdot6^2 + \frac{3000}{18}\cdot6^3\right)-\left(\frac{1500}{2}\cdot0^2 + \frac{3000}{18}\cdot0^3\right)}{\left(1500\cdot6 + \frac{3000}{12}\cdot6^2\right)-\left(1500\cdot0 + \frac{3000}{12}\cdot0^2\right)} = 3.5
\end{align*}
{\large \textbf{Finding the Magnitude of the Force}}
\begin{align*}
  w(x) = \left(\frac{w_B - w_A}{L}\right)x + w_A = \left(\frac{4500 - 1500}{6}\right)x + 1500 = \frac{3000}{6}x + 1500 = 500x + 1500
\end{align*}
\begin{gather*}
  W = \int_{0}^{L} w(x) \,dx = \int_{0}^{6} 500x + 1500 \,dx = \left[\frac{500}{2}x^2 + 1500x\right]_{0}^{6} \\
  = \left(\frac{500}{2}\cdot6^2 + 1500 \cdot 6\right)-\left(\frac{500}{2}\cdot0^2 + 1500 \cdot 0\right) = 18000\ Nm
\end{gather*}

Thus, with those two calculations, we can find that the equivalent force for the distributed
load that acts upon the beam is a force of $1.8 \times 10^4\ Nm$ acting in the $-\hat{j}$ direction
at a point $3.5\ m$ from the leftmost point of the beam. This is illustrated in Figure \ref{fig:019}.
\begin{figure}[H]
  \centering
  \includestandalone{figures/fig_019}
  \caption{Equivalent Force}
  \label{fig:019}
\end{figure}

\end{document}
