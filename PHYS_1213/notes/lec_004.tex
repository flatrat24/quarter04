\documentclass[12pt]{article}

\input{../../xlatex/imports/preamble}

\title{Lecture 004}
\date{September 25, 2024}

\begin{document}

\newpage
\section{Electric Potential Energy}
\label{sec:electricPotentialEnergy}

\subsection{Review of Work and Energy}
\label{ssec:reviewOfWorkAndEnergy}

When a force $\overrightarrow{F}$ acts upon a particle that moves from point $a$ to point $b$,
the work done on that particle is given by an integral.
\begin{formula}{Work Done by a Force}
  \begin{equation*}
    W_{a \rightarrow b} = \int_{a}^{b} \overrightarrow{F} \cdot d\overrightarrow{r} \, = \int_{a}^{b} Fcos(\theta)dr \,
  \end{equation*}
\end{formula}
If the force acting upon that particle is \textit{conservative}, the work done by $\overrightarrow{F}$
can be expressed in terms of potential energy ($U$).
\begin{formula}{Work in Terms of Potential Energy}
  \begin{equation*}
    W_{a \rightarrow b} = U_a - U_b = -(U_b - U_a) = - \Delta U
  \end{equation*}
\end{formula}
The work-energy theorem states that the net work done on a particle is equal to the change
in kinetic energy of that particle: $W_{net} = \Delta K = K_b - K_a$. Under the assumption
that only conservative forces are acting upon the object, then:
\begin{formula}{Conservation of Mechanical Energy}
  \begin{align*}
    K_b - K_a &= -(U_b - U_a) \\
    K_a + U_a &= K_b + U_b
  \end{align*}
\end{formula}



\subsection{Electric Potential Energy}
\label{ssec:electricPotentialEnergy}

\begin{figure}[H]
  \centering
  \includestandalone{figures/fig_016}
  \caption{Potential Energy in an Electric Field}
  \label{fig:016}
\end{figure}

Electric potential energy applies within electric fields. In Figure \ref{fig:016}, a particle with
a positive charge in a field, it would have the highest potential energy closes to the
\textbf{source} of the positive electric field.

\begin{figure}[H]
  \centering
  \includestandalone{figures/fig_017}
  \caption{Potential Energy in an Electric Field}
  \label{fig:017}
\end{figure}

In Figure \ref{fig:017}, the positively charged particle would have a relatively low potential
energy. The work done on the particle in Figure \ref{fig:017} can be modeled as:
\begin{align*}
  W &= F(r_b - r_a) \\
  W &= Eq(r_b - r_a)
\end{align*}

This only holds for particles in uniform electric fields. If, for example, two charged particles
exist in space and perform work on each other, an integral would be required to find the net
work done on each particle. This integral is the general form to find the work done on a
particle by an electric field.

\begin{formula}{Work Done on a Particle by an Electric Field}
  \begin{align*}
    W_{a \rightarrow b} &= \int_{r_a}^{r_b} Fcos(\theta) dl \,
  \end{align*}
\end{formula}

\begin{formula}{Electric Potential Energy of Two Point Charges}
  \begin{equation*}
    U = \frac{1}{4 \pi \epsilon_0}\frac{qq_0}{r}
  \end{equation*}
\end{formula}

Electric potential energy is not a property of a single particle. In a system of two particles,
the electric potential energy is a shared property of both particles, just as gravitational
potential energy is shared between both you and the Earth.

\begin{wrapfigure}[]{l}{0.4\textwidth}
  \centering
  \includestandalone{figures/fig_022}
  \caption{System of Particles}
  \label{fig:022}
\end{wrapfigure}

If there is a system of charges in space, what is the total potential energy of this system?
In other words, how much energy would it take to bring all the particles together from infinity?

The total $U$ would just be the potential energy that each particle creates relative to each
other particle:
\begin{equation*}
  U = k\left(\frac{q_aq_b}{r_{ab}} + \frac{q_bq_c}{r_{bc}} + \frac{q_cq_a}{r_{ca}}\right)
\end{equation*}

Thus, the potential energy of a system of particles is just the sum of all potential energies
between each of those particles. Similarly, the work done on a particle as it moves through
the system is simply the final potential energy minus the initial potential energy.

\end{document}
