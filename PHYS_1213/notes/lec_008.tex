\documentclass[12pt]{article}

\input{../../xlatex/imports/preamble}

\title{Lecture 008}
\date{October 16, 2024}

\begin{document}
\newpage
\section{Circuits}
\label{sec:circuits}

\subsection{Current Throughout a Circuit}
\label{ssec:currentThroughoutACircuit2}

\begin{figure}[H]
  \centering
  \begin{subfigure}[b]{0.45\textwidth}
    \centering
    \includestandalone{figures/fig_030}
    \begin{equation*}
      R_{eq} = R_1 + R_2 + R_3
    \end{equation*}
    \caption{Resistors in Series}
    \label{fig:030}
  \end{subfigure}
  \begin{subfigure}[b]{0.45\textwidth}
    \centering
    \includestandalone{figures/fig_031}
    \begin{equation*}
      \frac{1}{R_{eq}} = \frac{1}{r_1} + \frac{1}{r_2} + \frac{1}{r_3}
    \end{equation*}
    \caption{Resistors in Parallel}
    \label{fig:031}
  \end{subfigure}
\end{figure}

When resistors are connected in series like in Figure \ref{fig:030}, the resistance over
all the resistors can be added directly. When connected in parallel such as in Figure
\ref{fig:031}, the resistance is added as reciprocal.

Notably, when resistors are connected in parallel, the equivalent resistance for the
circuit at large is less than any of the individual resistors. If $R_1=1 \Omega$, $R_2=2 \Omega$, 
and $R_3=3 \Omega$, then:
\begin{align*}
  \frac{1}{R_{eq}} &= \frac{1}{R_1} + \frac{1}{R_2} + \frac{1}{R_3} = \frac{1}{1} + \frac{1}{2} + \frac{1}{3} = \frac{11}{6} \\
  R_{eq}           &= \frac{6}{11}
\end{align*}

\subsection{Measuring Circuits}
\label{ssec:measuringCircuits}

\begin{definition}{Ammeter}
  A device used to measure the current flowing throughout a circuit at a specific point.
\end{definition}

When using an ammeter, it is imperative to connect it \textbf{in series} with the circuit.
\begin{figure}[H]
  \centering
  \includestandalone{figures/fig_032}
\end{figure}
This is because current is conserved throughout a singular path. However, if you were to
provide the current with an alternative path (ammeter connected in parallel), the current
can now travel along either path, making it inaccurate to just measure the current in a
single path.

\begin{definition}{Voltmeter}
  A device used to measure the voltage drop between two points of a circuit.
\end{definition}

A voltmeter must be connected \textbf{in parallel} with the circuit.
\begin{figure}[H]
  \centering
  \includestandalone{figures/fig_033}
\end{figure}
This is because voltage is conserved across branches of a parallel junction. If connected
in series, the voltmeter would just measure the voltage across itself.

\subsection{Kirchoff's Laws}
\label{ssec:kirchoffsLaws}

\begin{definition}{Kirchoff's First Law}
  The current flowing into a junction must be equal to the current flowing out of the
  junction. In other words, current is conserved throughout a circuit.
\end{definition}

\begin{wrapfigure}[7]{r}{0.3\textwidth}
  \centering
  \includestandalone{figures/fig_027}
  \caption{}
  \label{fig:0272}
\end{wrapfigure}
By selecting a junction, such as the one in Figure \ref{fig:0272}, Kirchoff's First Law
states that the total current flowing into the junction will be the same as the current flowing
out of it. Thus:
\begin{equation*}
  \overrightarrow{I}_1 + \overrightarrow{I}_2 + \overrightarrow{I}_3 = 0 \ \ \ \textup{or}\ \ \ |I_1| = |I_2| + |I_3|
\end{equation*}

\begin{definition}{Kirchoff's Second Law}
  Around any closed loop of a circuit, the total voltage drop is zero.
\end{definition}

\begin{wrapfigure}[7]{l}{0.4\textwidth}
  \centering
  \includestandalone{figures/fig_034}
  \caption{}
  \label{fig:034}
\end{wrapfigure}
By isolating a full loop within a circuit, Kirchoff's Second Law states that the total
voltage drop throughout the chosen loop will be zero. In Figure \ref{fig:034}, the closed
loop denoted by the orange box will have a total voltage drop of zero. This means that
the voltage gained as particles pass through the battery will be the same voltage lost as
particles travel over the two resistors.

\newpage
\subsection{Capacitors in Circuits}
\label{ssec:capacitorsInCircuits}

\begin{wrapfigure}[4]{l}{0.2\textwidth}
  \vspace{-20pt}
  \centering
  \includestandalone{figures/fig_035}
\end{wrapfigure}
In a circuit of a capacitor and battery, the battery will charge the capacitor for a period
of time until a charge of $Q=CV$ has built up between the plates of the capacitor. During
this period of charging, there is a current throughout the circuit.
\vspace{12pt}

\begin{wrapfigure}[5]{r}{0.2\textwidth}
  \vspace{-20pt}
  \centering
  \includestandalone{figures/fig_036}
\end{wrapfigure}
Now consider a fully charged capacitor in a circuit with a resistor and a switch. The switch
currently is open, meaning that no current is moving throughout the circuit. In this situation,
there is a voltage drop across the capacitor: $V_{i}=\frac{Q_{i}}{C}$. At the moment the
switch closes, a current begins to flow.
\begin{figure}[H]
  \vspace{-20pt}
  \centering
  \includestandalone{figures/fig_037}
  \vspace{-20pt}
\end{figure}
At the instant the switch closes and the current begins to flow, the current will be
$I_{i}=\frac{V_{i}}{R}$. However, there is only so much charge in the capacitor; supplying
the current will cause it to run out. Since both $V=IR$ and $V=\frac{Q}{C}$ are true, they can be
combined into $IR=\frac{Q}{C}$ or $I=\frac{Q}{RC}$.

$I$, being the current in the circuit, conceptually is just the amount of charge leaving
the capacitor, it can be expressed as $-\frac{dQ}{dt}$. Thus:

\begin{formula}{Current as a Result of a Capacitor}
  \begin{equation*}
    -I = \frac{dQ}{dt} = -\frac{Q}{RC}
  \end{equation*}
\end{formula}
The amount of current in the circuit is a function of the amount of charge left in the
capacitor. Since the amount of charge left in the capacitor over time is dependent of the
amount of current flowing, the amount of charge in the capacitor decays exponentially.
\begin{formula}{Charge in a Capacitor as a Function of Time}
  \begin{equation*}
    Q(t) = Q_{i}e^{-\frac{t}{RC}}
  \end{equation*}
  Every $RC$ seconds, $Q$ decreases by a factor of $\frac{1}{e}$. Since the rate at
  which $Q$ decreases is determined by $RC$, it is considered the
  \textbf{time constant}: $\tau$
  \begin{equation*}
    Q(t) = Q_{i}e^{-\frac{t}{\tau}}
  \end{equation*}
\end{formula}

\end{document}
