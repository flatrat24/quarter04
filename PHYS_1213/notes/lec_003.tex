\documentclass[12pt]{article}

%%%% GRAPHICS %%%%
\usepackage{tikz}
\usepackage{circuitikz}
\usetikzlibrary{arrows.meta}
\usepackage{tikz-3dplot}
\usepackage{graphicx}
\usepackage{pgfplots}
  \pgfplotsset{compat=1.18}
\usetikzlibrary{arrows}
\newcommand{\midarrow}{\tikz \draw[-triangle 90] (0,0) -- +(.1,0);}

%%%% FIGURES %%%%
\usepackage{subcaption}
\usepackage{wrapfig}
\usepackage{float}
\usepackage[skip=5pt, font=footnotesize]{caption}

%%%% FORMATTING %%%%
\usepackage{parskip}
\usepackage{tcolorbox}
\usepackage{ulem}
% \usepackage{fancyhdr}

%%%% TABLE FORMATTING %%%%
\usepackage{tabularray}
\UseTblrLibrary{booktabs}

%%%% MATH AND LOGIC %%%%
\usepackage{xifthen}
\usepackage{amsmath}
\usepackage{amssymb}
\usepackage{amsfonts}

%%%% TEXT AND SYMBOLS %%%%
\usepackage[T1]{fontenc}
\usepackage{textcomp}
\usepackage{gensymb}

%%%% OTHER %%%%
\usepackage{standalone}

%%%% LOGIC SYMBOLS %%%%
\newcommand*\xor{\oplus}

%%%% STYLES %%%%

% Packages
\usepackage{fullpage}
\usepackage{titlesec}
\usepackage[rgb]{xcolor}
\selectcolormodel{natural}
\usepackage{ninecolors}
\selectcolormodel{rgb}

% Colors
\definecolor{pg}{HTML}{24273A}
\definecolor{fg}{HTML}{FFFFFF}
\definecolor{bg}{HTML}{24273A}
\definecolor{re}{HTML}{d20f39}
\definecolor{gr}{HTML}{40a02b}
\definecolor{ye}{HTML}{df8e1d}
\definecolor{or}{HTML}{fe640b}
\definecolor{bl}{HTML}{1e66f5}
\definecolor{ma}{HTML}{8839ef}
\definecolor{cy}{HTML}{179299}
\definecolor{pi}{HTML}{ea76cb}

\usepackage{nameref}
\makeatletter
\newcommand*{\currentname}{\@currentlabelname}
\makeatother

\titleformat{\section}
  {\normalfont\scshape\Large\bfseries}
  {\thesection}
  {0.75em}
  {}

\titleformat{\subsection}
  {\normalfont\scshape\large\bfseries}
  {\thesubsection}
  {0.75em}
  {}

\titleformat{\subsubsection}
  {\normalfont\scshape\normalsize\bfseries}
  {\thesubsubsection}
  {0.75em}
  {}

% Formula
\newcounter{formula}[section]
\newenvironment{formula}[1]{
  \stepcounter{formula}
  \begin{tcolorbox}[
    standard jigsaw, % Allows opacity
    colframe={fg},
    boxrule=1px,
    colback=bg,
    opacityback=0,
    sharp corners,
    sidebyside,
    righthand width=18px,
    coltext={fg}
  ]
  \centering
  \textbf{\uline{#1}}
}{
  \tcblower
  \textbf{\thesection.\theformula}
  \end{tcolorbox}
}

% Definition
\newcounter{definition}[section]

\newenvironment{definition*}[1]{
  \begin{tcolorbox}[
    standard jigsaw, % Allows opacity
    colframe={fg},
    boxrule=1px,
    colback=bg,
    opacityback=0,
    sharp corners,
    coltext={fg}
  ]
  \textbf{#1 \hfill}
  \vspace{5px}
  \hrule
  \vspace{5px}
  \noindent
}{
  \end{tcolorbox}
}

\newenvironment{definition}[1]{
  \stepcounter{definition}
  \begin{tcolorbox}[
    standard jigsaw, % Allows opacity
    colframe={fg},
    boxrule=1px,
    colback=bg,
    opacityback=0,
    sharp corners,
    coltext={fg}
  ]
  \textbf{#1 \hfill \thesection.\thedefinition}
  \vspace{5px}
  \hrule
  \vspace{5px}
  \noindent
}{
  \end{tcolorbox}
}

% Example Problem
\newcounter{example}[section]
\newenvironment{example}{
  \stepcounter{example}
  \begin{tcolorbox}[
    standard jigsaw, % Allows opacity
    colframe={fg},
    boxrule=1px,
    colback=bg,
    opacityback=0,
    sharp corners,
    coltext={fg}
  ]
  \textbf{Example \hfill \thesection.\theexample}
  \vspace{5px}
  \hrule
  \vspace{5px}
  \noindent
}{
  \end{tcolorbox}
}

\tikzset{
  cubeBorder/.style=fg,
  cubeFilling/.style={fg!20!bg, opacity=0.25},
  gridLine/.style={very thin, gray},
  graphLine/.style={-latex, thick, fg},
}

\pgfplotsset{
  basicAxis/.style={
    grid,
    major grid style={line width=.2pt,draw=fg!50!bg},
    axis lines = box,
    axis line style = {line width = 1px},
  }
}

%%%% REFERENCES %%%%
\usepackage{hyperref}
\hypersetup{
  colorlinks  = true,
  linkcolor   = pi,
  anchorcolor = pi,
  citecolor   = pi,
  filecolor   = pi,
  menucolor   = pi,
  runcolor    = pi,
  urlcolor    = pi,
}

\author{Ethan Anthony}


\title{Lecture 003}
\date{September 18, 2024}

\begin{document}

\newpage
\section{Gauss's Law}
\label{sec:gausssLaw}

How can the charges be determined only knowing what electric field is present?

\begin{definition}{Electric Flux}
  \begin{wrapfigure}[]{r}{0.2\textwidth}
    \centering
    \includestandalone{figures/fig_012}
    \caption{Flux}
    \label{fig:012}
  \end{wrapfigure}

  Flux is a mathematical way of describing how much "flow" is going through an area.
  Electric flux refers to the amount of \textbf{electric field} that is going through an
  area. The value of electric charge depends on:
  \begin{enumerate}
    \itemsep0em
    \item The \textbf{strength} of the electric field
    \item The \textbf{area} through which the electric field acts on
    \item The \textbf{angle} between the field and the area; the direction of the electric field
      that acts upon the area
  \end{enumerate}

  \begin{equation*}
    \Phi = E \cdot A = EAcos(\theta)
  \end{equation*}
\end{definition}

\begin{wrapfigure}[5]{l}{0.4\textwidth}
  \centering
  \includestandalone{figures/fig_013}
  \caption{Flux}
  \label{fig:013}
\end{wrapfigure}

An electric field going through an object will result in a net zero electric flux. This is
because electric flux can be measured in \textbf{inward} and \textbf{outward} electric flux.

\begin{formula}{True Electric Flux}
  \begin{equation*}
    \Phi = \oint_{}^{} \overrightarrow{E} \cdot d \overrightarrow{A} \,
  \end{equation*}
\end{formula}

\begin{formula}{Gauss's Law}
  \begin{equation*}
    \Phi = \oint_{}^{} \overrightarrow{E} \cdot d \overrightarrow{A} \, = \frac{q_{enclosed}}{\in_0}
  \end{equation*}
\end{formula}

This shows that the total electric flux ($\Phi$) through an object is determined by
the amount of charge enclosed within that object divided by a constant of nature ($\epsilon_0$).

Let's say we have an infinitely long wire of charge. The charge per unit length is $\lambda$.
What is the electric field at some point at a distance $r$ from the wire?

\begin{figure}[H]
  \centering
  \includestandalone{figures/fig_014}
\end{figure}

Continuous charge distributions often have a uniform \textbf{charge density}.
\begin{center}
  $\lambda$ = "lambda" = linear charge density (charge per unit length; $\frac{C}{m}$)

  $\sigma$ = "sigma" = surface charge density (charge per unit area; $\frac{C}{m^2}$)

  $\rho$ = "rho" = volume charge density (charge per unit volume; $\frac{C}{m^3}$)
\end{center}

To solve this, a Gaussian Surface equidistant from the wire at all points would be easiest
to use; a cylinder wrapped around the wire would be simplest.

\begin{figure}[H]
  \centering
  \includestandalone{figures/fig_015}
\end{figure}

With this Gaussian Surface defined, $q_{\textup{enclosed}}$ is the length of wire enclosed
by the cylinder multiplied by the charge per unit length: $q_{\textup{enclosed}} = \lambda L$.

Since the surface is perpendicular to the Electric Field at every point, the dot product of
the electric flux and the area is just simple multiplication.
\begin{align*}
  \oint_{}^{} \overrightarrow{E} \cdot d \overrightarrow{A} \, &= \frac{q_{\textup{enclosed}}}{\epsilon_0} \\
  E_{\bot}A &= \frac{q_{\textup{enclosed}}}{\epsilon_0} \\
  E(2\pi rL) &= \frac{\lambda L}{\epsilon_0} \\
  E &= \frac{\frac{\lambda L}{\epsilon_0}}{2\pi rL} \\
  E &= \frac{\lambda L}{2\pi rL\epsilon_0} \\
  E &= \frac{\lambda}{2\pi r\epsilon_0}
\end{align*}
Thus, the electric field near a long wire of charge is modeled as $E = \frac{\lambda}{2\pi r\epsilon_0}$.
This is an approximation, but still very accurate when the distance $r$ is small in comparison
to the length of the wire.


\end{document}
