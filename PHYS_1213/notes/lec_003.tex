\documentclass[12pt]{article}

\input{../../xlatex/imports/preamble}

\title{Lecture 003}
\date{September 18, 2024}

\begin{document}

\newpage
\section{Gauss's Law}
\label{sec:gausssLaw}

How can the charges be determined only knowing what electric field is present?

\begin{definition}{Electric Flux}
  \begin{wrapfigure}[]{r}{0.2\textwidth}
    \centering
    \includestandalone{figures/fig_012}
    \caption{Flux}
    \label{fig:012}
  \end{wrapfigure}

  Flux is a mathematical way of describing how much "flow" is going through an area.
  Electric flux refers to the amount of \textbf{electric field} that is going through an
  area. The value of electric charge depends on:
  \begin{enumerate}
    \itemsep0em
    \item The \textbf{strength} of the electric field
    \item The \textbf{area} through which the electric field acts on
    \item The \textbf{angle} between the field and the area; the direction of the electric field
      that acts upon the area
  \end{enumerate}

  \begin{equation*}
    \Phi = E \cdot A = EAcos(\theta)
  \end{equation*}
\end{definition}

\begin{wrapfigure}[5]{l}{0.4\textwidth}
  \centering
  \includestandalone{figures/fig_013}
  \caption{Flux}
  \label{fig:013}
\end{wrapfigure}

An electric field going through an object will result in a net zero electric flux. This is
because electric flux can be measured in \textbf{inward} and \textbf{outward} electric flux.

\begin{formula}{True Electric Flux}
  \begin{equation*}
    \Phi = \oint_{}^{} \overrightarrow{E} \cdot d \overrightarrow{A} \,
  \end{equation*}
\end{formula}

\begin{formula}{Gauss's Law}
  \begin{equation*}
    \Phi = \oint_{}^{} \overrightarrow{E} \cdot d \overrightarrow{A} \, = \frac{q_{enclosed}}{\in_0}
  \end{equation*}
\end{formula}

This shows that the total electric flux ($\Phi$) through an object is determined by
the amount of charge enclosed within that object divided by a constant of nature ($\epsilon_0$).

Let's say we have an infinitely long wire of charge. The charge per unit length is $\lambda$.
What is the electric field at some point at a distance $r$ from the wire?

\begin{figure}[H]
  \centering
  \includestandalone{figures/fig_014}
\end{figure}

Continuous charge distributions often have a uniform \textbf{charge density}.
\begin{center}
  $\lambda$ = "lambda" = linear charge density (charge per unit length; $\frac{C}{m}$)

  $\sigma$ = "sigma" = surface charge density (charge per unit area; $\frac{C}{m^2}$)

  $\rho$ = "rho" = volume charge density (charge per unit volume; $\frac{C}{m^3}$)
\end{center}

To solve this, a Gaussian Surface equidistant from the wire at all points would be easiest
to use; a cylinder wrapped around the wire would be simplest.

\begin{figure}[H]
  \centering
  \includestandalone{figures/fig_015}
\end{figure}

With this Gaussian Surface defined, $q_{\textup{enclosed}}$ is the length of wire enclosed
by the cylinder multiplied by the charge per unit length: $q_{\textup{enclosed}} = \lambda L$.

Since the surface is perpendicular to the Electric Field at every point, the dot product of
the electric flux and the area is just simple multiplication.
\begin{align*}
  \oint_{}^{} \overrightarrow{E} \cdot d \overrightarrow{A} \, &= \frac{q_{\textup{enclosed}}}{\epsilon_0} \\
  E_{\bot}A &= \frac{q_{\textup{enclosed}}}{\epsilon_0} \\
  E(2\pi rL) &= \frac{\lambda L}{\epsilon_0} \\
  E &= \frac{\frac{\lambda L}{\epsilon_0}}{2\pi rL} \\
  E &= \frac{\lambda L}{2\pi rL\epsilon_0} \\
  E &= \frac{\lambda}{2\pi r\epsilon_0}
\end{align*}
Thus, the electric field near a long wire of charge is modeled as $E = \frac{\lambda}{2\pi r\epsilon_0}$.
This is an approximation, but still very accurate when the distance $r$ is small in comparison
to the length of the wire.


\end{document}
