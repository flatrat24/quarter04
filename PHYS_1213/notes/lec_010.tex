\documentclass[12pt]{article}

\input{../../xlatex/imports/preamble}

\title{Lecture 010}
\date{November 07, 2024}

\begin{document}

\newpage
\section{Electromagnetism}
\label{sec:electromagnetism}

Another way to relate electricity to magnetism is that changing magnetic fields
\textbf{create} EMF. In other ways, a time-varying magnetic field can make current flow.
In Figure \ref{fig:044}, as the magnet moves away from the circuit, a current is created
causing the led to light up.

\begin{figure}[H]
  \centering
  \includestandalone{figures/fig_044}
  \caption{Magnetic Field Creating Current}
  \label{fig:044}
\end{figure}

\subsection{Faraday's law}
\label{ssec:faradaysLaw}

This behavior follows \textbf{Faraday's Law} which models the relationship between the
EMF created and the change in the magnetic field.

\begin{formula}{Faraday's Law}

  \vspace{12pt}
  The change in voltage ($EMF$) is equal to the rate of change of the magnetic flux
  ($\frac{d \Phi}{dt}$).
  \begin{equation*}
    EMF = -\frac{d \Phi}{dt}
  \end{equation*}
\end{formula}

The EMF created is similar to the voltage provided by a battery in a circuit. However, 
EMF is distributed evenly throughout a circuit. To calculate it integrate an electric
field around a loop:
\begin{equation*}
  EMF = \oint_{}^{} \overline{E} \cdot \, d \overline{L}
\end{equation*}

Faraday's Law scales with the number of loops being considered. If $N$ loops were to be
stacked on top of each other, the formula to calculate EMF would become:
\begin{equation*}
  EMF = -N \frac{d \Phi}{dt}
\end{equation*}

\subsection{Magnetic Flux}
\label{ssec:magneticFlux}

If $EMF$ is created by a change in magnetic flux, then what are the ways magnetic flux
can change? If $\Phi_B = BA\cos(\theta)$, then $\Phi_B$ can change if:
\begin{itemize}
  \itemsep0em
  \item There is a change in the \textbf{strength of the B field}
  \item There is a change in the \textbf{area of the loop}
  \item There is a change in the \textbf{angle between the field and the loop}
\end{itemize}

\subsection{Lenz's Law}
\label{ssec:lenzsLaw}

Why is there a negative sign in Faraday's Law? It indicates the direction of the induced
$EMF$ by the change in magnetic field.

\begin{definition}{Lenz's Law}
  The direction of the induced current's magnetic field will oppose the \textbf{change}
  in the original B field (not the B field itself, just the change).
\end{definition}

Remember that the direction of a magnetic field created by a current is determined by
the right hand rule and $I \times r$.

\begin{figure}[H]
  \centering
  \includestandalone{figures/fig_045}
  \caption{Moving a Bar in a Circuit}
  \label{fig:045}
\end{figure}

In Figure \ref{fig:045}, there is a bar completing a circuit. As the bar is moved to the
left ($v$), the magnetic flux is \textbf{increasing}. In other words, the amount of magnetic
field enclosed within the loop is increasing \textbf{into the page}, meaning that the
current induced will create a magnetic field pointed \textbf{out of the page}, thus the
current flows counterclockwise.

Because there is now a current flowing, the magnetic field will exert a force
($F = IL \times B$) on the wire. This is the force ($F$) opposing the movement of the bar.
The force caused by this related to the \textbf{eddy current}.

\begin{definition}{Eddy Current}
  The induced current in a loop when the size of the loop changes, thus changing the
  magnetic flux in the system. This current will always create a force opposing the
  direction of the movement.
\end{definition}

\begin{figure}[H]
  \centering
  \includestandalone{figures/fig_046}
  \caption{EMF Created}
  \label{fig:046}
\end{figure}

When determining the amount of $EMF$ created by the setup in Figure \ref{fig:046},
Faraday's Law can be used: $EMF = - \frac{d \Phi_B}{dt}$. The amount of magnetic flux is
modeled as $xLB$. Since $x$ is changing at a rate of $v$ while $B$ and $L$ remain constant,
the rate of change of $\Phi_B$ is $BLv$. Thus: $EMF = BLv$.

If the bar has a resistance of $R$, then the power ($P = \frac{V^2}{R}$) dissipated over
the bar can be expressed as:
\begin{equation*}
  P = \frac{(BLv)^2}{R}
\end{equation*}
Similarly, the amount of force being applied to the bar ($ILB$) multiplied with the
distance/time ($v$) the bar is moving is the amount of power being applied to the bar.
\begin{align*}
  P &= (ILB)(v) \\
  P &= (\frac{BLv}{R}LB)(v) \\
  P &= \frac{(BLv)^2}{R}
\end{align*}
Evidently, the power dissipated over the resistor is the same as the power applied to the
bar. Thus, the system conserves energy.

\end{document}
