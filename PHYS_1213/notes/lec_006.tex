\documentclass[12pt]{article}

%%%% GRAPHICS %%%%
\usepackage{tikz}
\usepackage{circuitikz}
\usetikzlibrary{arrows.meta}
\usepackage{tikz-3dplot}
\usepackage{graphicx}
\usepackage{pgfplots}
  \pgfplotsset{compat=1.18}
\usetikzlibrary{arrows}
\newcommand{\midarrow}{\tikz \draw[-triangle 90] (0,0) -- +(.1,0);}

%%%% FIGURES %%%%
\usepackage{subcaption}
\usepackage{wrapfig}
\usepackage{float}
\usepackage[skip=5pt, font=footnotesize]{caption}

%%%% FORMATTING %%%%
\usepackage{parskip}
\usepackage{tcolorbox}
\usepackage{ulem}
% \usepackage{fancyhdr}

%%%% TABLE FORMATTING %%%%
\usepackage{tabularray}
\UseTblrLibrary{booktabs}

%%%% MATH AND LOGIC %%%%
\usepackage{xifthen}
\usepackage{amsmath}
\usepackage{amssymb}
\usepackage{amsfonts}

%%%% TEXT AND SYMBOLS %%%%
\usepackage[T1]{fontenc}
\usepackage{textcomp}
\usepackage{gensymb}

%%%% OTHER %%%%
\usepackage{standalone}

%%%% LOGIC SYMBOLS %%%%
\newcommand*\xor{\oplus}

%%%% STYLES %%%%

% Packages
\usepackage{fullpage}
\usepackage{titlesec}
\usepackage[rgb]{xcolor}
\selectcolormodel{natural}
\usepackage{ninecolors}
\selectcolormodel{rgb}

% Colors
\definecolor{pg}{HTML}{24273A}
\definecolor{fg}{HTML}{FFFFFF}
\definecolor{bg}{HTML}{24273A}
\definecolor{re}{HTML}{d20f39}
\definecolor{gr}{HTML}{40a02b}
\definecolor{ye}{HTML}{df8e1d}
\definecolor{or}{HTML}{fe640b}
\definecolor{bl}{HTML}{1e66f5}
\definecolor{ma}{HTML}{8839ef}
\definecolor{cy}{HTML}{179299}
\definecolor{pi}{HTML}{ea76cb}

\usepackage{nameref}
\makeatletter
\newcommand*{\currentname}{\@currentlabelname}
\makeatother

\titleformat{\section}
  {\normalfont\scshape\Large\bfseries}
  {\thesection}
  {0.75em}
  {}

\titleformat{\subsection}
  {\normalfont\scshape\large\bfseries}
  {\thesubsection}
  {0.75em}
  {}

\titleformat{\subsubsection}
  {\normalfont\scshape\normalsize\bfseries}
  {\thesubsubsection}
  {0.75em}
  {}

% Formula
\newcounter{formula}[section]
\newenvironment{formula}[1]{
  \stepcounter{formula}
  \begin{tcolorbox}[
    standard jigsaw, % Allows opacity
    colframe={fg},
    boxrule=1px,
    colback=bg,
    opacityback=0,
    sharp corners,
    sidebyside,
    righthand width=18px,
    coltext={fg}
  ]
  \centering
  \textbf{\uline{#1}}
}{
  \tcblower
  \textbf{\thesection.\theformula}
  \end{tcolorbox}
}

% Definition
\newcounter{definition}[section]

\newenvironment{definition*}[1]{
  \begin{tcolorbox}[
    standard jigsaw, % Allows opacity
    colframe={fg},
    boxrule=1px,
    colback=bg,
    opacityback=0,
    sharp corners,
    coltext={fg}
  ]
  \textbf{#1 \hfill}
  \vspace{5px}
  \hrule
  \vspace{5px}
  \noindent
}{
  \end{tcolorbox}
}

\newenvironment{definition}[1]{
  \stepcounter{definition}
  \begin{tcolorbox}[
    standard jigsaw, % Allows opacity
    colframe={fg},
    boxrule=1px,
    colback=bg,
    opacityback=0,
    sharp corners,
    coltext={fg}
  ]
  \textbf{#1 \hfill \thesection.\thedefinition}
  \vspace{5px}
  \hrule
  \vspace{5px}
  \noindent
}{
  \end{tcolorbox}
}

% Example Problem
\newcounter{example}[section]
\newenvironment{example}{
  \stepcounter{example}
  \begin{tcolorbox}[
    standard jigsaw, % Allows opacity
    colframe={fg},
    boxrule=1px,
    colback=bg,
    opacityback=0,
    sharp corners,
    coltext={fg}
  ]
  \textbf{Example \hfill \thesection.\theexample}
  \vspace{5px}
  \hrule
  \vspace{5px}
  \noindent
}{
  \end{tcolorbox}
}

\tikzset{
  cubeBorder/.style=fg,
  cubeFilling/.style={fg!20!bg, opacity=0.25},
  gridLine/.style={very thin, gray},
  graphLine/.style={-latex, thick, fg},
}

\pgfplotsset{
  basicAxis/.style={
    grid,
    major grid style={line width=.2pt,draw=fg!50!bg},
    axis lines = box,
    axis line style = {line width = 1px},
  }
}

%%%% REFERENCES %%%%
\usepackage{hyperref}
\hypersetup{
  colorlinks  = true,
  linkcolor   = pi,
  anchorcolor = pi,
  citecolor   = pi,
  filecolor   = pi,
  menucolor   = pi,
  runcolor    = pi,
  urlcolor    = pi,
}

\author{Ethan Anthony}


\title{Lecture 006}
\date{October 02, 2024}

\begin{document}

\newpage
\section{Capacitors}
\label{sec:capacitor}

\begin{definition}{Capacitor}
  Circuit components that are used to store charge.
\end{definition}

The way capacitors are charged is by moving charge from one part of the capacitor to the other
side of the capacitor, leaving one side positively charged and the other negatively charged.
This process stores energy.

\begin{figure}[h]
  \centering
  \includestandalone{figures/fig_023}
  \caption{Charging a Capacitor}
  \label{fig:023}
\end{figure}

This change in charge between the two plates creates a electric field between them. Doubling
the charge of the capacitor will double the electric field.
\begin{formula}{Capacitance}
  \begin{align*}
    C &= \frac{Q}{\Delta V}
  \end{align*}
  Capacitance is measured in the following units:
  \begin{align*}
    \frac{C}{V} = \textup{Farad} = F
  \end{align*}
\end{formula}
The constant being multiplied by voltage is called the \textbf{capacitance}, which is the
measurement of a capacitor's capability to hold charge. Capacitance depends on the shape
and material of a capacitor.
\begin{formula}{Capacitance for Parallel Plates}
  \begin{figure}[H]
    \begin{subfigure}[H]{0.35\textwidth}
      \centering
      \begin{align*}
        E = \frac{\frac{Q}{A}}{\epsilon_0}\ &\textup{;}\ \Delta V = Ed \\
        \Delta V &= \frac{Qd}{\epsilon_0A} \\
        C &= \frac{\epsilon_0A}{d} \\
      \end{align*}
    \end{subfigure}
    \begin{subfigure}[H]{0.6\textwidth}
      As distance increases, the capacitance increases. This is because each bit of charge
      has less voltage drop it needs to go through to reach the other side.

      As area increases, so does capacitance because more area means there is more room to
      store charge.
    \end{subfigure}
  \end{figure}
\end{formula}

When you connect a capacitor to a battery, that battery maintains a potential difference
between the two plates. By doing this, the battery pushed charges from one side to the other,
thus performing \textbf{work}.

\begin{figure}[H]
  \centering
  \includestandalone{figures/fig_024}
  \caption{Battery Connected to a Capacitor}
  \label{fig:024}
\end{figure}

The energy required to charge this capacitor with the battery:
\begin{formula}{Energy to Charge a Capacitor}
  \begin{figure}[H]
    \begin{subfigure}[H]{0.3\textwidth}
      \begin{equation*}
        U = \frac{Q \Delta V}{2}
      \end{equation*}
    \end{subfigure}
    \begin{subfigure}[H]{0.3\textwidth}
      \begin{equation*}
        U = \frac{Q^2}{2C}
      \end{equation*}
    \end{subfigure}
    \begin{subfigure}[H]{0.3\textwidth}
      \begin{equation*}
        U = \frac{C \Delta V^2}{2}
      \end{equation*}
    \end{subfigure}
  \end{figure}
\end{formula}

\begin{figure}[H]
  \centering
  % TODO
  \textbf{ADD IN SYMBOLS FOR CIRCUIT DIAGRAMS}
  \caption{Circuit Diagram with Capacitor and Battery}
  \label{fig:0025}
\end{figure}

Any time there is a circuit diagram, it's important to think about where the particles are
flowing through the circuit, gaining and losing energy.

When capacitors are connected in parallel, the total capacitance of the capacitors will
simply be:
\begin{equation*}
  C_{parallel} = C_1 + C_2 + C_3 + ...
\end{equation*}
This is because the voltage drop across each capacitor will be the same.

When connected in series, the total capacitance is:
\begin{align*}
  \Delta V_{total} &= \Delta V_1 + \Delta V_2 + \Delta V_3 + ... \\
  \Delta V_{total} &= \frac{Q}{C_1} + \frac{Q}{C_2} + \frac{Q}{C_3} + ... \\
  \Delta V_{total} &= Q\left(\frac{1}{C_1} + \frac{1}{C_2} + \frac{1}{C_3} + ...\right) \\
  \frac{\Delta V_{total}}{Q} &= \frac{1}{C_1} + \frac{1}{C_2} + \frac{1}{C_3} + ... \\
  \frac{1}{C_{series}} &= \frac{1}{C_1} + \frac{1}{C_2} + \frac{1}{C_3} + ...
\end{align*}
This is because the charge buildup on each capacitor will be the same. When charge flows
from the end of one capacitor to the beginning of the next, that charge has nowhere to go
but only the next capacitor.

\subsection{Dielectric}
\label{ssec:dielectric}

\begin{definition}{Dielectric}
  A material put between plates of capacitors to maintain the separation as well as modify
  the capacitance of the capacitor.
\end{definition}

Capacitors need to be incredibly close together for them to have significant amounts of
capacitance. Because two oppositely charged plates will have some pull between them, an
object called a \textbf{dielectric} is usually put between the plates to keep them apart.

\begin{formula}{Dielectric Capacitance}
  \begin{align*}
    C_{dielectric} = \kappa C = k \frac{Q}{\Delta V}
  \end{align*}
\end{formula}

\begin{figure}[H]
  \centering
  \includestandalone{figures/fig_025}
  \caption{Dielectric in a Capacitor}
  \label{fig:025}
\end{figure}

\end{document}
