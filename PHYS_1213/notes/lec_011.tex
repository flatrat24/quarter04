\documentclass[12pt]{article}

%%%% GRAPHICS %%%%
\usepackage{tikz}
\usepackage{circuitikz}
\usetikzlibrary{arrows.meta}
\usepackage{tikz-3dplot}
\usepackage{graphicx}
\usepackage{pgfplots}
  \pgfplotsset{compat=1.18}
\usetikzlibrary{arrows}
\newcommand{\midarrow}{\tikz \draw[-triangle 90] (0,0) -- +(.1,0);}

%%%% FIGURES %%%%
\usepackage{subcaption}
\usepackage{wrapfig}
\usepackage{float}
\usepackage[skip=5pt, font=footnotesize]{caption}

%%%% FORMATTING %%%%
\usepackage{parskip}
\usepackage{tcolorbox}
\usepackage{ulem}
% \usepackage{fancyhdr}

%%%% TABLE FORMATTING %%%%
\usepackage{tabularray}
\UseTblrLibrary{booktabs}

%%%% MATH AND LOGIC %%%%
\usepackage{xifthen}
\usepackage{amsmath}
\usepackage{amssymb}
\usepackage{amsfonts}

%%%% TEXT AND SYMBOLS %%%%
\usepackage[T1]{fontenc}
\usepackage{textcomp}
\usepackage{gensymb}

%%%% OTHER %%%%
\usepackage{standalone}

%%%% LOGIC SYMBOLS %%%%
\newcommand*\xor{\oplus}

%%%% STYLES %%%%

% Packages
\usepackage{fullpage}
\usepackage{titlesec}
\usepackage[rgb]{xcolor}
\selectcolormodel{natural}
\usepackage{ninecolors}
\selectcolormodel{rgb}

% Colors
\definecolor{pg}{HTML}{24273A}
\definecolor{fg}{HTML}{FFFFFF}
\definecolor{bg}{HTML}{24273A}
\definecolor{re}{HTML}{d20f39}
\definecolor{gr}{HTML}{40a02b}
\definecolor{ye}{HTML}{df8e1d}
\definecolor{or}{HTML}{fe640b}
\definecolor{bl}{HTML}{1e66f5}
\definecolor{ma}{HTML}{8839ef}
\definecolor{cy}{HTML}{179299}
\definecolor{pi}{HTML}{ea76cb}

\usepackage{nameref}
\makeatletter
\newcommand*{\currentname}{\@currentlabelname}
\makeatother

\titleformat{\section}
  {\normalfont\scshape\Large\bfseries}
  {\thesection}
  {0.75em}
  {}

\titleformat{\subsection}
  {\normalfont\scshape\large\bfseries}
  {\thesubsection}
  {0.75em}
  {}

\titleformat{\subsubsection}
  {\normalfont\scshape\normalsize\bfseries}
  {\thesubsubsection}
  {0.75em}
  {}

% Formula
\newcounter{formula}[section]
\newenvironment{formula}[1]{
  \stepcounter{formula}
  \begin{tcolorbox}[
    standard jigsaw, % Allows opacity
    colframe={fg},
    boxrule=1px,
    colback=bg,
    opacityback=0,
    sharp corners,
    sidebyside,
    righthand width=18px,
    coltext={fg}
  ]
  \centering
  \textbf{\uline{#1}}
}{
  \tcblower
  \textbf{\thesection.\theformula}
  \end{tcolorbox}
}

% Definition
\newcounter{definition}[section]

\newenvironment{definition*}[1]{
  \begin{tcolorbox}[
    standard jigsaw, % Allows opacity
    colframe={fg},
    boxrule=1px,
    colback=bg,
    opacityback=0,
    sharp corners,
    coltext={fg}
  ]
  \textbf{#1 \hfill}
  \vspace{5px}
  \hrule
  \vspace{5px}
  \noindent
}{
  \end{tcolorbox}
}

\newenvironment{definition}[1]{
  \stepcounter{definition}
  \begin{tcolorbox}[
    standard jigsaw, % Allows opacity
    colframe={fg},
    boxrule=1px,
    colback=bg,
    opacityback=0,
    sharp corners,
    coltext={fg}
  ]
  \textbf{#1 \hfill \thesection.\thedefinition}
  \vspace{5px}
  \hrule
  \vspace{5px}
  \noindent
}{
  \end{tcolorbox}
}

% Example Problem
\newcounter{example}[section]
\newenvironment{example}{
  \stepcounter{example}
  \begin{tcolorbox}[
    standard jigsaw, % Allows opacity
    colframe={fg},
    boxrule=1px,
    colback=bg,
    opacityback=0,
    sharp corners,
    coltext={fg}
  ]
  \textbf{Example \hfill \thesection.\theexample}
  \vspace{5px}
  \hrule
  \vspace{5px}
  \noindent
}{
  \end{tcolorbox}
}

\tikzset{
  cubeBorder/.style=fg,
  cubeFilling/.style={fg!20!bg, opacity=0.25},
  gridLine/.style={very thin, gray},
  graphLine/.style={-latex, thick, fg},
}

\pgfplotsset{
  basicAxis/.style={
    grid,
    major grid style={line width=.2pt,draw=fg!50!bg},
    axis lines = box,
    axis line style = {line width = 1px},
  }
}

%%%% REFERENCES %%%%
\usepackage{hyperref}
\hypersetup{
  colorlinks  = true,
  linkcolor   = pi,
  anchorcolor = pi,
  citecolor   = pi,
  filecolor   = pi,
  menucolor   = pi,
  runcolor    = pi,
  urlcolor    = pi,
}

\author{Ethan Anthony}


\title{Lecture 011}
\date{November 13, 2024}

\begin{document}

\newpage
\section{Electromagnetic Waves}
\label{sec:electromagneticWaves}

\subsection{Maxwell's Equations}
\label{ssec:maxwellsEquations}

Based on everything thusfar, four primary equations represent the entirety of all of
what is known about electricity, magnetism, and how they relate. These equations are
referred to as the \textbf{Maxwell Equations}, and are in Figure
\ref{fig:maxwellsEquations}.

\begin{figure}[H]
  \centering
  \begin{tblr}{rl}
    \toprule
    \textbf{Equation} & \textbf{Law} \\
    \midrule
    $\oint_{}^{} \overrightarrow{E} \cdot \,d \overrightarrow{A} = \frac{Q_{enclosed}}{\epsilon_0}$ & \textbf{Gauss's Law} - Charges create diverging electric fields \\
    $\oint_{}^{} \overrightarrow{B} \cdot \,d \overrightarrow{A} = 0$                               & \textbf{Gauss's (Magentism) Law} - No magnetic monopoles        \\
    $\oint_{}^{} \overrightarrow{E} \cdot \,d \overrightarrow{l} = -\frac{d \Phi_B}{dt}$            & \textbf{Faraday's Law} - Changing B fields make E fields        \\
    $\oint_{}^{} \overrightarrow{B} \cdot \,d \overrightarrow{l} = \mu_0I_{enclosed}$               & \textbf{Ampere's Law} - Currents make B fields                  \\
    \bottomrule
  \end{tblr}
  \caption{Maxwell's Equations}
  \label{fig:maxwellsEquations}
\end{figure}

However, \textbf{Ampere's Law} doesn't properly describe what is happening all of the time.
Considering a capacitor that is being charged up, there is no current passing over the
capacitor. However, there \textit{is} a B field at that spot.

\begin{figure}[H]
  \centering
  \includestandalone{figures/fig_047}
  \caption{Ampere's Law with a Capacitor}
  \label{fig:047}
\end{figure}

Based on the \textbf{Biot-Savart Law}, there should be a B field created at point $A$ in
Figure \ref{fig:047}. However, a loop drawn around the electric field that intersects $A$
would lead to an Ampere's Law with no enclosed current and thus no B field.

As it turns out, the true Ampere's Law that accounts for this effect is:
\begin{equation*}
  \oint_{}^{} \overrightarrow{B} \cdot \,d \overrightarrow{l} = \mu_0\left(I_{enclosed} + \epsilon_0\frac{d \Phi_E}{dt}\right)
\end{equation*}

Thus, the true Maxwell's Equations are in Figure \ref{fig:trueMaxwell'sEquations}

\begin{figure}[H]
  \centering
  \begin{tblr}{rl}
    \toprule
    \textbf{Equation} & \textbf{Law} \\
    \midrule
    \textbf{Gauss's Law}             & $\oint_{}^{} \overrightarrow{E} \cdot \,d \overrightarrow{A} = \frac{Q_{enclosed}}{\epsilon_0}$ \\
    \textbf{Gauss's (Magentism) Law} & $\oint_{}^{} \overrightarrow{B} \cdot \,d \overrightarrow{A} = 0$  \\
    \textbf{Faraday's Law}           & $\oint_{}^{} \overrightarrow{E} \cdot \,d \overrightarrow{l} = -\frac{d \Phi_B}{dt}$ \\
    \textbf{Ampere's Law}            & $\oint_{}^{} \overrightarrow{B} \cdot \,d \overrightarrow{l} = \mu_0\left(I_{enclosed} + \epsilon_0\frac{d \Phi_E}{dt}\right)$ \\
    \bottomrule
  \end{tblr}
  \caption{True Maxwell's Equations}
  \label{fig:trueMaxwell'sEquations}
\end{figure}

\subsection{Electromagnetic Waves}
\label{ssec:electromagneticWaves}

When a particle vibrates, its changing electric field will create a magnetic field.
However, since the changing electric field changes at different rates, the magnetic field
will also change, thus creating an electric field. This reciprocal behavior goes on
forever, and produces what is called an \textbf{electromagnetic wave}.

\begin{figure}[H]
  \centering
  \includestandalone{figures/fig_048}
  \caption{Electromagnetic Wave}
  \label{fig:048}
\end{figure}

Through vector calculus, Maxwell was able to show mathematically how quickly an
electromagnetic wave moves:
\begin{equation*}
  v = \frac{1}{\sqrt{\mu_0 \epsilon_0}} = 3 \times 10^8 \frac{m}{s}
\end{equation*}
Coincidentally (not really coincidentally), this is the speed of light!

\vspace{8pt}
\hrule

Waves are described by: $A \cos \left(\frac{2 \pi}{\lambda}x - \frac{2 \pi}{T}t\right)$,
where $\lambda$ is the wavelength of the wave, $T$ is the period of the wave, $x$ is the
position of the wave, and $t$ is time.

Similarly, this equation can be expressed as:
\begin{equation*}
  y(x,y) = A \cos \left(kx - \omega t\right)
\end{equation*}
Where $k$ is $\frac{2 \pi}{\lambda}$ and is referred to as the wavenumber (spatial frequency)
of the wave and $\omega$ is $\frac{2 \pi}{T}$ and is the angular frequency of the wave.

With electromagnetic waves, the same equation is true for both the E and B fields.
\begin{formula}{Electromagnetic Wave Equations}
  \begin{align*}
    y(x,t) &= A \cos(k x-\omega t) \\
    E      &= E_{max} \cos(kx-\omega t) \\
    B      &= B_{max} \cos(kx-\omega t)
  \end{align*}
\end{formula}

Additionally, the E and B waves are always in \textbf{phase} with eachother. This means
that they oscillate over the same period and reach their peaks and troughs at the same time.
They are also both \textbf{perpendicular} to the direction of travel. Both of these things
together mean that we can relate them linearly as:
\begin{equation*}
  E = cB
\end{equation*}

\subsection{Electromagnetic Energy}
\label{ssec:electromagneticEnergy}

Waves transport things. In the case of electromagnetic waves, only energy is transferred.
There is no material being transferred (partly because there is no medium required for an
electromagnetic wave to propagate over).

The energy density in a given E or B field is:
\begin{align*}
  \textup{Energy Density}_E &= \frac{1}{2}\epsilon_0E^2 \\
  \textup{Energy Density}_B &= \frac{1}{2 \mu_0}B^2 \\
\end{align*}
However, the actual power that exists over a given unit area of an EM wave is:
\begin{equation*}
  S = \frac{1}{\mu_0}EB
\end{equation*}
And expressed as a vector, is:
\begin{equation*}
  \overrightarrow{S} = \frac{1}{\mu_0} \overrightarrow{E} \times \overrightarrow{B}
\end{equation*}
Generally, it's more relevant to consider the \textbf{average} power/area of a wave rather
than the power/area at a single point in time. This quantity is called the
\textbf{Intensity} of the wave as is:
\begin{equation*}
  I = S_{avg} = \frac{1}{2\mu_0} E_{max}B_{max}
\end{equation*}
Knowing all of this, how can the intensity of a wave be related to the specific E and B
fields?
\begin{formula}{Maximum Intensities of E and B Fields}

  \vspace{5pt}
  The intensity of a wave was previously defined as the power per area of that wave. 
  Mathematically, this can be expressed as:
  \begin{equation*}
    I = \frac{P}{A}
  \end{equation*}
  Using the intensity of any given EM wave, the relationship that exists between $E_{max}$,
  $B_{max}$, and $I$ are:
  \begin{equation*}
    I = \frac{1}{2 \mu_0}cB^2_{max} = \frac{1}{2}\epsilon_0cE^2_{max}
  \end{equation*}
\end{formula}

\subsection{Electromagnetic Momentum}
\label{ssec:electromagneticMomentum}

Electromagnetic waves also have momentum. This is how solar sails in space work. The
pressure exerted on the sail from electromagnetic waves creates small bits of momentum
that compound over time. The mathematical expression for this pressure is:

\begin{formula}{Radiation Pressure}
  \begin{align*}
    P_{\textup{absorbed}}  &= \frac{I}{c} \\
    P_{\textup{reflected}} &= \frac{2I}{c}
  \end{align*}
\end{formula}

Note that this works just like regular momentum; when the particles are fully reflected,
the amount of pressure is doubled when compared to when the particles are fully
absorbed. To translate this into force:

\begin{formula}{Radiation Force}
  \begin{align*}
    F_{\textup{absorbed}}  &= \frac{I}{c} A = \frac{IA}{c} \\
    F_{\textup{reflected}} &= \frac{2I}{c} A = \frac{2IA}{c}
  \end{align*}
\end{formula}

\end{document}
