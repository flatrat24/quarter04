\documentclass[12pt]{article}

%%%% GRAPHICS %%%%
\usepackage{tikz}
\usepackage{circuitikz}
\usetikzlibrary{arrows.meta}
\usepackage{tikz-3dplot}
\usepackage{graphicx}
\usepackage{pgfplots}
  \pgfplotsset{compat=1.18}
\usetikzlibrary{arrows}
\newcommand{\midarrow}{\tikz \draw[-triangle 90] (0,0) -- +(.1,0);}

%%%% FIGURES %%%%
\usepackage{subcaption}
\usepackage{wrapfig}
\usepackage{float}
\usepackage[skip=5pt, font=footnotesize]{caption}

%%%% FORMATTING %%%%
\usepackage{parskip}
\usepackage{tcolorbox}
\usepackage{ulem}
% \usepackage{fancyhdr}

%%%% TABLE FORMATTING %%%%
\usepackage{tabularray}
\UseTblrLibrary{booktabs}

%%%% MATH AND LOGIC %%%%
\usepackage{xifthen}
\usepackage{amsmath}
\usepackage{amssymb}
\usepackage{amsfonts}

%%%% TEXT AND SYMBOLS %%%%
\usepackage[T1]{fontenc}
\usepackage{textcomp}
\usepackage{gensymb}

%%%% OTHER %%%%
\usepackage{standalone}

%%%% LOGIC SYMBOLS %%%%
\newcommand*\xor{\oplus}

%%%% STYLES %%%%

% Packages
\usepackage{fullpage}
\usepackage{titlesec}
\usepackage[rgb]{xcolor}
\selectcolormodel{natural}
\usepackage{ninecolors}
\selectcolormodel{rgb}

% Colors
\definecolor{pg}{HTML}{24273A}
\definecolor{fg}{HTML}{FFFFFF}
\definecolor{bg}{HTML}{24273A}
\definecolor{re}{HTML}{d20f39}
\definecolor{gr}{HTML}{40a02b}
\definecolor{ye}{HTML}{df8e1d}
\definecolor{or}{HTML}{fe640b}
\definecolor{bl}{HTML}{1e66f5}
\definecolor{ma}{HTML}{8839ef}
\definecolor{cy}{HTML}{179299}
\definecolor{pi}{HTML}{ea76cb}

\usepackage{nameref}
\makeatletter
\newcommand*{\currentname}{\@currentlabelname}
\makeatother

\titleformat{\section}
  {\normalfont\scshape\Large\bfseries}
  {\thesection}
  {0.75em}
  {}

\titleformat{\subsection}
  {\normalfont\scshape\large\bfseries}
  {\thesubsection}
  {0.75em}
  {}

\titleformat{\subsubsection}
  {\normalfont\scshape\normalsize\bfseries}
  {\thesubsubsection}
  {0.75em}
  {}

% Formula
\newcounter{formula}[section]
\newenvironment{formula}[1]{
  \stepcounter{formula}
  \begin{tcolorbox}[
    standard jigsaw, % Allows opacity
    colframe={fg},
    boxrule=1px,
    colback=bg,
    opacityback=0,
    sharp corners,
    sidebyside,
    righthand width=18px,
    coltext={fg}
  ]
  \centering
  \textbf{\uline{#1}}
}{
  \tcblower
  \textbf{\thesection.\theformula}
  \end{tcolorbox}
}

% Definition
\newcounter{definition}[section]

\newenvironment{definition*}[1]{
  \begin{tcolorbox}[
    standard jigsaw, % Allows opacity
    colframe={fg},
    boxrule=1px,
    colback=bg,
    opacityback=0,
    sharp corners,
    coltext={fg}
  ]
  \textbf{#1 \hfill}
  \vspace{5px}
  \hrule
  \vspace{5px}
  \noindent
}{
  \end{tcolorbox}
}

\newenvironment{definition}[1]{
  \stepcounter{definition}
  \begin{tcolorbox}[
    standard jigsaw, % Allows opacity
    colframe={fg},
    boxrule=1px,
    colback=bg,
    opacityback=0,
    sharp corners,
    coltext={fg}
  ]
  \textbf{#1 \hfill \thesection.\thedefinition}
  \vspace{5px}
  \hrule
  \vspace{5px}
  \noindent
}{
  \end{tcolorbox}
}

% Example Problem
\newcounter{example}[section]
\newenvironment{example}{
  \stepcounter{example}
  \begin{tcolorbox}[
    standard jigsaw, % Allows opacity
    colframe={fg},
    boxrule=1px,
    colback=bg,
    opacityback=0,
    sharp corners,
    coltext={fg}
  ]
  \textbf{Example \hfill \thesection.\theexample}
  \vspace{5px}
  \hrule
  \vspace{5px}
  \noindent
}{
  \end{tcolorbox}
}

\tikzset{
  cubeBorder/.style=fg,
  cubeFilling/.style={fg!20!bg, opacity=0.25},
  gridLine/.style={very thin, gray},
  graphLine/.style={-latex, thick, fg},
}

\pgfplotsset{
  basicAxis/.style={
    grid,
    major grid style={line width=.2pt,draw=fg!50!bg},
    axis lines = box,
    axis line style = {line width = 1px},
  }
}

%%%% REFERENCES %%%%
\usepackage{hyperref}
\hypersetup{
  colorlinks  = true,
  linkcolor   = pi,
  anchorcolor = pi,
  citecolor   = pi,
  filecolor   = pi,
  menucolor   = pi,
  runcolor    = pi,
  urlcolor    = pi,
}

\author{Ethan Anthony}


\title{Lecture 007}
\date{October 09, 2024}

\begin{document}
\newpage
\section{Current}
\label{sec:current}

\subsection{Current Definition and Formula}
\label{ssec:currentDefinitionAndFormula}

\begin{definition}{Current}
  Current describes how much charge moves past some spot over any given second.
\end{definition}

\begin{formula}{Current}
  \begin{align*}
    I &= \frac{\Delta Q}{\Delta t} = \frac{dQ}{dt} \\
    \textup{Ampere} &= \frac{\textup{Coulomb}}{\textup{Second}}
  \end{align*}
\end{formula}

Current has a direction. The direction of it is just convention, but is such that a \textbf{positive}
current corresponds to negative charges moving in the negative direction or positive charges
are moving in the negative direction.

Currents are created by \textbf{electric fields}, which can be conceptualized by a difference
in electric potential (voltage).

\begin{figure}[H]
  \centering
  \includestandalone{figures/fig_026}
  \caption{Current in a Wire}
  \label{fig:026}
\end{figure}

\subsection{Current at a Specific Point}
\label{ssec:currentAtASpecificPoint}

Inside a wire, each particle is moving \textit{on average} with some velocity in some direction.
This average \textbf{drift velocity} ($v_d$) of the particles is what determines the current of the
circuit.

\begin{formula}{Current by Volume}
  \begin{align*}
    \Delta L &= v_{d} \Delta t \\
    I &= nv_{d}Aq \\
    J &= \frac{I}{A} = nv_{d}q
  \end{align*}
  \begin{tblr}{ccccc}
    $I= $ current & $n=$ particle density & $v_d= $ velocity & $A= $ area & $q=$ charge \\
  \end{tblr}
\end{formula}

Different materials allow particles to move through them with varying ease. This property
is called the conductivity of the material.
\begin{definition}{Conductivity}
  Describes how easily a particle can move through a material. Denoted as $\sigma$.
\end{definition}
The conductivity of a material is empirically derived, but conceptually can be thought of
as the amount of obstacles in the way of a particles movement. The "denser" the path is,
the harder it is for a particle to move through the material.
\begin{formula}{Energy in Terms of Conductivity}
  \begin{align*}
    J &= \sigma E \\
    J \frac{1}{\sigma} &= E \\
    J \rho &= E
  \end{align*}
  \begin{tblr}{c|c}
    $J = \textup{current density} = \frac{A}{m^2}$ & $\sigma = \textup{conductivity} = \frac{A}{Vm}$ \\
    $E = \textup{electric field} = \frac{N}{C}$ & $\rho = \textup{resistivity} = \frac{Vm}{A}$ \\
  \end{tblr}
\end{formula}

\subsection{Ohm's Law}
\label{ssec:ohmsLaw}

\begin{formula}{Ohm's Law}
  \begin{equation*}
    J = \frac{E}{\rho} \ \ \Rightarrow\ \  \frac{I}{A} = \frac{\frac{\Delta V}{L}}{\rho} \ \ \Rightarrow\ \  \Delta V = \frac{\rho L}{A}I
  \end{equation*}
  \begin{tblr}{c|c}
    $I = \textup{current} = A$ & $A = \textup{area} = m^2$ \\
    $\Delta V = \textup{voltage drop} = V$ & $L = \textup{length} = m$ \\
  \end{tblr}
  \vspace{12pt}

  $\frac{\rho L}{A}$ are all constants, and are thus combined into a single constant: resistance ($R$).
  \begin{equation*}
    \Delta V = IR
  \end{equation*}
\end{formula}

Ohm's Law provides an alternative way to calculate energy within a circuit in terms of
voltage, current, and resistance.
Notably, Ohm's law is not a law of nature, but only applies to certain "ohmic" materials. In
terms of \textbf{resistance}, it is measured in Ohms ($\Omega$) with units in $\frac{V}{A} = \frac{Js}{C^2}$.

The amount of resistance is dependent on its resistivity/conductivity. However, that's not
the only thing it is related to. The size and shape of the conductor also play a role in
its resistance. Since:
\begin{equation*}
  R = \frac{\rho L}{A}
\end{equation*}
An increase in length corresponds to an \textit{increase} in resistivity. Conversely, an increase in
area of the cross-section of the conductor would correspond to a \textit{decrease} in
resistivity.

\begin{figure}[H]
  \centering
  \begin{tblr}{ccc}
    \toprule
    \textbf{Material} & \textbf{Resistivity} $\rho$ & \textbf{Conductivity} $\sigma$ \\
    \midrule
    Aluminum          & $2.8  \times 10^{-8}$       & $3.5  \times 10^{7}$            \\
    Silver            & $1.59 \times 10^{-8}$       & $6.30 \times 10^{7}$            \\
    Copper            & $1.68 \times 10^{-8}$       & $5.96 \times 10^{7}$            \\
    Gold              & $2.44 \times 10^{-8}$       & $4.10 \times 10^{7}$            \\
    Calcium           & $3.36 \times 10^{-8}$       & $2.98 \times 10^{7}$            \\
    Tungsten          & $5.60 \times 10^{-8}$       & $1.79 \times 10^{7}$            \\
    Zinc              & $5.90 \times 10^{-8}$       & $1.69 \times 10^{7}$            \\
    Nickel            & $6.99 \times 10^{-8}$       & $1.43 \times 10^{7}$            \\
    Iron              & $1.00 \times 10^{-8}$       & $1.00 \times 10^{7}$            \\
    \bottomrule
  \end{tblr}
  \caption{Resistivity and Conductivity of Common Material}
  \label{fig:resistivityConductivity}
\end{figure}

\begin{wrapfigure}[]{l}{0.2\textwidth}
  \centering
  \vspace{-20pt}
  \includestandalone{figures/fig_028}
  % \caption{Voltage Across a Resistor}
  \label{fig:028}
\end{wrapfigure}

As a current travels across a resistor, there is a voltage drop. This is seen in Ohm's Law,
where the voltage drop ($\Delta V$) is equal to the product of the current ($I$) and the
resistance ($R$).
\begin{equation*}
  \Delta V = IR
\end{equation*}
The larger the current or resistance in the resistor, the larger the voltage drop will be.
This voltage drop is a loss of energy within the system. So where does the energy go?

\subsection{Dissipated Heat}
\label{ssec:dissipatedHeat}

When a voltage drop occurs over a resistor, energy is lost. That energy transforms into \textbf{heat}. To measure
the heat power dissipated by a resistor:

\begin{formula}{Heat}
  \begin{equation*}
    \frac{J}{s} = IV = I^2R = \frac{V^2}{R}
  \end{equation*}
\end{formula}

Since the resistance of a material can only be reliably calculated in the ways outlined
within the scope of PHYS 1213 for Ohmic materials, $I^2R$ and $\frac{V^2}{R}$ only apply
to Ohmic materials. However, $IV$ applied universally since it is not based on resistance.

\subsection{Current Throughout a Circuit}
\label{ssec:currentThroughoutACircuit}

\begin{wrapfigure}[]{r}{0.3\textwidth}
  \vspace{-40pt}
  \includestandalone{figures/fig_027}
  \caption{Charge in a Circuit}
  \label{fig:027}
\end{wrapfigure}

Current is just a measurement of how the physical electrons/protons in a circuit are \textbf{moving}
about. The movement of these particles depends on the amount of energy they have (voltage).

Current throughout a circuit is conserved. In other words, at any point, the total amount
of current flowing into a location must be the same as the amount flowing out of that
location.

In Figure \ref{fig:027}, the amount of current flowing into junction $A$ is always going
to equal the amount of current flowing out. In other words: $\overrightarrow{I_1} = \overrightarrow{I_2} + \overrightarrow{I_3}$.

\begin{figure}[H]
  \centering
  \includestandalone{figures/fig_029}
  \caption{Circuit Diagram}
  \label{fig:029}
\end{figure}

Considering the circuit in Figure \ref{fig:029} in terms of \textit{energy}, the energy of
the circuit is being supplied at the battery. The energy then travels through the circuit.
At each resistor, some amount of energy is being \textit{lost} in the form of heat. Over a
capacitor, the energy is charging up the capacitor to be stored in the form of potential
energy.

Importantly, no current is flowing across a capacitor, there is only a voltage drop across
them.

The voltage supply in a circuit is referred to as "emf", denoted as $\epsilon$.

\end{document}
