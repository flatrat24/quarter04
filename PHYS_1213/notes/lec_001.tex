\documentclass[12pt]{article}

\input{../../xlatex/imports/preamble}

\title{Lecture 001}
\date{September 09, 2024}

\begin{document}

\section{Introduction to Electricity and Magentism}
\label{sec:introductionToElectricityAndMagentism}

\subsection{Electric Charge}
\label{ssec:electricCharge}

\begin{definition}{Electric Charge}
  Subatomic particles (electrons and protons) carry small discrete packets of energy, 
  in positive and negative form, which is what we call electric charge.
\end{definition}

Like charges repel, opposite charges attract.

\begin{figure}[H]
  \centering
  \begin{subfigure}[H]{0.3\textwidth}
    \centering
    \includestandalone{figures/fig_001}
    \caption{Both Positive}
    \label{fig:001}
  \end{subfigure}
  \begin{subfigure}[H]{0.3\textwidth}
    \centering
    \includestandalone{figures/fig_002}
    \caption{Positive and Negative}
    \label{fig:002}
  \end{subfigure}
  \begin{subfigure}[H]{0.3\textwidth}
    \centering
    \includestandalone{figures/fig_003}
    \caption{Both Negative}
    \label{fig:003}
  \end{subfigure}
  \caption{Attraction}
  \label{fig:attraction}
\end{figure}

Charge is an inherent property of types of particles (protons and electrons). These particles
can move between systems, but still retain there charge. This is the how and why \textbf{charge is
always conserved}.

\begin{formula}{Mass of Protons and Electrons}
  \begin{align*}
    m_p = m_n &= 1.67 \times 10^{-27}kg \\
    m_e &= 9.11 \times 10^{-31}kg
  \end{align*}
\end{formula}

When particles move between objects, it is usually said that the electrons are what move
between the object, thus changing the net charge of the objects.

\begin{figure}[H]
  \centering
  \includestandalone{figures/fig_004}
  \caption{Electrons Moving Between Objects}
  \label{fig:004}
\end{figure}

It is generally metals that become charged because they are \textbf{conductive}. When a metal
becomes charged, its charge is spread about evenly throughout the surface because the forces
of the particles cause each particle to find a position where they are equidistant from every
other particle.

\begin{figure}[H]
  \centering
  \includestandalone{figures/fig_005}
  \caption{Even Distribution}
  \label{fig:005}
\end{figure}

The standard unit of charge is the \textbf{Coulomb} (C). 

\subsection{Forces Between Particles}
\label{ssec:forcesBetweenParticles}

\begin{figure}[H]
  \centering
  \includestandalone{figures/fig_006}
  \caption{Two Protons in Space}
  \label{fig:006}
\end{figure}

The force between two particles can be modeled in terms of the charge of each object, Coulomb's Constant
$k$, and the distance between the two objects.

\begin{formula}{Coulomb's Law}
  \begin{align*}
    F &= k \frac{q_1 q_2}{r^2}
  \end{align*}
  \hrule
  \begin{align*}
    F &= \textup{Force Between Particles} \\
    q_1 &= \textup{Charge of Particle One} \\
    q_2 &= \textup{Charge of Particle Two} \\
    r &= \textup{Distance Between Particles} \\
    k &= \textup{Coulomb's Constant;\ \ } 8.99 \times 10^9 \frac{N \times m^2}{s^2}
  \end{align*}
\end{formula}

\end{document}
