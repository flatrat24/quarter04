\documentclass[12pt]{article}

\input{../../xlatex/imports/preamble}

\title{Lecture 002}
\date{September 12, 2024}

\begin{document}

\begin{wrapfigure}[5]{r}{0.3\textwidth}
  \centering
  \begin{subfigure}[H]{0.3\textwidth}
    \centering
    \includestandalone{figures/fig_008}
    \caption{Continuous Charge}
    \label{fig:008}
  \end{subfigure}
\end{wrapfigure}

When dealing with discrete charges, it's simple to find the direction of force. But when
there is a continuous charge, how can you find the direction of force it exerts on a particle?

Using \textbf{integration}, you can calculate the direction and magnitude of the force.

\begin{formula}{Force of Continuous Objects}
  \begin{equation*}
    \int_{}^{} dF = \int_{}^{} \frac{kQdq}{r^2} \,
  \end{equation*}
  \hrule
  \begin{align*}
    dF &= \textup{Change in Force from Charge} \\
    k &= \textup{Coulomb's Constant} \\
    Q &= \textup{Net Charge of Continuous Object} \\
    dq &= \textup{Change in Instantaneous Charge} \\
    r &= \textup{Distance Between Objects} \\
  \end{align*}
\end{formula}

\textbf{Electric Fields} are created by charges particles that affect other charged particles
within the field. When two particles exert forces on each other, it is because both particles
are created some field around them that in turn affects other particles around it.

\begin{formula}{Strength of an Electric Field}
  \begin{tcolorbox}[
    standard jigsaw, % Allows opacity
    colframe=fg,
    boxrule=0px,
    opacityback=0,
    sidebyside,
    lefthand width=200px,
    coltext=fg,
  ]
  \textbf{Known Values}
  \begin{align*}
    E &= \textup{Strength of Electric Field} \\
    k &= \textup{Coulomb's Constant} \\
    q &= \textup{Charge of Particle} \\
    r &= \textup{Distance from Particle}
  \end{align*}
  \tcblower
  \begin{align*}
    E &= \frac{F}{q} \\
    F &= k \frac{q_1 \cdot q_2}{r^2} \\
    E &= k \frac{q}{r^2}
  \end{align*}
  \end{tcolorbox}
\end{formula}

When drawing an electric field, the lines have some useful properties:
\begin{enumerate}
  \itemsep0em
  \item At any point in space, the directionof the lines tells you the direction of the electric
    field.
  \item The \textit{denser} the lines, the \textit{stronger} the electric field
  \item Field lines only start and end at charges
  \item Field lines never cross each other
\end{enumerate}

\vspace{12pt}
\hrule

\begin{figure}[h]
  \centering
  \includestandalone{figures/fig_010} \\
  Given the above figures, particles \textbf{b} would experience the greatest force.
\end{figure}
\hrule

\begin{figure}[H]
  \centering
  \includestandalone{figures/fig_011}
  \caption{Infinite Charged Bars}
  \label{fig:011}
\end{figure}

In a steady state, the electric field within a conductor is equal to zero. This is true because
the external electric field acting on the conductor will cause particles within the conductor
to move to an equilibrium state to cancel out the field.

\end{document}
