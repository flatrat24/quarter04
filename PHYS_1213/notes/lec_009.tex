\documentclass[12pt]{article}

%%%% GRAPHICS %%%%
\usepackage{tikz}
\usepackage{circuitikz}
\usetikzlibrary{arrows.meta}
\usepackage{tikz-3dplot}
\usepackage{graphicx}
\usepackage{pgfplots}
  \pgfplotsset{compat=1.18}
\usetikzlibrary{arrows}
\newcommand{\midarrow}{\tikz \draw[-triangle 90] (0,0) -- +(.1,0);}

%%%% FIGURES %%%%
\usepackage{subcaption}
\usepackage{wrapfig}
\usepackage{float}
\usepackage[skip=5pt, font=footnotesize]{caption}

%%%% FORMATTING %%%%
\usepackage{parskip}
\usepackage{tcolorbox}
\usepackage{ulem}
% \usepackage{fancyhdr}

%%%% TABLE FORMATTING %%%%
\usepackage{tabularray}
\UseTblrLibrary{booktabs}

%%%% MATH AND LOGIC %%%%
\usepackage{xifthen}
\usepackage{amsmath}
\usepackage{amssymb}
\usepackage{amsfonts}

%%%% TEXT AND SYMBOLS %%%%
\usepackage[T1]{fontenc}
\usepackage{textcomp}
\usepackage{gensymb}

%%%% OTHER %%%%
\usepackage{standalone}

%%%% LOGIC SYMBOLS %%%%
\newcommand*\xor{\oplus}

%%%% STYLES %%%%

% Packages
\usepackage{fullpage}
\usepackage{titlesec}
\usepackage[rgb]{xcolor}
\selectcolormodel{natural}
\usepackage{ninecolors}
\selectcolormodel{rgb}

% Colors
\definecolor{pg}{HTML}{24273A}
\definecolor{fg}{HTML}{FFFFFF}
\definecolor{bg}{HTML}{24273A}
\definecolor{re}{HTML}{d20f39}
\definecolor{gr}{HTML}{40a02b}
\definecolor{ye}{HTML}{df8e1d}
\definecolor{or}{HTML}{fe640b}
\definecolor{bl}{HTML}{1e66f5}
\definecolor{ma}{HTML}{8839ef}
\definecolor{cy}{HTML}{179299}
\definecolor{pi}{HTML}{ea76cb}

\usepackage{nameref}
\makeatletter
\newcommand*{\currentname}{\@currentlabelname}
\makeatother

\titleformat{\section}
  {\normalfont\scshape\Large\bfseries}
  {\thesection}
  {0.75em}
  {}

\titleformat{\subsection}
  {\normalfont\scshape\large\bfseries}
  {\thesubsection}
  {0.75em}
  {}

\titleformat{\subsubsection}
  {\normalfont\scshape\normalsize\bfseries}
  {\thesubsubsection}
  {0.75em}
  {}

% Formula
\newcounter{formula}[section]
\newenvironment{formula}[1]{
  \stepcounter{formula}
  \begin{tcolorbox}[
    standard jigsaw, % Allows opacity
    colframe={fg},
    boxrule=1px,
    colback=bg,
    opacityback=0,
    sharp corners,
    sidebyside,
    righthand width=18px,
    coltext={fg}
  ]
  \centering
  \textbf{\uline{#1}}
}{
  \tcblower
  \textbf{\thesection.\theformula}
  \end{tcolorbox}
}

% Definition
\newcounter{definition}[section]

\newenvironment{definition*}[1]{
  \begin{tcolorbox}[
    standard jigsaw, % Allows opacity
    colframe={fg},
    boxrule=1px,
    colback=bg,
    opacityback=0,
    sharp corners,
    coltext={fg}
  ]
  \textbf{#1 \hfill}
  \vspace{5px}
  \hrule
  \vspace{5px}
  \noindent
}{
  \end{tcolorbox}
}

\newenvironment{definition}[1]{
  \stepcounter{definition}
  \begin{tcolorbox}[
    standard jigsaw, % Allows opacity
    colframe={fg},
    boxrule=1px,
    colback=bg,
    opacityback=0,
    sharp corners,
    coltext={fg}
  ]
  \textbf{#1 \hfill \thesection.\thedefinition}
  \vspace{5px}
  \hrule
  \vspace{5px}
  \noindent
}{
  \end{tcolorbox}
}

% Example Problem
\newcounter{example}[section]
\newenvironment{example}{
  \stepcounter{example}
  \begin{tcolorbox}[
    standard jigsaw, % Allows opacity
    colframe={fg},
    boxrule=1px,
    colback=bg,
    opacityback=0,
    sharp corners,
    coltext={fg}
  ]
  \textbf{Example \hfill \thesection.\theexample}
  \vspace{5px}
  \hrule
  \vspace{5px}
  \noindent
}{
  \end{tcolorbox}
}

\tikzset{
  cubeBorder/.style=fg,
  cubeFilling/.style={fg!20!bg, opacity=0.25},
  gridLine/.style={very thin, gray},
  graphLine/.style={-latex, thick, fg},
}

\pgfplotsset{
  basicAxis/.style={
    grid,
    major grid style={line width=.2pt,draw=fg!50!bg},
    axis lines = box,
    axis line style = {line width = 1px},
  }
}

%%%% REFERENCES %%%%
\usepackage{hyperref}
\hypersetup{
  colorlinks  = true,
  linkcolor   = pi,
  anchorcolor = pi,
  citecolor   = pi,
  filecolor   = pi,
  menucolor   = pi,
  runcolor    = pi,
  urlcolor    = pi,
}

\author{Ethan Anthony}


\title{Lecture 009}
\date{October 23, 2024}

\begin{document}
\newpage
\section{Magnets}
\label{sec:magnets}

The two fundamental forces covered so far in University Physics have been gravity and
electrical forces. Magnetism is a third fundamental force.

Similar to electricity in several ways, magnetic force is an attractive force between
oppositely "charged" magnets and a repulsive force between similarly "charged" magnets.
Additionally, magnets create a magnetic field around them that behaves similarly to the
electrical field created by charged particles.

However, magnetic force and electric force are fundamentally different despite their
similarities.

\subsection{Magnetic Fields}
\label{ssec:magneticFields}
Magnetic fields are referred to as $B$ fields. But what causes them? When a compass is
brought close to a current, the compass will point in a direction due to that current.
Thus, current (moving particles) creates magnetic fields.

Additionally, magnetic fields exert force onto charged particles. However, considering a
bar magnet such as the one in Figure \ref{fig:038}, where would the current or induced
force on the particles around the magnet be?
\begin{figure}[H]
  \centering
  \includestandalone{figures/fig_038}
  \caption{Regular Bar Magnet}
  \label{fig:038}
\end{figure}
The force would be induced onto the electrons near the magnet \textbf{orbiting} within
each atom. This orbit is a tiny current within the atom, but that current is what the
magnet is inducing/reacting to. The force exerted onto the particle is called the 
\textbf{Lorentz Force}.

\begin{formula}{Lorentz Force}
  \begin{align*}
    \textup{Force on Particle} &= \textup{Particle Charge}\ (\textup{Particle Velocity} \times \textup{Magnetic Field}) \\
    F &= q(v \times B) \\
    |F| &= qvBsin(\theta)
  \end{align*}
\end{formula}

\begin{wrapfigure}[6]{r}{0.25\textwidth}
  \vspace{-15pt}
  \centering
  \includestandalone{figures/fig_039}
  \caption{}
  \label{fig:039}
\end{wrapfigure}

Consider the particle in Figure \ref{fig:039}. In the top position, it is moving to the
right within a magnetic field directed out of the page. By crossing Velocity with the
Magentic Field ($\overrightarrow{v} \times \overrightarrow{B}$), the resulting force is
pointed downwards.

As it accelerates downward, its velocity will begin to point downwards as well, eventually
reaching the position at the right with a new acceleration still towards the center. This
continues as the particle maintains a circular motion around some point.

Because work is found using the dot product of Force and Velocity
($\overrightarrow{F} \cdot \overrightarrow{v}$), and the movement of a particle within
a magnetic field is circular, the work done on a particle by a magnetic field will always
be zero.

The basic principle of calculating the Lorentz Force can also be used to calculate the
force from a magnetic field on a current-carrying wire.

\begin{formula}{Lorentz Force (Wire)}
  \begin{equation*}
    F = IL \times B \ \ \ \ \ \ \ \ |F| = ILB sin(\theta)
  \end{equation*}
\end{formula}

The circular motion experienced by the particle in Figure \ref{fig:0039} is just centripetal
motion. Thus, the radius of the circle it travels can be found using kinematics, specifically:
$F = \frac{mv^2}{r}$.
\begin{wrapfigure}[1]{r}{0.45\textwidth}
  \vspace{8pt}
  \centering
  \includestandalone{figures/fig_039}
  \caption{Centripetal Motion of a Particle in Magnetic Field}
  \label{fig:0039}
\end{wrapfigure}
\begin{formula}{Gyroradius}
  \begin{align*}
    F = \frac{mv^2}{r} &\rightarrow r = \frac{mv^2}{F} \\
    r = \frac{mv^2}{q(v \times B)} &\rightarrow r = \frac{mv^2}{qvB}
  \end{align*}
  {\Large $r = \frac{mv}{qB}$}
\end{formula}
Thus, the radius of the circle traveled scales directly with the mass and velocity of
the particle, and inversely with the charge and the strength of the B field.

\subsection{Sources of Magnetic Fields}
\label{ssec:sourcesOfMagneticFields}

Magnetic fields are always created by currents. Specifically, the magnetic field moved in
circles around the direction of the current.
\begin{figure}[H]
  \centering
  \includestandalone{figures/fig_040}
  \caption{Current-Induced Magnetic Field}
  \label{fig:040}
\end{figure}
The relationship between the direction of the
current and magnetic field are related through the right-hand rule. If the current moves
in the direction of the thumb, then the magnetic field moved in circles following the
curling of the fingers.

Since magnetic fields are created by moving charged particles, the electrons in the cloud
of an atom would also induce a field. Normally, when countless atoms are scattered, all
oriented in various directions, the net magnetic field generated would be zero.

However, such as in Figure \ref{fig:041}, when atoms are aligned, the tiny amounts of
magnetic field created by each particle will compound with the next, thus creating a
magnetic field, the magnitude of which is the sum of all the tiny ones.

\begin{figure}[H]
  \centering
  \includestandalone{figures/fig_041}
  \caption{Atoms Aligned in a Bar Magnet}
  \label{fig:041}
\end{figure}

\subsection{Strength of a Magnetic Field}
\label{ssec:strengthOfAMagneticField}

Magnetic fields (which are created by currents) exert force onto moving particles (other
currents). Thus, wires carrying currents or moving particles individually will exert some
magnetic force on each other.

\begin{figure}[H]
  \centering
  \includestandalone{figures/fig_042}
  \caption{Two Wires in Parallel}
  \label{fig:042}
\end{figure}
\vspace{-20pt}
In the case of two parallel wires with currents running in the same direction, such as the
wires in Figure \ref{fig:042}, they would both experience force towards each other. If one
of the currents were to be reversed, the force would be away from each other.

How strong are B fields actually? The strength of a B field is modeled by the 
\textbf{Biot-Savart Law}.

\begin{formula}{Biot-Savart Law}
  \begin{center}
    \begin{tblr}{rl}
      $\mu_{0}$ & constant of nature; $\frac{\mu_{0}}{4 \pi}=10^{-7}$ \\
      $q$       & charge of the particle \\
      $v$       & velocity of the particle \\
      $r$       & distance between the particle and a point in the B field \\
      $\hat{r}$ & unit vector of the direction between the particle and B point \\
      \bottomrule
    \end{tblr}
  \end{center}
  \vspace{12pt}
  \begin{equation*}
    B = \frac{\mu_{0}}{4 \pi}\frac{q(v \times \hat{r})}{r^{2}}
  \end{equation*}
\end{formula}

This formula shows how to relate the strength of a magnetic field to the velocity, charge,
and position of a particle. However, what if a continuous charge (rather than a discrete
particle) is considered?
\begin{equation*}
  B = \frac{\mu_{0}}{4 \pi}\frac{q(v \times \hat{r})}{r^{2}} \rightarrow d \overrightarrow{B} = \frac{\mu_{0}}{4 \pi}\frac{Id \overrightarrow{L} \times \hat{r}}{r^{2}}
\end{equation*}
By remodeling the formula in terms of the current in the wire and the length along the wire,
all that is left is to integrate it with the proper bounds.

\begin{example}
  How would the magnitude and direction of the B field created by a quarter-circle of wire
  at point $p$ be calculated? Using the Biot-Savart Law, and modeling it in terms of incremental
  distances over the wire, the following can be done:
  \begin{wrapfigure}[5]{r}{0.3\textwidth}
    \centering
    \includestandalone{figures/fig_043}
  \end{wrapfigure}
  \begin{equation*}
    B = \frac{\mu_{0}}{4 \pi}\frac{q(v \times \hat{r})}{r^{2}} \rightarrow d \overrightarrow{B} = \frac{\mu_{0}}{4 \pi}\frac{Id \overrightarrow{L} \times \hat{r}}{r^{2}}
  \end{equation*}
  Since $L$ and $\hat{r}$ are perpendicular, the cross product just become multiplication.
  \begin{align*}
    d \overrightarrow{B} &= \frac{\mu_{0}}{4 \pi}\frac{I}{r^{2}} d\overrightarrow{L} \\
    \int_{}^{} d \overrightarrow{B} \, &= \int_{0}^{\frac{\pi r}{2}} \frac{\mu_{0}}{4 \pi}\frac{I}{r^{2}} \, d\overrightarrow{L} \\
    \int_{}^{} d \overrightarrow{B} \, &= \frac{\mu_{0}}{4 \pi}\frac{I}{r^{2}} \int_{0}^{\frac{\pi r}{2}} d\overrightarrow{L} \\
    \int_{}^{} d \overrightarrow{B} \, &= \frac{\mu_{0}}{4 \pi}\frac{I}{r^{2}} \frac{\pi r}{2} = \frac{\mu_0I}{8r}
  \end{align*}
  Thus, $\frac{\mu_0I}{8r}$ is the magnitude of the B field at point $p$, and its direction
  can be determined using the right-hand rule. It points out of the page.
\end{example}

Following a similar process, the strength of a B field at a point $r$ distance away from
a straight and infinite wire is:
\begin{equation*}
  B = \frac{\mu_0I}{2 \pi r}
\end{equation*}

\subsection{Ampere's Law}
\label{ssec:amperesLaw}

\begin{formula}{Ampere's Law}
  \begin{equation*}
    \oint_{}^{} \overrightarrow{B} \cdot \, d \overrightarrow{L} = \mu_0 I_{through}
  \end{equation*}
\end{formula}

Given some current, an arbitrary loop can be drawn around the current. Integrating the 
B field dotted with the length of the loop gives something directly proportional to the
"enclosed" current.
\begin{equation*}
  \oint_{}^{} \overrightarrow{B} \cdot \, d \overrightarrow{L} = \mu_0 I_{through}
\end{equation*}
If the loop is made to be perfectly centered around the current and circular, then B
is constant and always parallel to the wire, so it can be removed from the integral.
\begin{align*}
  \overrightarrow{B}\oint_{0}^{2 \pi r} d \overrightarrow{L} &= \mu_0 I_{through} \\
  2 \pi rB &= \mu_0 I_{through} \\
  B &= \frac{\mu_0 I_{through}}{2 \pi r}
\end{align*}
Lastly, if no current in enclosed within the loop, then no field will be calculated by 
Ampere's Law.


\end{document}
