\documentclass[12pt]{article}

%%%% GRAPHICS %%%%
\usepackage{tikz}
\usepackage{circuitikz}
\usetikzlibrary{arrows.meta}
\usepackage{tikz-3dplot}
\usepackage{graphicx}
\usepackage{pgfplots}
  \pgfplotsset{compat=1.18}
\usetikzlibrary{arrows}
\newcommand{\midarrow}{\tikz \draw[-triangle 90] (0,0) -- +(.1,0);}

%%%% FIGURES %%%%
\usepackage{subcaption}
\usepackage{wrapfig}
\usepackage{float}
\usepackage[skip=5pt, font=footnotesize]{caption}

%%%% FORMATTING %%%%
\usepackage{parskip}
\usepackage{tcolorbox}
\usepackage{ulem}
% \usepackage{fancyhdr}

%%%% TABLE FORMATTING %%%%
\usepackage{tabularray}
\UseTblrLibrary{booktabs}

%%%% MATH AND LOGIC %%%%
\usepackage{xifthen}
\usepackage{amsmath}
\usepackage{amssymb}
\usepackage{amsfonts}

%%%% TEXT AND SYMBOLS %%%%
\usepackage[T1]{fontenc}
\usepackage{textcomp}
\usepackage{gensymb}

%%%% OTHER %%%%
\usepackage{standalone}

%%%% LOGIC SYMBOLS %%%%
\newcommand*\xor{\oplus}

%%%% STYLES %%%%

% Packages
\usepackage{fullpage}
\usepackage{titlesec}
\usepackage[rgb]{xcolor}
\selectcolormodel{natural}
\usepackage{ninecolors}
\selectcolormodel{rgb}

% Colors
\definecolor{pg}{HTML}{24273A}
\definecolor{fg}{HTML}{FFFFFF}
\definecolor{bg}{HTML}{24273A}
\definecolor{re}{HTML}{d20f39}
\definecolor{gr}{HTML}{40a02b}
\definecolor{ye}{HTML}{df8e1d}
\definecolor{or}{HTML}{fe640b}
\definecolor{bl}{HTML}{1e66f5}
\definecolor{ma}{HTML}{8839ef}
\definecolor{cy}{HTML}{179299}
\definecolor{pi}{HTML}{ea76cb}

\usepackage{nameref}
\makeatletter
\newcommand*{\currentname}{\@currentlabelname}
\makeatother

\titleformat{\section}
  {\normalfont\scshape\Large\bfseries}
  {\thesection}
  {0.75em}
  {}

\titleformat{\subsection}
  {\normalfont\scshape\large\bfseries}
  {\thesubsection}
  {0.75em}
  {}

\titleformat{\subsubsection}
  {\normalfont\scshape\normalsize\bfseries}
  {\thesubsubsection}
  {0.75em}
  {}

% Formula
\newcounter{formula}[section]
\newenvironment{formula}[1]{
  \stepcounter{formula}
  \begin{tcolorbox}[
    standard jigsaw, % Allows opacity
    colframe={fg},
    boxrule=1px,
    colback=bg,
    opacityback=0,
    sharp corners,
    sidebyside,
    righthand width=18px,
    coltext={fg}
  ]
  \centering
  \textbf{\uline{#1}}
}{
  \tcblower
  \textbf{\thesection.\theformula}
  \end{tcolorbox}
}

% Definition
\newcounter{definition}[section]

\newenvironment{definition*}[1]{
  \begin{tcolorbox}[
    standard jigsaw, % Allows opacity
    colframe={fg},
    boxrule=1px,
    colback=bg,
    opacityback=0,
    sharp corners,
    coltext={fg}
  ]
  \textbf{#1 \hfill}
  \vspace{5px}
  \hrule
  \vspace{5px}
  \noindent
}{
  \end{tcolorbox}
}

\newenvironment{definition}[1]{
  \stepcounter{definition}
  \begin{tcolorbox}[
    standard jigsaw, % Allows opacity
    colframe={fg},
    boxrule=1px,
    colback=bg,
    opacityback=0,
    sharp corners,
    coltext={fg}
  ]
  \textbf{#1 \hfill \thesection.\thedefinition}
  \vspace{5px}
  \hrule
  \vspace{5px}
  \noindent
}{
  \end{tcolorbox}
}

% Example Problem
\newcounter{example}[section]
\newenvironment{example}{
  \stepcounter{example}
  \begin{tcolorbox}[
    standard jigsaw, % Allows opacity
    colframe={fg},
    boxrule=1px,
    colback=bg,
    opacityback=0,
    sharp corners,
    coltext={fg}
  ]
  \textbf{Example \hfill \thesection.\theexample}
  \vspace{5px}
  \hrule
  \vspace{5px}
  \noindent
}{
  \end{tcolorbox}
}

\tikzset{
  cubeBorder/.style=fg,
  cubeFilling/.style={fg!20!bg, opacity=0.25},
  gridLine/.style={very thin, gray},
  graphLine/.style={-latex, thick, fg},
}

\pgfplotsset{
  basicAxis/.style={
    grid,
    major grid style={line width=.2pt,draw=fg!50!bg},
    axis lines = box,
    axis line style = {line width = 1px},
  }
}

%%%% REFERENCES %%%%
\usepackage{hyperref}
\hypersetup{
  colorlinks  = true,
  linkcolor   = pi,
  anchorcolor = pi,
  citecolor   = pi,
  filecolor   = pi,
  menucolor   = pi,
  runcolor    = pi,
  urlcolor    = pi,
}

\author{Ethan Anthony}


\title{Lecture 004}
\date{September 25, 2024}

\begin{document}

\newpage
\section{Electric Potential}
\label{sec:electricPotential}

\begin{definition}{Electric Potential}
  The amount of energy per unit of electric charge required to move a particle throughout
  an electric field.
  \begin{align*}
    "\textup{\textbf{Potential}}" &= V = \frac{U}{q} \\
    \frac{J}{C} &= \textup{ Volt} = V
  \end{align*}
  This measurement is done in \textbf{Voltage}, and often measured as a difference between
  two points. For example, $V_{ab} = V_a = V_b$ states that the potential of $a$ with
  respect to $b$ equals the work ($J$) done by the electric force when a UNIT ($C$)
  charge moves from $a$ to $b$.
\end{definition}

To calculate the potential $V$ due to a single point charge, the following formula is used:

\begin{formula}{Potential}
  \begin{equation*}
    V = \frac{1}{4 \pi \epsilon_0}\frac{q}{r}
  \end{equation*}
\end{formula}

A change in potential (a "potential difference") causes a charged object to gain or lose
potential energy. This gain or loss is on a per-charge basis.

\begin{formula}{Change in Potential}
  \begin{equation*}
    \Delta V = V_2 - V_1 = \frac{\Delta U}{q} = -\int_{}^{} \overrightarrow{E} \cdot d \overrightarrow{r} \,
  \end{equation*}
\end{formula}

\begin{wrapfigure}[11]{l}{0.5\textwidth}
  \centering
  \includestandalone{figures/fig_018}
  \caption{Particles Moving in an Electric Field}
  \label{fig:018}
\end{wrapfigure}

In Figure \ref{fig:018}, the change in potential energy of each particle is of equal magnitude,
but since one is moving from a high potential energy area to a low one, while the other moves
in the opposite direction, they would have opposite changes. This is analogous to raising an
object up versus lowering it down an equal distance.

If there is a point charge in space, where should the "zero" of the potential energy be places?

The potential difference between two points $a$ and $b$ can be modeled as:
\begin{equation*}
  \Delta V = V_b - V_a = \frac{\Delta U}{q} = -\int_{}^{} \overrightarrow{E} \cdot d \overrightarrow{r} \, = kq \left(\frac{1}{r_b} - \frac{1}{r_a}\right)
\end{equation*}
\begin{wrapfigure}[]{r}{0.3\textwidth}
  \centering
  \includestandalone{figures/fig_019}
  \caption{Point charge in space}
  \label{fig:019}
\end{wrapfigure}
If we take this as point $b$ approaches infinity, we get:
\begin{equation*}
  \lim_{b \to \infty} kq \left(\frac{1}{r_b} - \frac{1}{r_a}\right) \rightarrow kq\left(\frac{1}{\infty} - \frac{1}{r_a}\right) \rightarrow kq\left(0 - \frac{1}{r_a}\right) \rightarrow -kq\frac{1}{r_a}
\end{equation*}
Thus:
\begin{equation*}
  \Delta V = V_b - V_a = k\frac{q}{r_b}-k\frac{q}{r_a} \textup{\ \ \ \textbf{or}\ \ \ } V_a = \frac{kq}{r_a}
\end{equation*}

\subsection{Equipotential}
\label{ssec:equipotential}

In an electric field, there are lines going through the electric field that have the same
potential energy across it. In a simple electric field, such as the one in Figure \ref{fig:020},
the lines will be perfectly horizontal. Notably, equipotential lines will always be perpendicular to the electric field.
\begin{figure}[H]
  \centering
  \includestandalone{figures/fig_020}
  \caption{Equipotential Lines}
  \label{fig:020}
\end{figure}
In the case of a point charge, such as that in Figure \ref{fig:021}, the equipotential lines
will form concentric circles around the particle.
\begin{figure}[H]
  \centering
  \includestandalone{figures/fig_021}
  \caption{Equipotential Around Point Charge}
  \label{fig:021}
\end{figure}

\subsection{Electronvolt}
\label{ss3c:electronvolt}

When measuring the energy change of a particle as it moves through an electric field, rather
than measuring in $qV$, a new unit to describe energy exists, the electronvolt. A single
electronvolt is defined as $1.6 \times 10^-{19} J$

\begin{definition}{Electronvolt}
  $1 eV$ is defined as the amount of energy change for a particle of charge "$e$" moving 
  through a $1V$ potential drop.
\end{definition}

\end{document}
